\chapter{OOP:宏伟蓝图\label{ch26}}
\section{概览 OOP}
\subsection{属性继承搜索}
在 Python 对象模型中,类和通过类产生的实例是两种不同的对象模型:
\begin{description}
    \item[类]类是实例工厂。类的属性提供了行为(数据以及函数),所有从类产生的实例都继承了该类的属性。
    \item[实例]代表程序领域中具体的元素。实例的属性记录了每个实例自己的数据。
\end{description}
\subsection{方法调用}
方法可以通过实例 bob.giveRaise() 或类 Employee.giveRaise(bob) 来调用,这两种形式都可以在我们的脚本中发挥作用。 这些调用还说明了 OOP 中的两个关键思想:运行 bob.giveRaise() 方法调用,Python:
\begin{enumerate}
    \item 通过继承搜索从 bob 中查找 GiveRaise
    \item 将 bob 传递给位于特殊 self 参数中的 GiveRaise 函数
\end{enumerate}

当你调用 Employee.giveRaise(bob) 时,你只需自己执行这两个步骤。从技术上讲,此描述是默认情况(Python 还有其他方法类型),但它适用于用该语言编写的绝大多数 OOP 代码。
\subsection{编写树类}
从操作的角度来看,当 def 出现在类的内部时,通常称为方法。而且会自动接收第一个特殊参数(按照惯例称为 self),这个参数提供了被处理的实例的引用。所有你自己向方法中传入的参数都被赋给了 self 后面的参数。