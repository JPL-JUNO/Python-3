\chapter{迭代和推导}
\section{迭代器:初次探索}
出于明确性,这里倾向于使用术语可迭代对象来指代有一个支持 iter 调用的对象,使用术语迭代器来指代一个(iter 调用为传入的可迭代对象返回的)支持 next(I) 调用的对象。
\subsection{迭代协议:文件迭代器}
所有带有 \_\_next\_\_ 方法的对象会前进到下一个结果,而在一系列结果的末尾时,则会引发 StopIteration 异常,这种对象在 Python 中也被称为迭代器。任何这类对象也能以 for 循环或其他迭代工具遍历,因为所有迭代工具内部工作起来都是在每次迭代中调用 \_\_next\_\_,并且捕捉 StopIteration 异常来确定何时离开。

while 循环会比基于迭代器的 for 循环运行得更慢,因为迭代器在 Python 中是以 C 语言的速度运行的,而 while 循环版本则是通过Python 虚拟机运行 Python 字节码的。任何时候,我们把 Python 代码换成 C 程序代码,速度都应该会变快。然而,并非绝对如此。
\subsection{手动迭代:iter 和 next}
迭代协议还有一点值得注意。当 for 循环开始时,会通过它传给 iter 内置函数,以便从可迭代对象中获得一个迭代器,返回的对象含有需要的 next 方法。
\subsubsection{完整得迭代协议}
作为更正式的定义,\autoref{fig14-1} 描绘了这个完整的迭代协议,Python 中的每个迭代工具都使用它,并受到各种对象类型的支持。 它实际上基于两个对象,由迭代工具在两个不同的步骤中使用:
\begin{itemize}
    \item 您请求迭代的可迭代对象,其 \_\_iter\_\_ 由 iter 运行
    \item  由迭代器返回的迭代器对象,在迭代过程中实际产生值,其 \_\_next\_\_ 由 next 运行,并在完成产生结果时引发 StopIteration
\end{itemize}
\figures{fig14-1}{Python 迭代协议,由 for 循环、推导式、映射等使用,并受文件、列表、字典、\autoref{ch20} 的生成器等支持。 有些对象既是迭代上下文又是可迭代对象,例如生成器表达式和 3.X 风格的某些工具(例如 map 和 zip)。 有些对象既是可迭代的又是迭代器,为 iter() 调用返回自身,这就是一个无操作。}

文件对象就是自己的迭代器。由于文件只支持一次迭代(它们不能通过方向查找来支持多重扫描),文件有自己的 __next__ 方法。

列表以及很多其他的内置对象,不是自身的迭代器,因为它们支持多次打开迭代器,例如,嵌套循环中可能在不同位置有多次迭代。对这样的对象,我们必须调用 iter 来启动迭代。