\chapter{类代码编写基础\label{ch27}}
类有三个主要的不同之处。从最底层来看,类几乎就是命名空间,很像 \label{p5} 研究过的模块。但是,和模块不同的是,类也支持多个对象的产生、命名空间继承以及运算符重载。
\section{类产生多个实例对象}
要了解多个对象的概念是如何工作的,得先了解 Python 的 OOP 模型中的两种对象:类对象和实例对象。类对象提供默认行为,是实例对象的工厂。实例对象是程序处理的实际对象:各自都有独立的命名空间,但是继承(可自动存取)创建该实例的类中的变量名。类对象来自于语句,而实例来自于调用。每次调用一个类,就会得到这个类的新的实例。
\subsection{类对象提供默认行为}
以下是 Python 类主要特性的要点:
\begin{description}
    \item[class 语句创建类对象并将其赋值给变量名]就像函数 def 语句,Python class语句也是可执行语句。执行时,会产生新的类对象,并将其赋值给 class 头部的变量名。此外,就像def  应用,class 语句一般是在其所在文件导入时执行的。
    \item[class 语句内的赋值语句会创建类的属性]就像模块文件一样,class 语句内的顶层的赋值语句(不是在 def 之内)会产生类对象中的属性。从技术角度来讲,class 语句的作用域会变成类对象的属性的命名空间,就像模块的全局作用域一样。执行 class 语句后,类的属性可由变量名点号运算获取 object.name。
    \item[类属性提供对象的状态和行为]类对象的属性记录状态信息和行为,可由这个类所创建的所有实例共享。
\end{description}
\subsection{实例对象是具体的元素}
以下是类的实例内含的重点概要:
\begin{description}
    \item[像函数那样调用类对象会创建新的实例对象]每次类调用时,都会建立并返回新的实例对象。实例代表了程序领域中的具体元素。
    \item[每个实例对象继承类的属性并获得了自己的命名空间]由类所创建的实例对象是新命名空间。一开始是空的,但是会继承创建该实例的类对象内的属性。
    \item[在方法内对 self 属性做赋值运算会产生每个实例自己的属性]在类方法函数内,第一个参数(按惯例称为 self)会引用正处理的实例对象。对 self 的属性做赋值运算,会创建或修改实例内的数据,而不是类的数据。
\end{description}
\section{类通过继承进行定制}
除了作为工厂来生成多个实例对象之外,类也可引入新组件(子类)来进行修改,而不对现有组件进行原地的修改。

在 Python 中,实例从类中继承,而类继承于超类。以下是属性继承机制的核心观点:
\begin{itemize}
    \item 超类列在了类开头的括号中
    \item 类从其超类中继承属性
    \item 实例会继承所有可访问类的属性
    \item 每个 object.attribute 都会开启新的独立搜索
    \item 逻辑的修改是通过创建子类,而不是修改超类
\end{itemize}
\section{类可以截获 Python 运算符}
运算符重载就是让用类写成的对象,可截获并响应用在内置类型上的运算:加法、切片、打印和点号运算等。这只是自动分发机制:表达式和其他内置运算流程要经过类的实现来控制。这里也和模块没有什么相似之处:模块可以实现函数调用,而不是表达式的行为。

\begin{itemize}
    \item 以双下划线命名的方法是特殊钩子。Python 运算符重载的实现是提供特殊命名的方法来拦截运算。Python 语言替每种运算和特殊命名的方法之间,定义了固定不变的映射关系。
    \item 当实例出现在内置运算时,这类方法会自动调用。
    \item 类可覆盖多数内置类型运算。
    \item 运算符覆盖方法没有默认值,而且也不需要。
    \item 新式类有一些默认的运算符重载方法,但是不属于常见运算。
    \item 运算符可让类与 Python 的对象模型相集成。
\end{itemize}