\chapter{元编程\label{ch09}}
任何时候当需要创建高度重复的代码(或者需要复制粘贴源代码)时,通常都需要寻找一个更加优雅的解决方案。元编程的主要目标是创建函数和类,并用它们来操纵代码(比如说修改、生成或者包装已有的代码)。Python 中基于这个目的的主要特性包括装饰器、类装饰器以及元类。但是,还有许多其他有用的主题——包括对象签名、用 exec()来执行代码以及检查函数和类的内部结构——也进入了我们的视野。
\section{给函数添加一个包装}
我们想给函数加上一个包装层(wrapper layer)以添加额外的处理(例如,记录日志、计时统计)。如果需要用额外的代码对函数做包装,可以定义一个装饰器函数。
\section{编写装饰器时如何保存函数的元数据}
每当定义一个装饰器时,应该总是记得为底层的包装函数添加 functools 库中的 \verb|@wraps| 装饰器。编写装饰器的一个重要部分就是拷贝装饰器的元数据。如果忘记使用 \verb|@wraps|,就会发现被包装的函数丢失了所有有用的信息。
\section{对装饰器进行解包装}
\section{定义一个可接受参数的装饰器}
\section{定义一个属性可由用户修改的装饰器}
