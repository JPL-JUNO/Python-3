\chapter{multiprocessing 模块\label{ch09}}
\section{multiprocessing 模块综述}
The multiprocessing module provides a low-level interface to process- and thread-based parallelism. Its main components are as follows:
\begin{description}
    \item[进程]一个当前进程的派生(forked)拷贝,创建了一个新的进程标识符,并且任务在操作系统中以一个独立的子进程运行。你可以启动并查询进程的状态并给它提供一个目标方法来运行。
    \item[池]包装了进程或线程。在一个方便的工作者线程池中共享一块工作并返回聚合的结果。
    \item[队列]一个先进先出(FIFP)的队列允许多个生产者和消费者。
    \item[Pipe]A uni- or bidirectional communication channel between two processes.
    \item[Manager]A high-level managed interface to share Python objects between processes.
    \item[ctypes]允许在进程派生(forked)后,在父子进程间共享原生数据类型(例如,整型数、浮点数和字节数)。
    \item[同步原语]锁和信号量在进程间同步控制流。
\end{description}

\section{使用多进程和多线程来估算 $\pi$}
\subsection{Using numpy}
The main reason that numpy is faster than pure Python when solving the same problem is that numpy is creating and manipulating the same object types at a very low level in contiguous blocks of RAM, rather than creating many higher-level Python objects that each require individual management and addressing.