\chapter{日期和时间}
不同于 int、float 和 str,Python 没有包含对应日期和时间的原生类型,不过提供了 3 个相应的模块,可以采用多种表示来管理日期和时间值。

time 模块由底层 C 库提供与事件相关的函数。它包含一些函数,可以用于获取时钟时间和处理器运行时间,还提供了多种表示来管理日期和时间值。

datetime 模块为日期、时间以及日期时间值提供了一个更高层接口。datetime 中的类支持算术、比较和时区设置。

calendar 模块可以创建周、月和年的格式化表示。它还可以用来计算重复事件,给定日期是星期几,以及其他基于日历的值。
\section{time: 时钟日期}
The time module provides access to several types of clocks, each useful for different purposes. The standard system calls such as \verb|time()| report the system “wall clock” time. The \verb|monotonic()| clock can be used to measure elapsed time in a long-running process because it is guaranteed never to move backward, even if the system time is changed. For performance testing, \verb|perf_counter()| provides access to the clock with the highest available resolution, which makes short time measurements more accurate. The CPU time is available through \verb|clock()|, and \verb|process_time()| returns the combined processor time and system time.

\section{datetime: 日期和时间值管理}
\begin{table}
    \caption{strptime/strftime 格式化代码}
    \centering
    \begin{tabular}{rll}
        \hline
        Symbol & Meaning                                             & Example                    \\
        \hline
        \%a    & 缩写的星期                                               & 'Wed'                      \\
        \%A    & 完整的星期                                               & 'Wednesday'                \\
        \%w    & Weekday number: 0 (Sunday) through 6 (Saturday)     & '3'                        \\
        \%d    & Day of the month (zero padded)                      & '13'                       \\
        \%b    & Abbreviated month name                              & 'Jan'                      \\
        \%B    & Full month name                                     & 'January'                  \\
        \%m    & Month of the year                                   & '01'                       \\
        \%y    & Year without century                                & '16'                       \\
        \%Y    & Year with century                                   & '2016'                     \\
        \%H    & Hour from 24-hour clock                             & '17'                       \\
        \%I    & Hour from 12-hour clock                             & '05'                       \\
        \%p    & AM/PM                                               & 'PM'                       \\
        \%M    & Minutes                                             & '00'                       \\
        \%S    & Seconds                                             & '00'                       \\
        \%f    & Microseconds                                        & '000000'                   \\
        \%z    & UTC offset for time zone–aware objects              & '-0500'                    \\
        \%Z    & Time zone name                                      & 'EST'                      \\
        \%j    & Day of the year                                     & '013'                      \\
        \%W    & Week of the year                                    & '02'                       \\
        \%c    & Date and time representation for the current locale & 'Wed Jan 13 17:00:00 2016' \\
        \%x    & Date representation for the current locale          & '01/13/16'                 \\
        \%X    & Time representation for the current locale          & '17:00:00'                 \\
        \%\%   & A literal \% character                              & '\%'                       \\
        \hline
    \end{tabular}

\end{table}
\section{calendar: 处理日期}