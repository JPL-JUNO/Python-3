\chapter{文本\label{01}}
\section{re: 正则表达式}
\subsection{模式语法}
\subsubsection{重复}
模式中有 5 中表示重复的方法
\begin{enumerate}
    \item 元字符 *,则表示重复 0 次或多次(运行一个模式重复 0 次是指这个模式即使不出现也可以出现匹配)
    \item + ,那么模式必须至少出现 1 次才能匹配
    \item ? 表示出现 0 次或 1 次
    \item 如果需要指定出现次数,需要在模式后面使用 $\{m\}$,这里 $m$ 是模式应重复的次数
    \item 如果要允许一个可变但有限的重复次数,那么可以使用 $\{m, n\}$,这里 $m$ 是最小重复次数,$n$ 是最大重复次数。如果省略 $n$ (${m, }$) 则表示值必须知道出现 $m$ 次,但没有最大限制
\end{enumerate}
\subsubsection{字符集}
A character set is a group of characters, any one of which can match at that point in the pattern. For example, [ab] would match either a or b.

A character set can also be used to exclude specific characters. The carat $\hat{}$ means to look for characters that are not in the set following the carat.

As character sets grow larger, typing every character that should (or should not) match becomes tedious. A more compact format using \textbf{character ranges} can be used to define a character set to include all of the contiguous characters between the specified start and stop points.

As a special case of a character set, the meta-character dot, or period (.), indicates that the pattern should match any single character in that position.

\subsubsection{转义码}
\begin{table}
    \centering
    \caption{Regular Expression Escape Codes}
    \label{tbl1-1}
    \begin{tabular}{ll}
        \hline
        Code & Meaning \\
        \hline
        \\d   & A digit                                \\
        \\D   & A non-digit                            \\
        \\s   & Whitespace (tab, space, newline, etc.) \\
        \\S   & Non-whitespace                         \\
        \\w   & Alphanumeric                           \\
        \\W   & Non-alphanumeric                       \\
        \hline
    \end{tabular}

\end{table}
\subsubsection{锚定}
使用锚定指令指定模式在输入文本中的相对位置。
\begin{table}
    \centering
    \caption{正则表达式锚定码}
    \label{tbl1-2}
    \begin{tabular}{ll}
        \hline
        锚定码      & 含义       \\
        \hline
        $\hat{}$ & 字符串或行的开头 \\
        \$       & 字符串或行末尾  \\
        \\A       & 字符串开头        \\
        \\Z       & 字符串末尾        \\
        \\b       & 单词开头或末尾的空串   \\
        \\B       & 不在单词开头或末尾的空串 \\
        \hline
    \end{tabular}

\end{table}