\chapter{Unicode 文本和字节序列\label{Ch04}}
\section{字节概要}
bytes 或 bytearray 对象的各个元素是介于 0$\sim$255(含)之间的整数。然而,二进制序列的切片始终是同一类型的二进制序列,包括长度为 1 的切片。
\section{处理解码和编码问题}
虽然有个一般性的 \verb|UnicodeError| 异常,但是报告错误时几乎都会指明具体的异常:\verb|UnicodeEncodeError|(把字符串转换成二进制序列时)或 \verb|UnicodeDecodeError|(把二进制序列转换成字符串时)。如果源码的编码与预期不符,加载 Python 模块时还可能抛出 \verb|SyntaxError|。
\subsection{处理 UnicodeEncodeError}
多数非 UTF 编解码器只能处理 Unicode 字符的一小部分子集。把文本转换成字节序列时,如果目标编码中没有定义某个字符,那就会抛出 UnicodeEncodeError 异常,除非把 errors 参数传给编码方法或函数,对错误进行特殊处理。

ASCII is a common subset to all the encodings that I know about, therefore encoding should always work if the text is made exclusively of ASCII characters. Python 3.7 added a new boolean method \verb|str.isascii()| to check whether your Unicode text is 100\% pure ASCII. If it is, you should be able to encode it to bytes in any encoding without raising UnicodeEncodeError.

\subsection{处理UnicodeDecodeError}
不是每一个字节都包含有效的 ASCII 字符,也不是每一个字符序列都是有效的 UTF-8 或 UTF-16。因此,把二进制序列转换成文本时,如果假设是这两个编码中的一个,遇到无法转换的字节序列时会抛出 UnicodeDecodeError。

\section{Unicode 文本排序}
Python 比较任何类型的序列时,会一一比较序列里的各个元素。对字符串来说,比较的是码位。可是在比较非 ASCII 字符时,得到的结果不尽如人意。

在 Python 中,非 ASCII 文本的标准排序方式是使用 locale.strxfrm 函数,根据 locale 模块的文档,这个函数会“把字符串转换成适合所在区域进行比较的形式”。使用 locale.strxfrm 函数之前,必须先为应用设定合适的区域设置,还要祈祷操作系统支持这项设置。
\subsection{}

\section{Unicode 数据库}
\subsection{字符的数值意义}
