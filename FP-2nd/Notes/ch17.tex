\chapter{迭代器、生成器与经典协程\label{ch17}}
\section{序列可以迭代的原因:iter 函数}
解释器需要迭代对象 x 时,会自动调用 iter(x)。
内置的 iter 函数执行以下操作:
\begin{enumerate}
    \item 检查对象是否实现了 \verb|__iter__| 方法,如果实现了就调用它,获取一个迭代器。
    \item 如果没有实现 \verb|__iter__| 方法,但是实现了 \verb|__getitem__| 方法,Python 会创建一个迭代器,尝试按顺序(从索引 0 开始)获取元素。
    \item 如果尝试失败,Python 抛出 TypeError 异常,通常会提示“C object is not iterable”(C对象不可迭代),其中 C 是目标对象所属的类。
\end{enumerate}

这是鸭子类型(duck typing)的极端形式:不仅实现了特殊方法的 \verb|__iter__| 的对象被视作可迭代对象,实现了 \verb|__getitem__| 方法的对象也被视作可迭代对象。

If a class provides \verb|__getitem__|, the iter() built-in accepts an instance of that class as iterable and builds an iterator from the instance. Python’s iteration machinery will call \verb|__getitem__| with indexes starting from 0, and will take an IndexError as a signal that there are no more items. Although \verb|__getitem__| could provide items, it is not recognized as such by an isinstance against abc.Iterable.

从 Python 3.10 开始,检查对象 x 能否迭代,最准确的方法是调用 iter() 函数,如果不可以迭代,则处理 TypeError 异常。This is more accurate than using isinstance(x,abc.Iterable), because iter(x) also considers the legacy \verb|__getitem__| method, while the Iterable ABC does not.
\section{可迭代对象与迭代器}
使用 iter 内置函数可以获取迭代器的对象。如果对象实现了能返回迭代器的 \verb|__iter__| 方法,那么对象就是可迭代的。序列都可以迭代;实现了 \verb|__getitem__| 方法,而且其参数是从 0 开始的索引,这种对象也可以迭代。

我们要明确可迭代的对象和迭代器之间的关系:\textbf{Python 从可迭代的对象中获取迭代器}。

Python 标准的迭代器接口有以下两个方法:
\begin{enumerate}
    \item \verb|__next__| 返回序列中的下一项,如果没有项了,则抛出 StopIteration 。
    \item \verb|__iter__| 放回 self,以便在预期内可迭代对象的地方使用迭代器。
\end{enumerate}

\figures{fig17-1}{
    Iterable 和 Iterator 抽象基类。以斜体显示的是抽象方法。具体的 \texttt{Iterable.\_\_iter\_\_} 方法应该返回一个 Iterator 实例。具体的 Iterator 类必须实现 \texttt{\_\_next\_\_} 方法。\texttt{Iterator.\_\_iter\_\_} 方法直接返回实例本身
}
\subsection{不要把可迭代对象变成迭代器}
构建可迭代的对象和迭代器时经常会出现错误,原因是混淆了二者。要知道,可迭代的对象有个 \verb|__iter__| 方法,每次都实例化一个新的迭代器;而迭代器要实现 \verb|__next__| 方法,返回单个元素,此外还要实现 \verb|__iter__| 方法,返回迭代器本身。

因此,迭代器可以迭代,但是可迭代的对象不是迭代器。

\subsection{生成器的工作原理}
只要 Python 函数的定义体中有 yield 关键字,该函数就是生成器函数。调用生成器函数时,会返回一个生成器对象。也就是说,生成器函数是生成器工厂。

生成器函数会创建一个生成器对象,包装生成器函数的定义体。把生成器传给 next(...) 函数时,生成器函数会向前,执行函数定义体中的下一个 yield 语句,返回产出的值,并在函数定义体的当前位置暂停。最终,函数的定义体返回时,外层的生成器对象会抛出 StopIteration 异常——这一点与迭代器协议一致。


应该这样说:函数返回值;调用生成器函数返回生成器;生成器产出或生成值。生成器不会以常规的方式“返回”值:生成器函数定义体中的 return 语句会触发生成器对象抛出 StopIteration 异常。如果生成器中由 return x 语句,则调用方能从 StopIteration 异常中获取 x 的值,但是我们往往把这个操作交给 yield from 句法。