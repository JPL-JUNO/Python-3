\chapter{函数是一等对象}
在 Python 中,函数是一等对象。编程语言理论家把“一等对象”定义为满足下述条件的程序实体:
\begin{itemize}
    \item 在运行时创建
    \item 能赋值给变量或数据结构中的元素
    \item 能作为参数传给函数
    \item 能作为函数的返回结果
\end{itemize}
\section{把函数视作对象}
\section{高阶函数}
接受函数为参数,或者把函数作为结果返回的函数是\textbf{高阶函数}(higher-order function)。
\section{匿名函数}
lambda 关键字在 Python 表达式内创建匿名函数。

然而,Python 简单的句法限制了 lambda 函数的定义体只能使用纯表达式。换句话说,lambda 函数的定义体中不能赋值,也不能使用 while 和 try 等 Python 语句。可以有新出现的 := 赋值表达式。But if you need
it, your lambda is probably too complicated and hard to read, and it should be refactored into a regular function using def.
\section{可调用对象}
除了用户定义的函数,调用运算符(即 ())还可以应用到其他对象上。如果想判断对象能
否调用,可以使用内置的 callable() 函数。Python 数据模型文档列出了 9 种可调用对象。
\begin{description}
    \item[用户定义的函数] 使用 def 语句或 lambda 表达式创建。
    \item[内置函数] 使用 C 语言(CPython)实现的函数,如 len 或 time.strftime。
    \item[内置方法] 使用 C 语言实现的方法,如 dict.get。
    \item[方法] 在类的定义体中定义的函数。
    \item[类] 调用类时会运行类的 \verb|__new__| 方法创建一个实例,然后运行 \verb|__init__| 方法,初始化实例,最后把实例返回给调用方。因为 Python 没有 new 运算符,所以调用类相当于调用函数。(通常,调用类会创建那个类的实例,不过覆盖 \verb|__new__| 方法的话,也可能出现其他行为。)
    \item[类的实例] 如果类定义了 \verb|__call__| 方法,那么它的实例可以作为函数调用。
    \item[生成器函数] 使用 yield 关键字的函数或方法。调用生成器函数返回的是生成器对象。
\end{description}
\section{用户定义的可调用类型}
不仅 Python 函数是真正的对象,任何 Python 对象都可以表现得像函数。为此,只需实现实例方法 \verb|__call__|。
\section{支持函数式编程的包}
\subsection{operator 模块}
\subsection{使用 functools.partial 冻结参数}
partial 它可以根据提供的一个可调用对象产生一个新的可调用对象,为原可调用对象的某些参数绑定预定的值。使用这个函数可以把接受一个或多个参数的函数改造成需要更少参数的回调的 API。
