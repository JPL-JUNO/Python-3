\chapter{}
\section{用正则表达式查找文本模式}
向 re.compile()传入一个字符串值,表示正则表达式,它将返回一个 Regex 模式
对象(或者就简称为 Regex 对象)。

Regex 对象的 search()方法查找传入的字符串,寻找该正则表达式的所有匹配。如果字符串中没有找到该正则表达式模式,search()方法将返回 None。如果找到了该模式,search()方法将返回一个 Match 对象。Match 对象有一个 group()方法,它返回被查找字符串中实际匹配的文本。


虽然在 Python 中使用正则表达式有几个步骤,但每一步都相当简单。
\begin{enumerate}
    \item 用 import re 导入正则表达式模块。
    \item 用 re.compile()函数创建一个 Regex 对象(记得使用原始字符串)。
    \item 向 Regex 对象的 search()方法传入想查找的字符串。它返回一个 Match 对象。
    \item 调用 Match 对象的 group()方法,返回实际匹配文本的字符串。
\end{enumerate}
\section{用正则表达式匹配更多模式}
\subsection{利用括号分组}
\begin{tcolorbox}
    我觉得可以不翻译为分组,而是理解为组件。
\end{tcolorbox}
添加括号将在正则表达式中创建“分组”:然后可以使用 group()匹配对象方法,从一个分组中获取匹配的文本。正则表达式字符串中的第一对括号是第 1 组。第二对括号是第 2 组。向 group()匹配对象方法传入整数 1 或 2,就可以取得匹配文本的不同部分。向 group()方法传入 0 或不传入参数,将返回整个匹配的文本。\textbf{只找第一个不会多找}

如果想要一次就获取所有的分组,请使用 groups()方法。

括号在正则表达式中有特殊的含义,但是如果你需要在文本中匹配括号,怎么办?在这种情况下,就需要用斜杠对 ( 与 ) 进行字符转义。

在正则表达式中,以下字符具有特殊含义:
\begin{verbatim}
    . ^ $ * + ? { } ( ) [ ] \ |
\end{verbatim}

如果要检测包含这些字符的文本模式,那么就需要用斜杠对它们进行转义。

要确保正则表达式中,没有将转义的括号 ( 与 ) 误作为 ( 与 )。如果你收到类似 missing 或 unbalanced parenthesis 的错误信息,则可能是忘记了为分组添加转义的右括号。
\subsection{用管道匹配多个分组}
字符 $|$ 称为“管道”。希望匹配许多表达式中的一个时,就可以使用它。例如,正则表达式 r`Batman$|$Tina Fey' 将匹配 `Batman' 或 `Tina Fey'。

如果 Batman 和 Tina Fey 都出现在被查找的字符串中,第一次出现的匹配文本,将作为 Match 对象返回。

\begin{tcolorbox}
    利用 findall()方法,可以找到“所有”匹配的地方。
\end{tcolorbox}
\subsection{用问号实现可选匹配}
有时候,想匹配的模式是可选的。就是说,不论这段文本在不在,正则表达式都会认为匹配。字符 ? 表明它前面的分组在这个模式中是可选的。你可以认为 ? 是在说,“匹配这个问号之前的分组零次或一次”。
\subsection{用星号匹配零次或多次}
*(称为星号)意味着“匹配零次或多次”,即星号之前的分组,可以在文本中出现任意次。它可以完全不存在,或一次又一次地重复。
\subsection{用加号匹配一次或多次}
* 意味着“匹配零次或多次”,+(加号)则意味着“匹配一次或多次”。星号不要求分组出现在匹配的字符串中,但加号不同,加号前面的分组必须“至少出现一次”。这不是可选的。
\subsection{用花括号匹配特定次数}
如果想要一个分组重复特定次数,就在正则表达式中该分组的后面,跟上花括号包围的数字。除了一个数字,还可以指定一个范围,即在花括号中写下一个最小值、一个逗号和一个最大值。也可以不写花括号中的第一个或第二个数字,不限定最小值或最大值。
\section{贪心和非贪心匹配}
Python 的正则表达式默认是“贪心”的,这表示在有二义的情况下,它们会尽可能匹配最长的字符串。花括号的“非贪心”版本匹配尽可能最短的字符串,即在结束的花括号后跟着一个问号。

\begin{tcolorbox}
    请注意,问号在正则表达式中可能有两种含义:声明非贪心匹配或表示可选的分组。这两种含义是完全无关的。
\end{tcolorbox}
\section{findall()方法}
除了 search 方法外,Regex 对象也有一个 findall() 方法。search() 将返回一个Match 对象,包含被查找字符串中的“第一次”匹配的文本,而 findall() 方法将返回一组字符串,包含被查找字符串中的所有匹配。

另一方面,findall()不是返回一个 Match 对象,而是返回一个字符串列表,前提是正则表达式没有分组(组件)。列表中的每个字符串都是一段被查找的文本,它匹配该正则表达式。
\section{字符分类}
有许多这样的“缩写字符分类”,如 \autoref{tbl7-1} 所示。
\begin{table}
    \centering
    \caption{常用字符分类的缩写代码}
    \label{tbl7-1}
    \begin{tabular}{ll}
        \hline
        缩写字符分类           & 表示                           \\
        \hline
        \textbackslash d & 0 到 9 的任何数字                  \\
        \textbackslash D & 除 0 到 9 的数字以外的任何字符           \\
        \textbackslash w & 任何字母、数字或下划线字符(可以认为是匹配“单词”字符) \\
        \textbackslash W & 除字母、数字和下划线以外的任何字符            \\
        \textbackslash s & 空格、制表符或换行符(可以认为是匹配“空白”字符)    \\
        \textbackslash S & 除空格、制表符和换行符以外的任何字符           \\
        \hline
    \end{tabular}
\end{table}
\section{建立自己的字符分类}
你可以用方括号定义自己的字符分类。也可以使用短横表示字母或数字的范围。例如,字符分类 [a-zA-Z0-9] 将匹配所有小写字母、大写字母和数字。请注意,在方括号内,普通的正则表达式符号不会被解释。这意味着,你不需要前面加上反斜杠转义。

\textbf{注意:这是逻辑上或的关系}

通过在字符分类的左方括号后加上一个插入字符(\^{})就可以得到“非字符类”(逻辑非)。

\section{插入字符和美元字符}
可以在正则表达式的开始处使用插入符号(\^{}),表明匹配必须发生在被查找文本开始处。类似地,可以再正则表达式的末尾加上美元符号(\$),表示该字符串必须以这个正则表达式的模式结束。可以同时使用 \^{} 和 \$,表明整个字符串必须匹配该模式,也就是说,只匹配该字符串的某个子集是不够的。
\section{通配字符}
在正则表达式中,.(句点)字符称为“通配符”。它匹配除了换行之外的所有字符。要记住,句点字符只匹配一个字符。
\subsection{用点-星匹配所有字符}
句点字符表示“除换行外所有单个字符”,星号字符表示“前面字符出现零次或多次”。

点-星使用“贪心”模式:它总是匹配尽可能多的文本。要用“非贪心”模式匹配所有文本,就使用点-星和问号。像和大括号一起使用时那样,问号告诉 Python 用非贪心模式匹配。
\subsection{用句点字符匹配换行}
点-星将匹配除换行外的所有字符。通过传入 re.DOTALL 作为 re.compile() 的第二个参数,可以让句点字符匹配所有字符,包括换行字符。

\section{正则表达式符号复习}
\begin{itemize}
    \item ?匹配零次或一次前面的分组。
    \item  *匹配零次或多次前面的分组。
    \item  +匹配一次或多次前面的分组。
    \item  \{n\}匹配 n 次前面的分组。
    \item  \{n,\}匹配 n 次或更多前面的分组。
    \item  \{,m\}匹配零次到 m 次前面的分组。
    \item  \{n,m\}匹配至少 n 次、至多 m 次前面的分组。
    \item  \{n,m\}? 或 *? 或 +? 对前面的分组进行非贪心匹配。
    \item  \^{}spam 意味着字符串必须以 spam 开始。
    \item  spam\$ 意味着字符串必须以 spam 结束。
    \item  . 匹配所有字符,换行符除外。
    \item \lstinline![abc]! 匹配方括号内的任意字符(诸如 a、b 或 c)。
    \item  \lstinline![^abc]! 匹配不在方括号内的任意字符。
    \item  \textbackslash d、\textbackslash w 和 \textbackslash s 分别匹配数字、单词和空格。
    \item  \textbackslash D、\textbackslash W 和 \textbackslash S 分别匹配出数字、单词和空格外的所有字符。
\end{itemize}
\section{不区分大小写的匹配}
有时候你只关心匹配字母,不关心它们是大写或小写。要让正则表达式不区分大小写,可以向 re.compile()传入 re.IGNORECASE 或 re.I,作为第二个参数。

\section{用 sub()方法替换字符串}
正则表达式不仅能找到文本模式,而且能够用新的文本替换掉这些模式。Regex 对象的 sub()方法需要传入两个参数。第一个参数是一个字符串,用于取代发现的匹配。第二个参数是一个字符串,即正则表达式。sub()方法返回替换完成后的字符串。

有时候,你可能需要使用匹配的文本本身,作为替换的一部分。在 sub()的第一个参数中,可以输入\textbackslash 1、\textbackslash 2、\textbackslash 3\dots。表示“在替换中输入分组 1、2、3\dots的文本”。

\section{管理复杂的正则表达式}
如果要匹配的文本模式很简单,正则表达式就很好。但匹配复杂的文本模式,可能需要长的、费解的正则表达式。你可以告诉 re.compile(),忽略正则表达式字符串中的空白符和注释,从而缓解这一点。要实现这种详细模式,可以向 re.compile()传入变量 re.VERBOSE,作为第二个参数。

你可以将正则表达式放在多行中,并加上注释正则表达式字符串中的注释规则,与普通的 Python 代码一样:\# 符号和它后面直到行末的内容,都被忽略。而且,表示正则表达式的多行字符串中,多余的空白字符也不认为是要匹配的文本模式的一部分。这让你能够组织正则表达式,让它更可读。

\section{组合使用 re.IGNOREC ASE、re.DOTALL 和 re.VERBOSE}
如果你希望在正则表达式中使用 re.VERBOSE 来编写注释,还希望使用 re.IGNORECASE 来忽略大小写,该怎么办?遗憾的是,re.compile()函数只接受一个值作为它的第二参数。可以使用管道字符($|$)将变量组合起来,从而绕过这个限制。管道字符在这里称为“按位或”操作符。