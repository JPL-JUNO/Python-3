\chapter{}
\section{用正则表达式查找文本模式}
向 re.compile()传入一个字符串值,表示正则表达式,它将返回一个 Regex 模式
对象(或者就简称为 Regex 对象)。

Regex 对象的 search()方法查找传入的字符串,寻找该正则表达式的所有匹配。如果字符串中没有找到该正则表达式模式,search()方法将返回 None。如果找到了该模式,search()方法将返回一个 Match 对象。Match 对象有一个 group()方法,它返回被查找字符串中实际匹配的文本。


虽然在 Python 中使用正则表达式有几个步骤,但每一步都相当简单。
\begin{enumerate}
    \item 用 import re 导入正则表达式模块。
    \item 用 re.compile()函数创建一个 Regex 对象(记得使用原始字符串)。
    \item 向 Regex 对象的 search()方法传入想查找的字符串。它返回一个 Match 对象。
    \item 调用 Match 对象的 group()方法,返回实际匹配文本的字符串。
\end{enumerate}
\section{用正则表达式匹配更多模式}
\subsection{利用括号分组}
\begin{tcolorbox}
    我觉得可以不翻译为分组,而是理解为组件。
\end{tcolorbox}
添加括号将在正则表达式中创建“分组”:然后可以使用 group()匹配对象方法,从一个分组中获取匹配的文本。正则表达式字符串中的第一对括号是第 1 组。第二对括号是第 2 组。向 group()匹配对象方法传入整数 1 或 2,就可以取得匹配文本的不同部分。向 group()方法传入 0 或不传入参数,将返回整个匹配的文本。\textbf{只找第一个不会多找}

如果想要一次就获取所有的分组,请使用 groups()方法。

括号在正则表达式中有特殊的含义,但是如果你需要在文本中匹配括号,怎么办?在这种情况下,就需要用斜杠对 ( 与 ) 进行字符转义。

在正则表达式中,以下字符具有特殊含义:
\begin{verbatim}
    . ^ $ * + ? { } ( ) [ ] \ |
\end{verbatim}

如果要检测包含这些字符的文本模式,那么就需要用斜杠对它们进行转义。

要确保正则表达式中,没有将转义的括号 ( 与 ) 误作为 ( 与 )。如果你收到类似 missing 或 unbalanced parenthesis 的错误信息,则可能是忘记了为分组添加转义的右括号。
\subsection{用管道匹配多个分组}
字符 $|$ 称为“管道”。希望匹配许多表达式中的一个时,就可以使用它。例如,正则表达式 r`Batman$|$Tina Fey' 将匹配 `Batman' 或 `Tina Fey'。

如果 Batman 和 Tina Fey 都出现在被查找的字符串中,第一次出现的匹配文本,将作为 Match 对象返回。

\begin{tcolorbox}
    利用 findall()方法,可以找到“所有”匹配的地方。
\end{tcolorbox}
\subsection{用问号实现可选匹配}
有时候,想匹配的模式是可选的。就是说,不论这段文本在不在,正则表达式都会认为匹配。字符 ? 表明它前面的分组在这个模式中是可选的。你可以认为 ? 是在说,“匹配这个问号之前的分组零次或一次”。
\subsection{用星号匹配零次或多次}
*(称为星号)意味着“匹配零次或多次”,即星号之前的分组,可以在文本中出现任意次。它可以完全不存在,或一次又一次地重复。
\subsection{用加号匹配一次或多次}
* 意味着“匹配零次或多次”,+(加号)则意味着“匹配一次或多次”。星号不要求分组出现在匹配的字符串中,但加号不同,加号前面的分组必须“至少出现一次”。这不是可选的。
\subsection{用花括号匹配特定次数}
如果想要一个分组重复特定次数,就在正则表达式中该分组的后面,跟上花括号包围的数字。除了一个数字,还可以指定一个范围,即在花括号中写下一个最小值、一个逗号和一个最大值。也可以不写花括号中的第一个或第二个数字,不限定最小值或最大值。
\section{贪心和非贪心匹配}
Python 的正则表达式默认是“贪心”的,这表示在有二义的情况下,它们会尽可能匹配最长的字符串。花括号的“非贪心”版本匹配尽可能最短的字符串,即在结束的花括号后跟着一个问号。

\begin{tcolorbox}
    请注意,问号在正则表达式中可能有两种含义:声明非贪心匹配或表示可选的分组。这两种含义是完全无关的。
\end{tcolorbox}