\chapter{网络编程\label{ch14}}
\section{几个网络模块}
\subsection{模块 socket}
网络编程中的一个基本组件是\textbf{套接字}(socket)。套接字基本上是一个信息通道,两端各有一个程序。这些程序可能位于(通过网络相连的)不同的计算机上,通过套接字向对方发送信息。在 Python 中,大多数网络编程都隐藏了模块 socket 的基本工作原理,不与套接字直接交互。

套接字分为两类:服务器套接字和客户端套接字。创建服务器套接字后,让它等待连接请求的到来。这样,它将在某个网络地址(由IP地址和端口号组成)处监听,直到客户端套接字建立连接。随后,客户端和服务器就能通信了。

套接字是模块 socket 中 socket 类的实例。实例化套接字时最多可指定三个参数:一个地址族(默认为 \verb|socket.AF_INET|);是流套接字(\verb|socket.SOCK_STREAM|,默认设置)还是数据报套接字(\verb|socket.SOCK_DGRAM|);协议(使用默认值 0 就好)。创建普通套接字时,不用提供任何参数。

服务器套接字先调用方法 bind,再调用方法 listen 来监听特定的地址。然后,客户端套接字就可连接到服务器了,办法是调用方法 connect 并提供调用方法 bind 时指定的地址(在服务器端,可使用函数s ocket.gethostname 获取当前机器的主机名)。这里的地址是一个格式为 (host, port)的元组,其中 host 是主机名(如www.example.com),而 port 是端口号(一个整数)。方法 listen 接受一个参数——待办任务清单的长度(即最多可有多少个连接在队列中等待接纳,到达这个数量后将开始拒绝连接)。

服务器套接字开始监听后,就可接受客户端连接了,这是使用方法 accept 来完成的。这个方法将阻断(等待)到客户端连接到来为止,然后返回一个格式为 (client, address) 的元组,其中client是一个客户端套接字,而 address 是前面解释过的地址。服务器能以其认为合适的方式处理客户端连接,然后再次调用 accept 以接着等待新连接到来。这通常是在一个无限循环中完成的。
\subsection{模块 urllib 和 urllib2}
在可供使用的网络库中,urllib 和 urllib2 可能是投入产出比最高的两个。它们让你能够通过网络访问文件,就像这些文件位于你的计算机中一样。只需一个简单的函数调用,就几乎可将统一资源定位符(URL)可指向的任何动作作为程序的输入。
\subsubsection{打开远程文件}
几乎可以像打开本地文件一样打开远程文件,差别是只能使用读取模式,以及使用模块 urllib.request 中的函数 urlopen,而不是 open(或 file)。
\subsubsection{获取远程文件}
函数 urlopen 返回一个类似于文件的对象,可从中读取数据。如果要让 urllib 替你下载文件,并将其副本存储在一个本地文件中,可使用 urlretrieve。这个函数不返回一个类似于文件的对象,而返回一个格式为 (filename, headers) 的元组,其中 filename 是本地文件的名称(由 urllib自动创建),而 headers 包含一些有关远程文件的信息(这里不会介绍 headers,如果你想更深入地了解它,请在有关 urllib 的标准库文档中查找 urlretrieve)。如果要给下载的副本指定文件名,可通过第二个参数来提供。

如果你没有指定文件名,下载的副本将放在某个临时位置,可使用函数 open 来打开。但使用完毕后,你可能想将其删除,以免占用磁盘空间。要清空这样的临时文件,可调用函数 urlcleanup 且不提供任何参数,它将负责替你完成清空工作。