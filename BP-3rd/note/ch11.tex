\chapter{文件\label{ch11}}
\section{打开文件}
\subsection{文件模式}
显式地指定读取模式的效果与根本不指定模式相同。写入模式让你能够写入文件,并在文件不存在时创建它。独占写入模式更进一步,在文件已存在时引发 FileExistsError 异常。在写入模式下打开文件时,既有内容将被删除(截断),并从文件开头处开始写入;如果要在既有文件末尾继续写入,可使用附加模式。

`+' 可与其他任何模式结合起来使用,表示既可读取也可写入。例如,要打开一个文本文件进行读写,可使用 `r+'。(你可能还想结合使用 seek,详情请参阅本章后面的旁注“随机存取”。)请注意,`r+' 和 `w+' 之间有个重要差别:后者截断文件,而前者不会这样做。默认模式为 `rt',这意味着将把文件视为经过编码的 Unicode 文本,因此将自动执行解码和编码,且默认使用 UTF-8 编码。要指定其他编码和 Unicode 错误处理策略,可使用关键字参数 encoding 和 errors。
\begin{table}
    \centering
    \caption{函数 open 的参数 mode 的最常见取值}
    \label{tbl11-1}
    \begin{tabular}{ll}
        \hline
        值   & 描述                  \\
        \hline
        `r' & 读取模式(默认值)           \\
        `w' & 写入模式                \\
        `x' & 独占写入模式              \\
        `a' & 附加模式                \\
        `b' & 二进制模式(与其他模式结合使用)    \\
        `t' & 文本模式(默认值,与其他模式结合使用) \\
        `+' & 读写模式(与其他模式结合使用)     \\
        \hline
    \end{tabular}
\end{table}
\section{文件的基本方法}
\begin{tcolorbox}[title=三个标准流]
    一个标准数据输入源是 sys.stdin。当程序从标准输入读取时,你可通过输入来提供文本,也可使用管道将标准输入关联到其他程序的标准输出。

    你提供给 print 的文本出现在 sys.stdout中,向 input 提供的提示信息也出现在这里。写入到 sys.stdout 的数据通常出现在屏幕上,但可使用管道将其重定向到另一个程序的标准输入。

    错误消息(如栈跟踪)被写入到 sys.stderr,但与写入到 sys.stdout 的内容一样,可对其进行重定向。
\end{tcolorbox}