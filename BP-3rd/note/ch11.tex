\chapter{文件\label{ch11}}
\section{打开文件}
\subsection{文件模式}
显式地指定读取模式的效果与根本不指定模式相同。写入模式让你能够写入文件,并在文件不存在时创建它。独占写入模式更进一步,在文件已存在时引发 FileExistsError 异常。在写入模式下打开文件时,既有内容将被删除(截断),并从文件开头处开始写入;如果要在既有文件末尾继续写入,可使用附加模式。

`+' 可与其他任何模式结合起来使用,表示既可读取也可写入。例如,要打开一个文本文件进行读写,可使用 `r+'。(你可能还想结合使用 seek,详情请参阅本章后面的旁注“随机存取”。)请注意,`r+' 和 `w+' 之间有个重要差别:后者截断文件,而前者不会这样做。默认模式为 `rt',这意味着将把文件视为经过编码的 Unicode 文本,因此将自动执行解码和编码,且默认使用 UTF-8 编码。要指定其他编码和 Unicode 错误处理策略,可使用关键字参数 encoding 和 errors。
\begin{table}
    \centering
    \caption{函数 open 的参数 mode 的最常见取值}
    \label{tbl11-1}
    \begin{tabular}{ll}
        \hline
        值   & 描述                  \\
        \hline
        `r' & 读取模式(默认值)           \\
        `w' & 写入模式                \\
        `x' & 独占写入模式              \\
        `a' & 附加模式                \\
        `b' & 二进制模式(与其他模式结合使用)    \\
        `t' & 文本模式(默认值,与其他模式结合使用) \\
        `+' & 读写模式(与其他模式结合使用)     \\
        \hline
    \end{tabular}
\end{table}
\section{文件的基本方法}
\begin{tcolorbox}[title=三个标准流]
    一个标准数据输入源是 sys.stdin。当程序从标准输入读取时,你可通过输入来提供文本,也可使用管道将标准输入关联到其他程序的标准输出。

    你提供给 print 的文本出现在 sys.stdout中,向 input 提供的提示信息也出现在这里。写入到 sys.stdout 的数据通常出现在屏幕上,但可使用管道将其重定向到另一个程序的标准输入。

    错误消息(如栈跟踪)被写入到 sys.stderr,但与写入到 sys.stdout 的内容一样,可对其进行重定向。
\end{tcolorbox}

\subsection{使用管道重定向输出}
\begin{tcolorbox}[title=随机存取]
    随机存取在本章中,我将文件都视为流,只能按顺序从头到尾读取。实际上,可在文件中移动,只访问感兴趣的部分(称为随机存取)。为此,可使用文件对象的两个方法:seek 和 tell。方法 \verb|seek(offset[, whence])| 将当前位置(执行读取或写入的位置)移到 offset 和whence 指定的地方。参数 offset 指定了字节(字符)数,而参数 whence 默认为 \verb|io.SEEK_SET(0)|,这意味着偏移量是相对于文件开头的(偏移量不能为负数)。参数 whence 还可设置为 \verb|io.SEEK_CUR(1)| 或 \verb|io.SEEK_END(2)|,其中前者表示相对于当前位置进行移动(偏移量可以为负),而后者表示相对于文件末尾进行移动。\href{random_access}{示例}

    方法 tell() 返回当前位于文件的什么位置。
\end{tcolorbox}
\subsection{读取和写入行}
要读取一行(从当前位置到下一个分行符的文本),可使用方法 readline。调用这个方法时,可不提供任何参数(在这种情况下,将读取一行并返回它);也可提供一个非负整数,指定 readline 最多可读取多少个字符。要读取文件中的所有行,并以列表的方式返回它们,可使用方法 readlines。

方法 writelines 与 readlines 相反:接受一个字符串列表(实际上,可以是任何序列或可迭代对象),并将这些字符串都写入到文件(或流)中。请注意,写入时不会添加换行符,因此你必须自行添加。另外,没有方法 writeline,因为可以使用 write。

\subsection{关闭文件}
别忘了调用方法close将文件关闭。通常,程序退出时将自动关闭文件对象(也可能在退出程序前这样做),因此是否将读取的文件关闭并不那么重要。然而,关闭文件没有坏处,在有些操作系统和设置中,还可避免无意义地锁定文件以防修改。另外,这样做还可避免用完系统可能指定的文件打开配额。

对于写入过的文件,一定要将其关闭,因为 Python 可能\textbf{缓冲}你写入的数据(将数据暂时存储在某个地方,以提高效率)。因此如果程序因某种原因崩溃,数据可能根本不会写入到文件中。安全的做法是,使用完文件后就将其关闭。如果要重置缓冲,让所做的修改反映到磁盘文件中,但又不想关闭文件,可使用方法 flush。然而,需要注意的是,根据使用的操作系统和设置,flush 可能出于锁定考虑而禁止其他正在运行的程序访问这个文件。只要能够方便地关闭文件,就应将其关闭。

要确保文件得以关闭,可使用一条 try/finally 语句,并在 finally 子句中调用 close。实际上,有一条专门为此设计的语句,那就是 with 语句。with 语句让你能够打开文件并将其赋给一个变量(这里是 somefile)。在语句体中,你将数据写入文件(还可能做其他事情)。到达该语句末尾时,将自动关闭文件,即便出现异常亦如此。

\begin{tcolorbox}[title=上下文管理器]
    上下文管理器 with 语句实际上是一个非常通用的结构,允许你使用所谓的上下文管理器。上下文管理器是支持两个方法的对象:\verb|__enter__|和\verb|__exit__|。

    方法\verb|__enter__|不接受任何参数,在进入with语句时被调用,其返回值被赋给关键字as后面的变量。

    方法\verb|__exit__|接受三个参数:异常类型、异常对象和异常跟踪。它在离开方法时被调用(通过前述参数将引发的异常提供给它)。如果\verb|__exit__|返回 False,将抑制所有的异常。文件也可用作上下文管理器。

    它们的方法\verb|__enter__|返回文件对象本身,而方法\verb|__exit__|关闭文件。
\end{tcolorbox}

\section{迭代文件内容}
\subsection{每次一个字符(或字节)}
一种最简单(也可能是最不常见)的文件内容迭代方式是,在 while 循环中使用方法 read。例如,你可能想遍历文件中的每个字符(在二进制模式下是每个字节)。如果你每次读取多个字符(字节),可指定要读取的字符(字节)数。
\subsection{每次一行}
处理文本文件时,你通常想做的是迭代其中的行,而不是每个字符。readline,可像迭代字符一样轻松地迭代行。
\subsection{读取所有内容}
如果文件不太大,可一次读取整个文件;为此,可使用方法 read 并不提供任何参数(将整个文件读取到一个字符串中),也可使用方法 readlines(将文件读取到一个字符串列表中,其中每个字符串都是一行)。通过这样的方式读取文件,可轻松地迭代字符和行。请注意,除进行迭代外,像这样将文件内容读取到字符串或列表中也对完成其他任务很有帮助。
\subsection{使用 fileinput 实现延迟行迭代}
有时候需要迭代大型文件中的行,此时使用 readlines 将占用太多内存。当然,你可转而结合使用 while 循环和 readline,但在 Python 中,在可能的情况下,应首选 for 循环,而这里就属于这种情况。你可使用一种名为延迟行迭代的方法——说它延迟是因为它只读取实际需要的文本部分。请注意,模块 fileinput 会负责打开文件,你只需给它提供一个文件名即可。
\subsection{文件迭代器}
该来看看最酷(也是最常见)的方法了。文件实际上是可迭代的,这意味着可在 for 循环中直接使用它们来迭代行。请注意,与其他文件一样,\verb|sys.stdin| 也是可迭代的,因此要迭代标准输入中的所有行。
\section{小结}
\begin{description}
    \item[类似于文件的对象]:类似于文件的对象是支持 read 和 readline(可能还有 write 和 writelines)等方法的对象。
    \item[打开和关闭文件]:要打开文件,可使用函数open,并向它提供一个文件名。如果要确保即便发生错误时文件也将被关闭,可使用with语句。
    \item[模式和文件类型]:打开文件时,还可指定模式,如`r'(读取模式)或`w'(写入模式)。通过在模式后面加上`b',可将文件作为二进制文件打开,并关闭 Unicode 编码和换行符替换。
    \item[标准流]:三个标准流(模块 sys 中的 stdin、stdout和stderr)都是类似于文件的对象,它们实现了 UNIX 标准 I/O 机制(Windows 也提供了这种机制)。
    \item[读取和写入]:要从文件或类似于文件的对象中读取,可使用方法 read;要执行写入操作,可使用方法 write。
    \item[读取和写入行]:要从文件中读取行,可使用 readline 和 readlines;要写入行,可使用 writelines。
    \item[迭代文件内容]:迭代文件内容的方法很多,其中最常见的是迭代文本文件中的行,这可通过简单地对文件本身进行迭代来做到。还有其他与较旧 Python 版本兼容的方法,如使用 readlines。
\end{description}