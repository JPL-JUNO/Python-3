\chapter{魔法函数、特性和迭代器\label{chapter09}}
在Python中,有些名称很特别,开头和结尾都是两个下划线。你在本书前面已经见过一些,如\verb|__future__|。这样的拼写表示名称有特殊意义,因此绝不要在程序中创建这样的名称。在这样的名称中,很大一部分都是魔法(特殊)方法的名称。如果你的对象实现了这些方法,它们将在特定情况下(具体是哪种情况取决于方法的名称)被Python调用,而几乎不需要直接调用。

\section{如果你使用的不是 Python 3}
在Python 2.2中,Python对象的工作方式有了很大的变化。有一点需要注意:即便你使用的是较新的Python 2版本,有些功能(如特性和函数super)也不适用于旧式类。要让你的类是新式的,要么在模块开头包含赋值语句\verb|__metaclass__ = type|,要么直接或间接地继承内置类object或其他新式类。

\begin{pyc}
class NewStyle(object):
    more_code_here
class OldStyle:
    more_code_here
\end{pyc}

\begin{tcolorbox}
也可在类的作用域内给变量\verb|__metaclass__|赋值,但这样做只设置当前类的元类(metaclass)。元类是其他类所属的类,这是一个非常复杂的主题。
\end{tcolorbox}

请注意,在Python 3中没有旧式类,因此无需显式地继承object或将\verb|__metaclass__|设置为type。所有的类都将隐式地继承object。如果没有指定超类,将直接继承它,否则将间接地继承它。

\section{构造函数}
构造函数不同于普通方法的地方在于,将在对象创建后自动调用它们。
\begin{pyc}
class FooBar():
    def __init__(self):
        self.somevar = 42
f = FooBar()
f.somevar  # 42
\end{pyc}
如果给构造函数添加几个参数,结果将如何呢?
\begin{pyc}
class FooBar():
    def __init__(self, value=42):
        self.somevar = value
f = FooBar("This is constructor argument")
f.somevar  # 'This is constructor argument'
\end{pyc}

\begin{tcolorbox}
Python提供了魔法方法\verb|__del__|,也称作析构函数(destructor)。这个方法在对象被销毁(作为垃圾被收集)前被调用,但鉴于你无法知道准确的调用时间,建议尽可能不要使用\verb|__del__|。
\end{tcolorbox}
\subsection{重写普通方法和特殊的构造函数}
每个类都有一个或多个超类,并从它们那里继承行为。对类B的实例调用方法(或访问其属性)时,如果找不到该方法(或属性),将在其超类A中查找。

\begin{pyc}
class A:
    def hello(self):
        print("Hello, I'm A.")
class B(A):
    pass
a = A()
b = B()
a.hello()  # Hello, I'm A.
b.hello()  # Hello, I'm A.
\end{pyc}
由于类B自己没有定义方法hello,因此对其调用方法hello时,打印的是消息"Hello, I'm A."。

要在子类中添加功能,一种基本方式是添加方法。然而,你可能想重写超类的某些方法,以定制继承而来的行为。
\begin{pyc}
class B(A):
    def hello(self):
        print("Hello, I'm B")
b.hello()  # Hello, I'm B.
\end{pyc}

重写是继承机制的一个重要方面,对构造函数来说尤其重要。构造函数用于初始化新建对象的状态,而对大多数子类来说,除超类的初始化代码外,还需要有自己的初始化代码。虽然所有方法的重写机制都相同,但与重写普通方法相比,重写构造函数时更有可能遇到一个特别的问题:重写构造函数时,必须调用超类(继承的类)的构造函数,否则可能无法正确地初始化对象。
\begin{pyc}
class Bird:
    def __init__(self):
        self.hungry = True

    def eat(self):
        if self.hungry:
            print('Aaaah....')
            self.hungry = False
        else:
            print('No, thanks!')
b = Bird()
b.eat()  # Aaaah....
b.eat()  # No, thanks!
\end{pyc}
下面来看子类SongBird,它新增了鸣叫功能。
\begin{pyc}
class SongBird(Bird):
    def __init__(self):
        self.song = 'Squawk!'

    def sing(self):
        print(self.song)
sb = SongBird()
sb.sing()  # Squawk!
\end{pyc}
SongBird是Bird的子类,继承了方法eat,但如果你尝试调用它,将发现一个问题。

\verb|sb.eat()  # AttributeError: 'SongBird' object has no attribute 'hungry'|

异常清楚地指出了问题出在什么地方:SongBird没有属性hungry。为何会这样呢?因为在SongBird中重写了构造函数,但新的构造函数没有包含任何初始化属性hungry的代码。要消除这种错误,SongBird的构造函数必须调用其超类(Bird)的构造函数,以确保基本的初始化得以执行。为此,有两种方法:调用未关联的超类构造函数,以及使用函数super。

\subsection{调用未关联的超类构造函数}
本节介绍的方法主要用于解决历史遗留问题。在较新的Python版本中,显然应使用函数super。下面先给出前一节末尾问题的解决方案。

\begin{pyc}
class SongBird(Bird):
    def __init__(self):
        Bird.__init__(self)
        self.song = 'Squawk!'

    def sing(self):
        print(self.song)
\end{pyc}
在SongBird类中,只添加了一行,其中包含代码\verb|Bird.__init__(self)|。下面的代码表明这确实有用:
\begin{pyc}
sb.eat()  # Aaaah....
sb.eat()  # No, thanks!
\end{pyc}
这样做为何管用呢?对实例调用方法时,方法的参数self将自动关联到实例(称为关联的方法),这样的示例你见过多个。然而,如果你通过类调用方法(如\verb|Bird.__init__|),就没有实例与其相关联。在这种情况下,你可随便设置参数self。这样的方法称为未关联的。通过将这个未关联方法的self参数设置为当前实例,将使用超类的构造函数来初始化SongBird对象。
\subsection{使用函数 super}
如果你使用的不是旧版Python,就应使用函数super。\important{调用这个函数时,将当前类和当前实例作为参数。对其返回的对象调用方法时,调用的将是超类(而不是当前类)的方法}。在Python 3中调用函数super时,可不提供任何参数(通常也应该这样做)。
\begin{pyc}
class SongBird(Bird):
    def __init__(self):
        super().__init__()
        self.song = 'Squawk!'

    def sing(self):
        print(self.song)
\end{pyc}
\begin{tcolorbox}[title=使用函数super有何优点]
在我看来,相比于直接对超类调用未关联方法,使用函数super更直观,但这并非其唯一的优点。实际上,函数super很聪明,因此即便有多个超类,也只需调用函数super一次(条件是所有超类的构造函数也使用函数super)。另外,对于使用旧式类时处理起来很棘手的问题(如两个超类从同一个类派生而来),在使用新式类和函数super时将自动得到处理。你无需知道函数super的内部工作原理,但必须知道的是,使用函数super比调用超类的未关联构造函数(或其他方法)要好得多。

函数super返回的到底是什么呢?通常,你无需关心这个问题,只管假定它返回你所需的超类即可。实际上,它返回的是一个super对象,这个对象将负责为你执行方法解析。当你访问它的属性时,它将在所有的超类(以及超类的超类,等等)中查找,直到找到指定的属性或引发AttributeError异常。
\end{tcolorbox}
\section{元素访问}
\begin{tcolorbox}
在Python中,协议通常指的是规范行为的规则。协议指定应实现哪些方法以及这些方法应做什么。在Python中,多态仅仅基于对象的行为(而不基于祖先,如属于哪个类或其超类等),因此这个概念很重要:其他的语言可能要求对象属于特定的类或实现了特定的接口,而Python通常只要求对象遵循特定的协议。
\end{tcolorbox}

\subsection{基本的序列和映射协议}
序列和映射基本上是元素(item)的集合,要实现它们的基本行为(协议),不可变对象需要实现2个方法,而可变对象需要实现4个。

\begin{dinglist}{111}
	\item \verb|__len__(self)|:这个方法应返回集合包含的项数,对序列来说为元素个数,对映射来说为键值对数。如果\verb|__len__|返回零(且没有实现覆盖这种行为的\verb|__nonzero__|),对象在布尔上下文中将被视为假(就像空的列表、元组、字符串和字典一样)。
	\item \verb|__getitem__(self, key)|:这个方法应返回与指定键相关联的值。对序列来说,键应该是$0\sim n-1$的整数(也可以是负数,这将在后面说明),其中n为序列的长度。对映射来说,键可以是任何类型。
	\item \verb|__setitem__(self, key, value)|:这个方法应以与键相关联的方式存储值,以便以后能够使用\verb|__getitem__|来获取。当然,仅当对象可变时才需要实现这个方法。
	\item \verb|__delitem__(self, key)|:这个方法在对对象的组成部分使用\verb|__del__|语句时被调用,应删除与key相关联的值。同样,仅当对象可变(且允许其项被删除)时,才需要实现这个方法。
\end{dinglist}

对于这些方法,还有一些额外的要求。
\begin{dinglist}{111}
\item 对于序列,如果键为负整数,应从末尾往前数。换而言之,x[-n]应与x[len(x)-n]等效。
\item 如果键的类型不合适(如对序列使用字符串键),可能引发TypeError异常。
\item 对于序列,如果索引的类型是正确的,但不在允许的范围内,应引发IndexError异常。
\end{dinglist}

下面来试一试,看看能否创建一个无穷序列。
\begin{pyc}
def check_index(key):
    """
    Is the given key an acceptable index?
    To be acceptable, the key should be a non-negative integer. 
    If it is not an integer, a TypeError is raised; 
    if it is negative, an IndexError is raised (since the sequence is of infinite length).
    """
    if not isinstance(key, int):
        raise TypeError
    if key < 0:
        raise ValueError
        
class ArithmeticsSequence:
    def __init__(self, start=0, step=1):
        """        
        Initialize the arithmetic sequence.
        start - the first value in the sequence
        step - the difference between two adjacent values
        changed - a dictionary of values that have been modified by the user
        """
        self.start = start
        self.step = step
        self.changed = {}

    def __getitem__(self, key):
        """
        Get an item from the arithmetic sequence.
        """
        check_index(key)
        try:
            return self.changed[key]
        except KeyError:
            return self.start + key * self.step

    def __setitem__(self, key, value):
        """
        Change an item in the arithmetic sequence.
        """
        check_index(key)
        self.changed[key] = value
\end{pyc}

下面的示例演示了如何使用这个类:
\begin{pyc}
s = ArithmeticsSequence(1, 2)
s[4]  # 9
s[4] = 2
s[4]  # 2
\end{pyc}

请注意,禁止删除元素,因此没有实现\verb|__del__|,另外,这个类没有方法\verb|__len__|,因为其长度是无穷的。

\subsection{从 list、dict 和 str 派生}
基本的序列/映射协议指定的4个方法能够让你走很远,但序列还有很多其他有用的魔法方法和普通方法。要实现所有这些方法,不仅工作量大,而且难度不小。如果只想定制某种操作的行为,就没有理由去重新实现其他所有方法。这就是程序员的懒惰(也是常识)。

关键就是继承,在标准库中,模块collections提供了抽象和具体的基类,但你也可以继承内置类型。因此,如果要实现一种行为类似于内置列表的序列类型,可直接继承list。

\begin{pyc}
class CounterList(list):
    def __init__(self, *args):
        super().__init__(*args)
        self.counter = 0

    def __getitem__(self, index):
        self.counter += 1
        return super(CounterList, self).__getitem__(index)
        # return super().__getitem__(index)
\end{pyc}
\begin{tcolorbox}
重写\verb|__getitem__|并不能保证一定会捕捉用户的访问操作,因为还有其他访问列表内容的方式,如通过方法pop。
\end{tcolorbox}

\begin{pyc}
cl = CounterList(range(10))
cl  # [0, 1, 2, 3, 4, 5, 6, 7, 8, 9]
cl.reverse()
cl  # [9, 8, 7, 6, 5, 4, 3, 2, 1, 0]
cl.counter  # 0

cl[4] + cl[2] + cl[3]
cl.counter  # 3
\end{pyc}

如你所见,CounterList的行为在大多数方面都类似于列表,但它有一个counter属性(其初始值为0)。每当你访问列表元素时,这个属性的值都加1。执行加法运算\verb|cl[4] + cl[2] + cl[3]|后,counter的值递增三次,变成了3。

特殊(魔法)名称的用途很多,前面展示的只是冰山一角。魔法方法大多是为非常高级的用途准备的,因此这里不详细介绍。然而,如果你感兴趣,可以模拟数字,让对象像函数一样被调用,影响对象的比较方式,等等。要更详细地了解有哪些魔法方法,可参阅“Python Reference Manual”的Special method names一节。

\section{特性}
如果访问给定属性时必须采取特定的措施,那么像这样封装状态变量(属性)很重要。请看下面的Rectangle类:
\begin{pyc}
class Rectangle:
    def __init__(self):
        self.width = 0
        self.height = 0

    def set_size(self, size):
        self.width, self.height = size

    def get_size(self):
        return self.width, self.height
r = Rectangle()
r.width = 10
r.height = 5
r.get_size()  # (10, 5)
r.set_size((150, 100))
r.height  # 100
\end{pyc}
\verb|get_size|和\verb|set_size|是假想属性size的存取方法,这个属性是一个由width和height组成的元组。如果有一天你想修改实现,让size成为真正的属性,而width和height是动态计算出来的,就需要提供用于访问width和height的存取方法,使用这个类的程序也必须重写。应让客户端代码(使用你所编写代码的代码)能够以同样的方式对待所有的属性。

那么如何解决这个问题呢?给所有的属性都提供存取方法吗?这当然并非不可能,但如果有大量简单的属性,这样做就不现实(而且有点傻),因为将需要编写大量这样的存取方法,除了获取或设置属性外什么都不做。这将引入复制并粘贴(重复代码)的坏味,显然很糟糕(虽然在有些语言中,这样的问题很常见)。所幸Python能够替你隐藏存取方法,让所有的属性看起来都一样。通过存取方法定义的属性通常称为特性(property)。

\subsection{函数 property}
函数property使用起来很简单。如果你编写了一个类,如前一节的Rectangle类,只需再添加一行代码。
\begin{pyc}
class Rectangle:
    def __init__(self):
        self.width = 0
        self.height = 0

    def set_size(self, size):
        self.width, self.height = size

    def get_size(self):
        return self.width, self.height

    size = property(get_size, set_size)
\end{pyc}

实际上,调用函数property时,还可不指定参数、指定一个参数、指定三个参数或指定四个参数。如果没有指定任何参数,创建的特性将既不可读也不可写。如果只指定一个参数(获取方法),创建的特性将是只读的。第三个参数是可选的,指定用于删除属性的方法(这个方法不接受任何参数)。第四个参数也是可选的,指定一个文档字符串。这些参数分别名为fget、fset、fdel和doc。如果你要创建一个只可写且带文档字符串的特性,可使用它们作为关键字参数来实现。

\begin{tcolorbox}
你可能很好奇,想知道特性是如何完成其魔法的,下面就来说一说。如果你对此不感兴趣,可跳过这些内容。

property其实并不是函数,而是一个类。它的实例包含一些魔法方法,而所有的魔法都是由这些方法完成的。这些魔法方法为\verb|__get__|、\verb|__set__|和\verb|__delete__|,它们一道定义了所谓的描述符协议。只要对象实现了这些方法中的任何一个,它就是一个描述符。描述符的独特之处在于其访问方式。例如,读取属性(具体来说,是在实例中访问类中定义的属性)时,如果它关联的是一个实现了\verb|__get__|的对象,将不会返回这个对象,而是调用方法\verb|__get__|并将其结果返回。实际上,这是隐藏在特性、关联的方法、静态方法和类方法(详细信息请参阅下一小节)以及super后面的机制。

有关描述符的详细信息,请参阅\href{https://docs.python.org/3/howto/descriptor.html}{Descriptor HowTo Guide}。
\end{tcolorbox}
\subsection{静态方法和类方法}
静态方法和类方法是这样创建的:将它们分别包装在staticmethod和classmethod类的对象中。静态方法的定义中没有参数self,可直接通过类来调用。类方法的定义中包含类似于self的参数,通常被命名为cls。对于类方法,也可通过对象直接调用,但参数cls将自动关联到类。下面是一个简单的示例:
\begin{pyc}
class MyClass:
    def smeth():
        print('This is a static method')
    smeth = staticmethod(smeth)

    def cmeth(cls):
        print('This is a class method of', cls)
    cmeth = classmethod(cmeth)
\end{pyc}
像这样手工包装和替换方法有点繁琐。在Python 2.4中,引入了一种名为装饰器的新语法,可用于像这样包装方法。(实际上,装饰器可用于包装任何可调用的对象,并且可用于方法和函数。)可指定一个或多个装饰器,为此可在方法(或函数)前面使用运算符@列出这些装饰器(指定了多个装饰器时,应用的顺序与列出的顺序相反)。
\begin{pyc}
class MyClass:
    @staticmethod
    def smeth():
        print('This is a static method')

    @classmethod
    def cmeth(cls):
        print('This is a class method of', cls)
\end{pyc}

定义这些方法后,就可像下面这样使用它们(无需实例化类)。在Python中,静态方法和类方法以前一直都不太重要,主要是因为从某种程度上说,总是可以使用函数或关联的方法替代它们,而且早期的Python版本并不支持它们。因此,虽然较新的代码没有大量使用它们,但它们确实有用武之地(如工厂函数),因此你或许应该考虑使用它们。
\begin{pyc}
MyClass.smeth()  # This is a static method
MyClass.cmeth()  # This is a class method of <class '__main__.MyClass'>
\end{pyc}

\subsection{\texttt{\_\_getattr\_\_}、\texttt{\_\_setattr\_\_}等方法}
可以拦截对对象属性的所有访问企图,其用途之一是在旧式类中实现特性。要在属性被访问时执行一段代码,必须使用一些魔法方法。下面的四个魔法方法提供了你需要的所有功能(在旧式类中,只需使用后面三个)。
\begin{dinglist}{111}
\item \verb|__getattribute__(self, name)|:在属性被访问时自动调用(只适用于新式类)。
\item \verb|__getattr__(self, name)|:在属性被访问而对象没有这样的属性时自动调用。
\item \verb|__setattr__(self, name, value)|:试图给属性赋值时自动调用。
\item \verb|__delattr__(self, name)|:试图删除属性时自动调用。
\end{dinglist}
这些魔法方法使用起来要棘手些(从某种程度上说,效率也更低),但它们很有用,因为你可在这些方法中编写处理多个特性的代码。然而,在可能的情况下,还是使用函数property吧。

再来看前面的Rectangle示例:
\begin{pyc}
class Rectangle:
    def __init__(self, width, height):
        self.width = width
        self.height = height

    def __setattr__(self, name, value):
        if name == 'size':
            self.width, self.height = value
        else:
            self.__dict__[name] = value

    def __getattr__(self, name):
        if name == 'size':
            return self.width, self.height
        else:
            raise AttributeError()
\end{pyc}
这个版本需要处理额外的管理细节。对于这个代码示例,需要注意如下两点:
\begin{dinglist}{111}
\item 即便涉及的属性不是size,也将调用方法\verb|__setattr__|。因此这个方法必须考虑如下两种情形:如果涉及的属性为size,就执行与以前一样的操作;否则就使用魔法属性\verb|__dict__|。\verb|__dict__|属性是一个字典,其中包含所有的实例属性。之所以使用它而不是执行常规属性赋值,是因为旨在避免再次调用\verb|__setattr__|,进而导致无限循环。
\item 仅当没有找到指定的属性时,才会调用方法\verb|__getattr__|。这意味着如果指定的名称不是size,这个方法将引发AttributeError异常。这在要让类能够正确地支持hasattr和getattr等内置函数时很重要。如果指定的名称为size,就使用前一个实现中的表达式。
\end{dinglist}
\begin{tcolorbox}
前面说过,编写方法\verb|__setattr__|时需要避开无限循环陷阱,编写\verb|__getattribute__|时亦如此。由于它拦截对所有属性的访问(在新式类中),因此将拦截对\verb|__dict__|的访问!在\verb|__getattribute__|中访问当前实例的属性时,唯一安全的方式是使用超类的方法\verb|__getattribute__|(使用super)。
\end{tcolorbox}
\section{迭代器}
\subsection{迭代器协议}
迭代(iterate)意味着重复多次,就像循环那样。本书前面只使用for循环迭代过序列和字典,但实际上也可迭代其他对象:实现了方法\verb|__iter__|的对象。

方法\verb|__iter__|返回一个迭代器,它是包含方法\verb|__next__|的对象,而调用这个方法时可不提供任何参数。当你调用方法\verb|__next__|时,迭代器应返回其下一个值。如果迭代器没有可供返回的值,应引发StopIteration异常。(个人理解:\notes{next魔法函数就是将其返回值作为迭代中用的那个值,从迭代开始一直到结束每轮迭代都调用。不应该是什么返回下一值})你还可使用内置的便利函数next,在这种情况下,\verb|next(it)|与\verb|it.__next__()|等效。

这有什么意义呢?为何不使用列表呢?因为在很多情况下,使用列表都有点像用大炮打蚊子。例如,\important{如果你有一个可逐个计算值的函数,你可能只想逐个地获取值,而不是使用列表一次性获取。这是因为如果有很多值,列表可能占用太多的内存}。但还有其他原因:\tips{使用迭代器更通用、更简单、更优雅。}下面来看一个不能使用列表的示例,因为如果使用,这个列表的长度必须是无穷大的!

\begin{pyc}
class Fibs():
    def __init__(self):
        self.a = 0
        self.b = 1

    def __next__(self):
        self.a, self.b = self.b, self.a + self.b
        return self.a

    def __iter__(self):
        return self
\end{pyc}

注意到这个迭代器实现了方法\verb|__iter__|,而这个方法返回迭代器本身。在很多情况下,都在另一个对象中实现返回迭代器的方法\verb|__iter__|,并在for循环中使用这个对象。但推荐在迭代器中也实现方法\verb|__iter__|(并像刚才那样让它返回self),这样迭代器就可直接用于for循环中。

\begin{tcolorbox}[colback=red!20]
更正规的定义是,实现了方法\verb|__iter__|的对象是可迭代的,而实现了方法\verb|__next__|的对象是迭代器。
\end{tcolorbox}
首先,创建一个Fibs对象,然后就可在for循环中使用这个对象,如找出第一个大于1000的斐波那契数。
\begin{pyc}
fibs = Fibs()
for i in fibs:
    if i > 1000:
        print(i)  # 1597
        break
\end{pyc}

\begin{tcolorbox}
通过对可迭代对象调用内置函数iter,可获得一个迭代器。
\begin{pyc}
it = iter([1, 2, 3])
next(it)  # 1
next(it)  # 2
\end{pyc}
\end{tcolorbox}
\subsection{从迭代器创建序列}
除了对迭代器和可迭代对象进行迭代(通常这样做)之外,还可将它们转换为序列。在可以使用序列的情况下,大多也可使用迭代器或可迭代对象(诸如索引和切片等操作除外)。
\begin{pyc}
class TestIterator:
    value = 0

    def __next__(self):
        self.value += 1
        if self.value > 10:
            raise StopIteration
        return self.value

    def __iter__(self):
        return self
ti = TestIterator()
list(ti)  # [1, 2, 3, 4, 5, 6, 7, 8, 9, 10]
\end{pyc}
\section{生成器}
生成器是一个相对较新的Python概念。由于历史原因,它也被称为简单生成器(simple generator)。生成器和迭代器可能是近年来引入的最强大的功能,但生成器是一个相当复杂的概念,你可能需要花些功夫才能明白其工作原理和用途。

生成器是一种使用普通函数语法定义的迭代器。
\subsection{创建生成器}
创建一个将嵌套列表展开的函数,这个函数将一个类似于下面的列表作为参数:
\begin{pyc}
def flatten(nested):
    for sublist in nested:
        for element in sublist:
            yield element
\end{pyc}
包含yield语句的函数都被称为生成器。这可不仅仅是名称上的差别,生成器的行为与普通函数截然不同。\important{差别在于,生成器不是使用return返回一个值,而是可以生成多个值,每次一个。每次使用yield生成一个值后,函数都将冻结,即在此停止执行,等待被重新唤醒。被重新唤醒后,函数将从停止的地方开始继续执行}。

为使用所有的值,可对生成器进行迭代。
\begin{pyc}
nested = [[2, 2,], [4, 0], [5]]
for num in flatten(nested):
    print(num)
# or
list(flatten(nested))  # [2, 2, 4, 0, 5]
\end{pyc}

\begin{tcolorbox}[title=简单生成器,colback=red!5]
在Python 2.4中,引入了一个类似于列表推导的概念:生成器推导(也叫生成器表达式)。其工作原理与列表推导相同,但不是创建一个列表(即不立即执行循环),而是返回一个生成器,让你能够逐步执行计算。
\begin{pyc}
g = ((i + 2) ** 2 for i in range(2, 27))
next(g)  # 16
next(g)  # 25
\end{pyc}

如你所见,不同于列表推导,这里使用的是圆括号。在像这样的简单情形下,还不如使用列表推导;但如果要包装可迭代对象(可能生成大量的值),使用列表推导将立即实例化一个列表,从而丧失迭代的优势。

另一个好处是,直接在一对既有的圆括号内(如在函数调用中)使用生成器推导时,无需再添加一对圆括号。换而言之,可编写下面这样非常漂亮的代码:

\verb|sum(i ** 2 for i in range(10))|
\end{tcolorbox}
\subsection{递归式生成器}
前一节设计的生成器只能处理两层的嵌套列表,这是使用两个for循环来实现的。如果要处理任意层嵌套的列表,该如何办呢?对于每层嵌套,都需要一个for循环,但由于不知道有多少层嵌套,你必须修改解决方案,使其更灵活。该求助于递归了。

\begin{pyc}
def flatten(nested):
    try:
        for sublist in nested:
            for element in flatten(sublist):
                yield element
    except TypeError:
        yield nested
\end{pyc}

调用flatten时,有两种可能性(处理递归时都如此):基线条件和递归条件。在基线条件下,要求这个函数展开单个元素(如一个数)。在这种情况下,for循环将引发TypeError异常(因为你试图迭代一个数),而这个生成器只生成一个元素。然而,如果要展开的是一个列表(或其他任何可迭代对象),你就需要做些工作:遍历所有的子列表(其中有些可能并不是列表)并对它们调用flatten,然后使用另一个for循环生成展开后的子列表中的所有元素。
\begin{pyc}
list(flatten([[[1, 2, 2], 1, [1, 2, 1], 1], 1, [1, 1, 1]]))
# [1, 2, 2, 1, 1, 2, 1, 1, 1, 1, 1, 1]
\end{pyc}
然而,这个解决方案存在一个问题。如果nested是字符串或类似于字符串的对象,它就属于序列,因此不会引发TypeError异常,可你并不想对其进行迭代。
\begin{tcolorbox}
在函数flatten中,不应该对类似于字符串的对象进行迭代,主要原因有两个。首先,你想将类似于字符串的对象视为原子值,而不是应该展开的序列。其次,对这样的对象进行迭代会导致无穷递归,因为字符串的第一个元素是一个长度为1的字符串,而长度为1的字符串的第一个元素是字符串本身!
\end{tcolorbox}
要处理这种问题,必须在生成器开头进行检查。要检查对象是否类似于字符串,最简单、最快捷的方式是,尝试将对象与一个字符串拼接起来,并检查这是否会引发TypeError异常。

\begin{pyc}
def flatten(nested):
    try:
        try:
            nested + ''
        except TypeError:
            pass
        else:
            raise TypeError
        for sublist in nested:
            for element in flatten(sublist):
                yield element
    except TypeError:
        yield nested
\end{pyc}
如果表达式\verb|nested + ''|引发了TypeError异常,就忽略这种异常;如果没有引发TypeError异常,内部try语句中的else子句将引发TypeError异常,这样将在外部的excpet子句中原封不动地生成类似于字符串的对象。
\begin{pyc}
list(flatten(['foo', ['bar', ['baz']]]))
# ['foo', 'bar', 'baz']
\end{pyc}
这里没有执行类型检查:我没有检查nested是否是字符串,而只是检查其行为是否类似于字符串,即能否与字符串拼接。对于这种检查,一种更自然的替代方案是,使用isinstance以及字符串和类似于字符串的对象的一些抽象超类,但遗憾的是没有这样的标准类。另外,即便是对UserString来说,也无法检查其类型是否为str。

\subsection{通用生成器}
生成器是包含关键字yield的函数,但被调用时不会执行函数体内的代码,而是返回一个迭代器。每次请求值时,都将执行生成器的代码,直到遇到yield或return。yield意味着应生成一个值,而return意味着生成器应停止执行(即不再生成值;仅当在生成器调用return时,才能不提供任何参数)。

换而言之,生成器由两个单独的部分组成:生成器的函数和生成器的迭代器。生成器的函数是由def语句定义的,其中包含yield。生成器的迭代器是这个函数返回的结果。用不太准确的话说,这两个实体通常被视为一个,通称为生成器。
\begin{pyc}
def simple_generator():
    yield 1
simple_generator
# <function __main__.simple_generator()>
simple_generator()
# <generator object simple_generator at 0x000001BEF4C3D380>
\end{pyc}
对于生成器的函数返回的迭代器,可以像使用其他迭代器一样使用它。

\subsection{生成器的方法}
在生成器开始运行后,可使用生成器和外部之间的通信渠道向它提供值。这个通信渠道包含如下两个端点。
\begin{enumerate}
\item 外部世界:外部世界可访问生成器的方法send,这个方法类似于next,但接受一个参数(要发送的“消息”,可以是任何对象)。
\item 生成器:在挂起的生成器内部,yield可能用作表达式而不是语句。换而言之,当生成器重新运行时,yield返回一个值——通过send从外部世界发送的值。如果使用的是next,yield将返回None。
\end{enumerate}
请注意,仅当生成器被挂起(即遇到第一个yield)后,使用send(而不是next)才有意义。要在此之前向生成器提供信息,可使用生成器的函数的参数。
\begin{pyc}
def repeater(value):
    while True:
        new = (yield value)
        if new is not None:
            value = new
r = repeater(42)
next(r)  # 42
r.send('Hello, world')  # Hello, world
\end{pyc}
如果要以某种方式使用返回值,就不管三七二十一,将其用圆括号括起吧。生成器还包含另外两个方法:
\begin{enumerate}
\item 方法throw:用于在生成器中(yield表达式处)引发异常,调用时可提供一个异常类型、一个可选值和一个traceback对象。
\item 方法close:用于停止生成器,调用时无需提供任何参数。
\end{enumerate}
方法close(由Python垃圾收集器在需要时调用)也是基于异常的:在yield处引发GeneratorExit异常。因此如果要在生成器中提供一些清理代码,可将yield放在一条try/finally语句中。
\subsection{模拟生成器}
如果你使用的是较老的Python版本,就无法使用生成器。下面是一个简单的解决方案,让你能够使用普通函数模拟生成器。首先,在函数体开头插入如下代码\verb|results = []|。接下来,将类似于

\verb|yield some_expression|

的代码行替换为如下代码行:
\begin{pyc}
yield some_expression with this:
result.append(some_expression)
\end{pyc}
最后,在函数末尾添加如下代码\verb|return result|。

尽管使用这种方法并不能模拟所有的生成器,但可模拟大部分生成器。例如,这无法模拟无穷生成器,因为显然不能将这种生成器的值都存储到一个列表中。

下面使用普通函数重写了生成器flatten:
\begin{pyc}
def flatten(nested):
    result = []
    try:
        try:
            nested + ''
        except TypeError:
            pass
        else:
            raise TypeError
        for sublist in nested:
            for element in flatten(sublist):
                result.append(element)
    except TypeError:
        result.append(nested)
    return result
\end{pyc}
\section{八皇后问题}
\subsection{生成器的回溯}
对于逐步得到结果的复杂递归算法,非常适合使用生成器来实现。要在不使用生成器的情况下实现这些算法,通常必须通过额外的参数来传递部分结果,让递归调用能够接着往下计算。通过使用生成器,所有的递归调用都只需生成其负责部分的结果。前面的递归版flatten就是这样做的,你可使用这种策略来遍历图结构和树结构。

然而,在有些应用程序中,你不能马上得到答案。你必须尝试多次,且在每个递归层级中都如此。打个现实生活中的比方吧,假设你要去参加一个很重要的会议。你不知道会议在哪里召开,但前面有两扇门,而会议室就在其中一扇门的后面。你选择进入左边那扇门后,又看到两扇门。你再次选择进入左边那扇门,但发现走错了。因此你往回走,并进入右边那扇门,但发现也走错了。因此你继续往回走到起点,现在可以尝试进入右边那扇门。

\begin{tcolorbox}[title=图和树]
下面的网页提供了有关图和树的简明定义:
\begin{itemize}
\item \url{http://mathworld.wolfram.com/Graph.html}
\item \url{https://mathworld.wolfram.com/Tree.html}
\item \url{https://mathworld.wolfram.com/Tree.html}
\item \url{https://xlinux.nist.gov/dads/HTML/graph.html}
\end{itemize}
\end{tcolorbox}






























