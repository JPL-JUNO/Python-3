\chapter{字典:当索引行不通时\label{chapter04}}
需要将一系列值组合成数据结构并通过编号来访问各个值时,列表很有用。本章介绍一种可通过名称来访问其各个值的数据结构。这种数据结构称为映射(mapping)。字典是Python中唯一的内置映射类型,其中的值不按顺序排列,而是存储在键下。键可能是数、字符串或元组。

\section{字典的用途}
字典的名称指出了这种数据结构的用途。普通图书适合按从头到尾的顺序阅读,如果你愿意,可快速翻到任何一页,这有点像Python中的列表。字典(日常生活中的字典和Python字典)旨在让你能够轻松地找到特定的单词(键),以获悉其定义(值)。

\section{创建和使用字典}
字典以类似于下面的方式表示:

\verb|phonebook = {"Alice": '2341', "Neth": '9012', "Cecil": '3258'}|

字典由键及其相应的值组成,这种键值对称为项(item)。在前面的示例中,键为名字,而值为电话号码。每个键与其值之间都用冒号(:)分隔,项之间用逗号分隔,而整个字典放在花括号内。空字典(没有任何项)用两个花括号表示,类似于下面这样:\verb|{}|。

\warning{
    在字典(以及其他映射类型)中,键必须是独一无二的,而字典中的值无需如此。
}

\subsection{函数dict}
可使用函数dict从其他映射(如其他字典)或键-值对序列创建字典。

\begin{pyc}
items = [("name", "Gumby"), ("age", 42)]
d = dict(items)
d  # {'name': 'Gumby', 'age': 42}
d['name']  # 'Gumby'
\end{pyc}
还可使用关键字实参来调用这个函数:
\begin{pyc}
d = dict(name="Gumby", age=42)
d  # {'name': 'Gumby', 'age': 42}
\end{pyc}

尽管这可能是函数dict最常见的用法,但也可使用一个映射实参来调用它,这将创建一个字典,其中包含指定映射中的所有项。从映射创建字典时,如果该映射也是字典(毕竟字典是Python中唯一的内置映射类型),可不使用函数dict,而是使用字典方法copy。
\subsection{基本的字典操作}
字典的基本行为在很多方面都类似于序列。
\begin{dinglist}{43}
    \item len(d)返回字典d包含的项(键值对)数。
    \item d[k]返回与键k相关联的值。
    \item d[k] = v将值v关联到键k。
    \item del d[k]删除键为k的项。
    \item k in d检查字典d是否包含键为k的项。
\end{dinglist}

虽然字典和列表有多个相同之处,但也有一些重要的不同之处。
\begin{dinglist}{43}
    \item 键的类型:字典中的键可以是整数,但并非必须是整数。字典中的键可以是任何不可变的类型,如浮点数(实数)、字符串或元组。
    \item 自动添加:即便是字典中原本没有的键,也可以给它赋值,这将在字典中创建一个新项。然而,如果不使用append或其他类似的方法,就不能给列表中没有的元素赋值。
    \item 成员资格:表达式\verb|k in d|(其中d是一个字典)查找的是键而不是值,而表达式\verb|v in l|(其中l是一个列表)查找的是值而不是索引。这看似不太一致,但你习惯后就会觉得相当自然。毕竟如果字典包含指定的键,检查相应的值就很容易。
\end{dinglist}

\notes{相比于检查列表是否包含指定的值,检查字典是否包含指定的键的效率更高。数据结构越大,效率差距就越大。}

\begin{pyc}
x = []
x[42] = "Foobar"  # IndexError: list assignment index out of range

x = {}
x[42] = "Foobar"
x  # {42: 'Foobar'}
\end{pyc}

\begin{py}{字典示例}
people = {
    "Alice": {"Phone": '2341', "addr": 'Foo drive 23'},
    "Beth": {"Phone": '9102', 'addr': 'Bar street 42'},
    'Cecil': {'Phone': '3158', 'addr': 'Baz avenue 90'}
}

labels = {"Phone": "Phone number", 'addr': 'address'}
name = input('Name: ')
request = input("Phone number (p) or address (a)?")
if request == 'p':
    key = 'Phone'
if request == 'a':
    key = 'addr'

if name in people:
    print("{}'s {} is {}".format(name, labels[key], people[name][key]))
# Name: Beth
# Phone number (p) or address (a)? p
# Beth's phone number is 9102.
\end{py}

\subsection{将字符串格式设置功能用于字典}
\autoref{chapter03}介绍过,可使用字符串格式设置功能来设置值的格式,这些值是作为命名或非命名参数提供给方法format的。在有些情况下,通过在字典中存储一系列命名的值,可让格式设置更容易些。为此,必须使用\verb|format_map|来指出你将通过一个映射来提供所需的信息。

\begin{pyc}
phonebook = {"Alice": '2341', "Neth": '9012', "Cecil": '3258'}
"Cecil's phone number is {Cecil}.".format_map(phonebook)
# "Cecil's phone number is 3258."
\end{pyc}
像这样使用字典时,可指定任意数量的转换说明符,条件是所有的字段名都是包含在字典中的键。在模板系统中,这种字符串格式设置方式很有用(下面的示例使用的是HTML)。
\begin{pyc}
data = {'title': 'My Home Page', 'text': 'Welcome to my home page!'}
print(template.format_map(data))
# <html>
# <head><title>My Home Page</title></head>
# <body>
# <h1>My Home Page</h1>
# <p>Welcome to my home page!</p>
# </body>
\end{pyc}
\subsection{字典方法}
字典的方法很有用,但其使用频率可能没有列表和字符串的方法那样高。
\subsubsection{clear}
方法clear删除所有的字典项,这种操作是就地执行的,因此什么都不返回(或者说返回None)。
\begin{pyc}
d = {}
d['name'] = 'Gumby'
d['age'] = 42
d  # {'name': 'Gumby', 'age': 42}
return_value = d.clear()
print(return_value)  # None
\end{pyc}

这为何很有用呢?我们来看两个场景:
\begin{itemize}
    \item 场景一
    \begin{pyc}
x = {}
y = x
x['key'] = 'value'
y  # {'key': 'value'}
x = {}
y  # {'key': 'value'}
    \end{pyc}
    \item 场景二
    \begin{pyc}
x = {}
y = x
x['key'] = 'value'
y  # {'key': 'value'}
x.clear()
y  # {}
    \end{pyc}
\end{itemize}
在这两个场景中,x和y最初都指向同一个字典。在第一个场景中,我通过将一个空字典赋给x来“清空”它。这对y没有任何影响,它依然指向原来的字典。这种行为可能正是你想要的,但要删除原来字典的所有元素,必须使用clear。如果这样做,y也将是空的,如第二个场景所示。

\subsubsection{copy}
方法 copy 返回一个新字典,其包含的键-值对与原来的字典相同(\important{这个方法执行的是浅复制,因为值本身是原件,而非副本})。

\begin{pyc}
x = {'username': 'admin', 'machines': ['foo', 'bar', 'baz']}
y = x.copy()
y['username'] = 'mhl'
y['machines'].remove('bar')
y  # {'username': 'mhl', 'machines': ['foo', 'baz']}
x  # {'username': 'admin', 'machines': ['foo', 'baz']}

x = {'username': 'admin', 'machines': ['foo', 'bar', 'baz']}
y = x.copy()
y['username'] = 'mhl'
y['machines'] = 'bar'
y  # {'username': 'mhl', 'machines': 'bar'}
x  # {'username': 'admin', 'machines': ['foo', 'bar', 'baz']}
\end{pyc}

(个人理解:似乎只是复制了指向,如果指向的内容是可以修改的,那么将会同步修改,如果指向的值变化了,那么就不会同步变化。)

当替换副本中的值时,原件不受影响。然而,如果修改副本中的值(就地修改而不是替换),原件也将发生变化,因为原件指向的也是被修改的值(如这个示例中的'machines'列表所示)。

为避免这种问题,一种办法是执行深复制,即同时复制值及其包含的所有值,等等。为此,可使用模块copy中的函数deepcopy。
\begin{pyc}
from copy import deepcopy
d = {'names': ['Alfred', 'Bertrand']}
c = d.copy()
dc = deepcopy(d)
d['names'].append('Clive')
c  # {'names': ['Alfred', 'Bertrand', 'Clive']}
dc  # {'names': ['Alfred', 'Bertrand']}
\end{pyc}

\subsubsection{fromkeys}
方法 fromkeys 创建一个新字典,其中包含指定的键,且每个键对应的值都是None(默认)。如果你不想使用默认值None,可提供特定的值。
\begin{pyc}
{}.fromkeys(['name', 'age'])  # {'name': None, 'age': None}
dict.fromkeys(['name', 'age'])  # {'name': None, 'age': None}

dict.fromkeys(['name', 'age'], '(unknown)')
# {'name': '(unknown)', 'age': '(unknown)'}
\end{pyc}

第1行首先创建了一个空字典,再对其调用方法fromkeys来创建另一个字典,这显得有点多余。你可以不这样做,而是直接对dict(前面说过,dict是所有字典所属的类型。类和类型将调用方法fromkeys。
\subsubsection{get}
方法 get 为访问字典项提供了宽松的环境。通常,如果你试图访问字典中没有的项,将引发错误,而使用get不会这样:
\begin{pyc}
d = {}
print(d['name'])  # KeyError: 'name'
print(d.get('name'))  # None
\end{pyc}

使用get来访问不存在的键时,没有引发异常,而是返回None。你可指定“默认”值,这样将返回你指定的值而不是None,比如:\verb|d.get('name', 'NA')  # 'NA'|。如果字典包含指定的键,get的作用将与普通字典查找相同。
\begin{py}{get方法实例}
people = {
    "Alice": {"Phone": '2341', "addr": 'Foo drive 23'},
    "Beth": {"Phone": '9102', 'addr': 'Bar street 42'},
    'Cecil': {'Phone': '3158', 'addr': 'Baz avenue 90'}
}
labels = {'phone': 'phone number', 'addr': 'address'}
name = input('Name: ')
request = input('Phone number (p) or address (a): ')
key = request
if request == 'p':
    key = 'phone'
if request == 'a':
    key = 'addr'
person = people.get(name, {})
label = labels.get(key, key)
result = person.get(key, 'not available')
print("{}'s {} is {}.".format(name, label, result))
\end{py}

注意到get提高了灵活性,让程序在用户输入的值出乎意料时也能妥善处理。
\subsubsection{items}
方法 items 返回一个包含所有字典项的列表,其中每个元素都为(key, value)的形式。字典项
在列表中的排列顺序不确定。返回值属于一种名为\textbf{字典视图}的特殊类型。字典视图可用于迭代,视图的一个优点是不复制,它们始终是底层字典的反映,即便你修改了底层字典亦如此。
\begin{pyc}
d = {'title': 'Python Web Site', 'url': 'http://www.python.org', 'spam': 0}
d.items()
# dict_items([('title', 'Python Web Site'), ('url', 'http://www.python.org'), ('spam', 0)])
it = d.items()
len(it)  # 3
('spam', 0) in it  # True
d['spam'] = 1
('spam', 0) in it  # False
d['spam'] = 0
('spam', 0) in it  # True
\end{pyc}

\subsubsection{keys}
方法keys返回一个字典视图,其中包含指定字典中的键。
\subsubsection{values}
方法values返回一个由字典中的值组成的字典视图。不同于方法keys,方法values返回的视图可能包含重复的值。
\begin{pyc}
d = {1: 1, 2: 2, 3: 3, 4: 1}
d.values()
# dict_values([1, 2, 3, 1])
\end{pyc}
\subsubsection{pop}
方法pop可用于获取与指定键相关联的值,并将该键-值对从字典中删除。
\begin{pyc}
d = {'x': 1, 'y': 2}
ret = d.pop('x')
ret, d  # (1, {'y': 2})
\end{pyc}
\subsubsection{popitem}
方法popitem类似于\verb|list.pop|,但\verb|list.pop|弹出列表中的最后一个元素,而popitem随机地弹出一个字典项,因为字典项的顺序是不确定的,没有“最后一个元素”的概念。如果你要以高效地方式逐个删除并处理所有字典项,这可能很有用,因为这样无需先获取键列表。
\begin{pyc}
d = {'title': 'Python Web Site', 'url': 'http://www.python.org', 'spam': 0}
ret = d.popitem()
ret, d
# (('spam', 0), {'title': 'Python Web Site', 'url': 'http://www.python.org'})
\end{pyc}

\warning{
    如果希望方法 popitem以可预测的顺序弹出字典项,请参阅模块collections中的OrderedDict类。
}

\subsubsection{setdefault\label{setdefault}}
方法setdefault有点像get,因为它也获取与指定键相关联的值,但除此之外,setdefault还在字典不包含指定的键时,在字典中添加指定的键-值对。
\begin{pyc}
d = {}
d.setdefault('name', 'NA')  # 'NA'
d  # {'name': 'NA'}
d['name'] = 'Gumby'
d.setdefault('name', 'NA')  # 'Gumby'
d  # {'name': 'Gumby'}
\end{pyc}
指定的键不存在时,setdefault返回指定的值并相应地更新字典。如果指定的键存在,就返回其值,并保持字典不变。与get一样,值是可选的;如果没有指定,默认为None。

\warning{
    如果希望有用于整个字典的全局默认值,请参阅模块collections中的defaultdict类。
}
\subsubsection{update}
方法update使用一个字典中的项来更新另一个字典。
\begin{pyc}
d = {
    'title': 'Python Web Site',
    'url': 'http://www.python.org',
    'changed': 'Mar 14 22:09:15 MET 2016'
}
x = {'title': 'Python Language Website', 'Others': 'Others'}
d.update(x)
d
# {'title': 'Python Language Website',
#  'url': 'http://www.python.org',
#  'changed': 'Mar 14 22:09:15 MET 2016',
#  'Others': 'Others'}
\end{pyc}
\important{对于通过参数提供的字典,将其项添加到当前字典中。如果当前字典包含键相同的项,就替换它}。

可像调用本章前面讨论的函数dict(类型构造函数)那样调用方法update。这意味着调用update时,可向它提供一个映射、一个由键值对组成的序列(或其他可迭代对象)或关键字参数。
\section{小结}
本章介绍了如下内容。
\begin{dinglist}{43}
    \item 映射:映射让你能够使用任何不可变的对象(最常用的是字符串和元组)来标识其元素。Python只有一种内置的映射类型,那就是字典。
    \item 将字符串格式设置功能用于字典:要对字典执行字符串格式设置操作,不能使用format和命名参数,而必须使用\verb|format_map|。
    \item 字典方法
\end{dinglist}

\subsection{函数汇总}
\begin{table}[H]
    \caption{字典的方法汇总}
    \begin{tabularx}{\textwidth}{lX}
        \hline
        方法         & 描述                                                                \\
        \hline
        clear      & 方法clear删除所有的字典项,这种操作是就地执行的,因此什么都不返回(或者说返回None)。                   \\
        copy       & 返回一个新字典,其包含的键-值对与原来的字典相同(这个方法执行的是浅复制,因为值本身是原件,而非副本)。              \\
        fromkeys   & 创建一个新字典,其中包含指定的键,且每个键对应的值都是None(默认)。如果你不想使用默认值None,可提供特定的值。       \\
        get        & 获取字典对应键的值,不存在的键也不会报错,而是返回None                                     \\
        items      & 返回一个包含所有字典项的列表,其中每个元素都为(key, value)的形式。                           \\
        keys       & 返回一个字典视图,其中包含指定字典中的键。                                             \\
        values     & 返回一个由字典中的值组成的字典视图。                                                \\
        pop        & 获取与指定键相关联的值,并将该键-值对从字典中删除。                                        \\
        popitem    & 随机地弹出一个字典项,并返回弹出的键值对                                              \\
        setdefault & 像get,因为它也获取与指定键相关联的值,但除此之外,setdefault还在字典不包含指定的键时,在字典中添加指定的键-值对。 \\
        update     & 使用一个字典中的项来更新另一个字典。                                                \\
        \hline
    \end{tabularx}
\end{table}