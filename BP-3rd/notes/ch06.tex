\chapter{抽象\label{chapter06}}
本章介绍如何将语句组合成函数,这让你能够告诉计算机如何完成任务,且只需说一次,无需反复向计算机传达详细指令。
\section{懒惰是一种美德}
这里说的懒不是贬义词,而是说不做无谓的工作。抽象的方式是告诉计算机去做某件事,但是具体怎么做不需要指定,指定做的时候去调用就可以。

\section{抽象和结构}
抽象可节省人力,但实际上还有个更重要的优点:抽象是程序能够被人理解的关键所在(无论对编写程序还是阅读程序来说,这都至关重要)。计算机本身喜欢具体而明确的指令,但人通常不是这样的。

\begin{pyc}
page = download_page()
freqs = compute_frequencies(page)
for word, freq in freqs:
    print(word, freq)
\end{pyc}
看到这些代码,任何人都知道这个程序是做什么的。然而,至于具体该如何做,你未置一词。你只是让计算机去下载网页并计算使用频率,至于这些操作的具体细节,将在其他地方(独立的\textbf{函数定义})中给出。

\section{自定义函数}
函数执行特定的操作并返回一个值\footnote{实际上,在Python中并非所有的函数都返回值,没有返回值的为None},你可以调用它(调用时可能需要提供一些参数------放在圆括号中的内容)。一般而言,要判断某个对象是否可调用,可使用内置函数\verb|callable|。

\begin{pyc}
import math
x = 1
y = math.sqrt
callable(x)  # False
callable(y)  # True
\end{pyc}

函数是结构化编程的核心,函数的定义通过def(表示定义函数)定义语句。
\begin{pyc}
def hello(name):
    return 'Hello, ' + name + '!'
\end{pyc}
运行这些代码后,将有一个名为hello的新函数。它返回一个字符串,其中包含向唯一参数指定的人发出的问候语。你可像使用内置函数那样使用这个函数。

\begin{pyc}
hello('World')  # 'Hello, World!'
hello('Gumby')  # 'Hello, Gumby!'
\end{pyc}
\subsection{给函数编写文档}
要给函数编写文档,以确保其他人能够理解,可添加注释(以\#打头的内容)。还有另一种编写注释的方式,就是添加独立的字符串。放在函数开头的字符串称为\textbf{文档字符串}(docstring),将作为函数的一部分存储起来。可以使用\verb|__doc__|访问文档字符串:
\begin{pyc}
def square(x):
    'Calculate the square of the number x'
    return x * x
square.__doc__
# 'Calculate the square of the number x'
\end{pyc}

\begin{tcolorbox}
    \verb|__doc__|是函数的一个属性,属性名中的双下划线表示这是一个特殊的属性。
\end{tcolorbox}
特殊的内置函数help很有用。在交互式解释器中,可使用它获取有关函数的信息,其中包含函数的文档字符串。
\begin{pyc}
help(square)
# Help on function square in module __main__:

# square(x)
#     Calculate the square of the number x
\end{pyc}
\subsection{其实并不是函数的函数}
数学意义上的函数总是返回根据参数计算得到的结果。在Python中,有些函数什么都不返回。在诸如Pascal等的语言中,这样的函数可能另有其名(如过程),但在Python中,函数就是函数,即使它严格来说并非函数。什么都不返回的函数不包含return语句,或者包含return语句,但没有在return后面指定值。

\begin{pyc}
def test():
    print('This is a demo')
    return
    print('This is not')
x = test()  # This is a demo
print(x)  # None
\end{pyc}
这是一个你熟悉的值:None。由此可知,所有的函数都返回值。如果你没有告诉它们该返回什么,将返回None。

\begin{tcolorbox}
    不要让这种默认行为带来麻烦。如果你在if之类的语句中返回值,务必确保其他分支也返回值,以免在调用者期望函数返回一个序列时(举个例子),不小心返回了None。
\end{tcolorbox}
\section{参数魔法}
函数使用起来很简单,创建起来也不那么复杂,但要习惯参数的工作原理就不那么容易了。
\subsection{值从哪里来}
编写函数旨在为当前程序(甚至其他程序)提供服务,你的职责是\notes{确保它在提供的参数正确时完成任务,并在参数不对时以显而易见的方式失败}。
\begin{tcolorbox}
    在def语句中,位于函数名后面的变量通常称为形参,而调用函数时提供的值称为实参,在很重要的情况下,我会将实参称为值,以便将其与类似于变量的形参区分开来,其他情况不区分形参和实参。
\end{tcolorbox}
\subsection{我能修改参数吗}
函数通过参数获得了一系列的值,你能对其进行修改吗?在函数内部给参数赋值对外部没有任何影响。
\begin{pyc}
def try_to_change(n):
    n = 'Mr. Gumby'
name = 'Mrs. Entity'
try_to_change(name)
name  # 'Mrs. Entity'
\end{pyc}

传递并修改参数的效果类似于下面这样:
\begin{tcolorbox}[title=传递并修改参数的效果示意]
    \begin{pyc}
name = 'Mrs. Entity'
n = name
n = 'Mr. Gumby'
name  # 'Mrs. Entity'
    \end{pyc}
\end{tcolorbox}
变量n变了,但变量name没变。同样,在函数内部重新关联参数(即给它赋值)时,函数外部的变量不受影响。

字符串(以及数和元组)是不可变的(immutable),这意味着你不能修改它们(即只能替换为新值)。因此这些类型作为参数没什么可说的。但如果参数为可变的数据结构(如列表)呢?

\begin{pyc}
def change(n):
    n[0] = 'Mr. Gumby'
names = ['Mes. Entity', 'Mrs. Thing']
change(names)
names  # ['Mr. Gumby', 'Mrs. Thing']
\end{pyc}
在这个示例中,也在函数内修改了参数,但这个示例与前一个示例之间存在一个重要的不同。在前一个示例中,只是给局部变量赋了新值,而在这个示例中,修改了变量关联到的列表。这很奇怪吧?

\begin{tcolorbox}[title=等效示意]
    \begin{pyc}
names = ['Mrs. Entity', 'Mrs. Thing']
n = names
n[0] = 'Mr. Gumby'
names  # ['Mr. Gumby', 'Mrs. Thing']
    \end{pyc}
\end{tcolorbox}
\important{将同一个列表赋给两个变量时,这两个变量将同时指向这个列表},就这么简单。要避免这样的结果,必须创建列表的副本。\important{对序列执行切片操作时,返回的切片都是副本}。因此,如果你创建覆盖整个列表的切片,得到的将是列表的副本。
\begin{pyc}
names = ['Mrs. Entity', 'Mrs. Thing']
n = names[:]
n is names  # False
n == names  # True
\end{pyc}
现在n和names包含两个相等但不同的列表。现在如果(像在函数change中那样)修改n,将不会影响names。

\begin{pyc}
def change(n):
    n[0] = 'Mr. Gumby'
names = ['Mrs. Entity', 'Mrs. Thing']
change(names[:])
names  # ['Mrs. Entity', 'Mrs. Thing'] 
\end{pyc}
注意到参数n包含的是副本,因此原始列表是安全的。
\paragraph{为何要修改参数} 在提高程序的抽象程度方面,使用函数来修改数据结构(如列表或字典)是一种不错的方式。抽象的关键在于隐藏所有的更新细节,为此可使用函数。

\paragraph{如果参数是不可变的} 在有些语言(如C++、Pascal和Ada)中,经常需要给参数赋值并让这种修改影响函数外部的变量。在Python中,没法直接这样做,只能修改参数对象本身。但如果参数是不可变的(如数)呢?

不好意思,没办法。在这种情况下,应从函数返回所有需要的值(如果需要返回多个值,就以元组的方式返回它们)。

\begin{pyc}
    def inc(x): return x + 1
    foo = 10
    foo = inc(foo)
    foo  # 11
    \end{pyc}

如果一定要修改参数,可玩点花样,比如将值放在列表中,如下所示,但更清晰的解决方案是返回修改后的值。
\begin{pyc}
def inc(x): x[0] += 1
foo = [10]
inc(foo)
foo  # [11]
\end{pyc}
\subsection{关键字参数和默认值}
前面使用的参数都是位置参数,因为它们的位置至关重要——事实上比名称还重要。本节介绍的技巧让你能够完全忽略位置。要熟悉这种技巧需要一段时间,但随着程序规模的增大,你很快就会发现它很有用。

有时候,参数的排列顺序可能难以记住,尤其是参数很多时。为了简化调用工作,可指定参数的名称。
\begin{pyc}
def hello(greeting, name):
    print('{}, {}!'.format(greeting, name))
hello('Hello', 'Python')  # Hello, Python!
\end{pyc}
像这样使用名称指定的参数称为关键字参数,主要优点是有助于澄清各个参数的作用。这样,
函数调用不再像下面这样怪异而神秘:

\verb|store('Mr. Brainsample', 10, 20, 13, 5)|
关键字参数最大的优点在于,可以指定默认值。
\begin{pyc}
def hello(greeting='Hello', name='World'):
    print('{}, {}!'.format(greeting, name))
\end{pyc}
像这样给参数指定默认值后,调用函数时可不提供它!可以根据需要,一个参数值也不提供、提供部分参数值或提供全部参数值。

你可结合使用位置参数和关键字参数,但\important{必须先指定所有的位置参数},否则解释器将不知道它们是哪个参数(即不知道参数对应的位置)。
\begin{tcolorbox}
    通常不应结合使用位置参数和关键字参数,除非你知道这样做的后果。一般而言,除非必不可少的参数很少,而带默认值的可选参数很多,否则不应结合使用关键字参数和位置参数。
\end{tcolorbox}
\subsection{收集参数}
有时候,允许用户提供任意数量的参数很有用。
\begin{pyc}
def print_params(*params):
    print(params)
print_params('Testing')  # ('Testing',)
\end{pyc}
参数前面的星号将提供的所有值都放在一个元组中,也就是将这些值收集起来。赋值时带星号的变量收集多余的值。它收集的是列表而不是元组中多余的值,但除此之外,这两种用法很像。
\begin{pyc}
def print_params_2(title, *params):
    print(title)
    print(params)
print_params_2('Title', 1, 2, 3)
# Title
# (1, 2, 3)
\end{pyc}
因此星号意味着收集余下的位置参数。如果没有可供收集的参数,params将是一个空元组。

与赋值时一样,带星号的参数也可放在其他位置(而不是最后),但不同的是,在这种情况下你需要做些额外的工作:\important{使用名称来指定后续参数}。
\begin{pyc}
def in_the_middle(x, *y, z):
    print(x, ',', y, ',', z)
in_the_middle(1, 2, 3, 4, 6, 7, z=3)  # 1 , (2, 3, 4, 6, 7) , 3
in_the_middle(1, 2, 3, 4, 5)
# in_the_middle() missing 1 required keyword-only argument: 'z'
\end{pyc}
星号不会收集关键字参数,要收集关键字参数,可使用两个星号,这样得到的是一个字典而不是元组。
\begin{pyc}
print_params_2('Hmm...', something=42)
# print_params_2() got an unexpected keyword argument 'something'

def print_params_3(**params):
    print(params)
print_params_3(x=1, y=2, z=3)  # {'x': 1, 'y': 2, 'z': 3}
\end{pyc}

结合使用这些技术:
\begin{pyc}
def print_params_4(x, y, *pos_params, z=1, **key_params):
    print(x, y, z)
    print(pos_params)
    print(key_params)
print_params_4(1, 2, 7, 11, z=3, foo=5, bar=13)
# 1 2 3
# (7, 11)
# {'foo': 5, 'bar': 13}
\end{pyc}

\subsection{分配参数}
前面介绍了如何将参数收集到元组和字典中,但用同样的两个运算符(*和**)也可执行相反的操作。与收集参数相反的操作是什么呢?

\verb|def add(x, y): return x + y|,同时假设还有一个元组,其中包含两个你要相加的数,\verb|params = (1, 2)|,与前面执行的操作差不多是相反的:不是收集参数,而是分配参数。这是通过在调用函数(而不是定义函数)时使用运算符*实现的,\verb|add(*params)|。

这种做法也可用于参数列表的一部分,条件是这部分位于参数列表末尾。通过使用运算符**,可将字典中的值分配给关键字参数。
\begin{pyc}
def hello_3(name='CUI', greeting='Hello'):
    print('{}, {}'.format(greeting, name))
params = {'name': 'Sir Robin', 'greeting': 'Well met'}
hello_3(**params)  # Well met, Sir Robin
\end{pyc}
如果在定义和调用函数时都使用*或**,将只传递元组或字典,还不如不使用它们,因此,\notes{只有在定义函数(允许可变数量的参数)或调用函数时(拆分字典或序列)使用,星号才能发挥作用}。
\subsection{练习使用参数}
首先定义几个函数:
\begin{pyc}
def story(**kwds):
    return 'Once upon a time, there was a '\
        '{job} called {name}.'.format_map(kwds)
def power(x, y, *others):
    if others:
        print('Received redundant parameters: ', others)
    return pow(x, y)
def interval(start, stop=None, step=1):
    'Imitates range() for step > 0'
    if stop is None:
        start, stop = 0, start
    result = []
    i = start
    while i < stop:
        result.append(i)
        i += step
    return result
\end{pyc}
调用这些函数:
\begin{pyc}
print(story(job='king', name='Gumby'))
# Once upon a time, there was a king called Gumby.
print(story(name='Sir Robin', job='brave knight'))
# Once upon a time, there was a brave knight called Sir Robin.
params = {'job': 'language', 'name': 'Python'}
print(story(**params))
# Once upon a time, there was a language called Python
del params['job']
print(story(job='stroke of genius', **params))
# Once upon a time, there was a stroke of genius called Python.
power(2, 3)  # 8
power(3, 2)  # 9
power(y=3, x=2)  # 8
params = (5, ) * 2
power(*params)
power(3, 3, 'Hello, world!')
# Received redundant parameters:  ('Hello, world!',)
# 27
interval(10)  # [0, 1, 2, 3, 4, 5, 6, 7, 8, 9]
interval(1, 5)  # [1, 2, 3, 4]
interval(3, 12, 4)  # [3, 7, 11]
power(*interval(3, 7))
# Received redundant parameters:  (5, 6)
# 81
\end{pyc}

\section{作用域}
变量到底是什么呢?可将其视为指向值的名称。因此,执行赋值语句\verb|x = 1|后,名称x指向值1。这几乎与使用字典时一样(字典中的键指向值),只是你使用的是“看不见”的字典。实际上,这种解释已经离真相不远。有一个名为vars的内置函数,它返回这个不可见的字典:
\begin{pyc}
x = 1
scope = vars()
scope['x']
scope['x'] += 1
x  # 2
\end{pyc}
\begin{tcolorbox}[title=警告]
    一般而言,不应修改vars返回的字典,因为根据Python官方文档的说法,这样做的结果是不确定的。换而言之,可能得不到你想要的结果。
\end{tcolorbox}
这种“看不见的字典”称为命名空间或作用域。那么有多少个命名空间呢?除全局作用域外,每个函数调用都将创建一个。
\begin{pyc}
def foo(): x = 42
x = 1
foo()
x  # 1
\end{pyc}

在这里,函数foo修改(重新关联)了变量x,但当你最终查看时,它根本没变。这是因为调用foo时创建了一个新的命名空间,供foo中的代码块使用。赋值语句x = 42是在这个内部作用域(局部命名空间)中执行的,不影响外部(全局)作用域内的x。在函数内使用的变量称为局部变量(与之相对的是全局变量)。参数类似于局部变量,因此参数与全局变量同名不会有任何问题。

但如果要在函数中访问全局变量呢?如果只是想读取这种变量的值(不重新关联它),通常不会有任何问题。
\begin{pyc}
def combine(parameter): print(parameter + external)
external = 'berry'
combine('Shrub')  # Shrubberry
\end{pyc}

\warning{
    像这样访问全局变量是众多bug的根源。务必慎用全局变量。
}

\begin{tcolorbox}[title=遮盖的问题, breakable]
读取全局变量的值通常不会有问题,但还是存在出现问题的可能性。如果有一个局部变量或参数与你要访问的全局变量同名,就无法直接访问全局变量,因为它被局部变量遮住了。

如果需要,可使用函数globals来访问全局变量。这个函数类似于vars,\tips{返回一个包含全局变量的字典}。(locals返回一个包含局部变量的字典。)例如,在前面的示例中,如果有一个名为parameter的全局变量,就无法在函数combine中访问它,因为有一个与之同名的参数。然而,必要时可使\verb|globals()['parameter']|来访问它。

\begin{pyc}
def combine(parameter):
    print(parameter + globals()['parameter'])
parameter = 'berry'
combine('Shrub')
\end{pyc}
\end{tcolorbox}

重新关联全局变量(使其指向新值)是另一码事。在函数内部给变量赋值时,该变量默认为局部变量,除非你明确地告诉Python它是全局变量。
\begin{pyc}
x = 1
def change_global():
    global x
    x += 1
change_global()
x  # 2
\end{pyc}

\begin{tcolorbox}[title=作用域嵌套]
Python函数可以嵌套,即可将一个函数放在另一个函数内。嵌套通常用处不大,但有一个很突出的用途:使用一个函数来创建另一个函数。这意味着可像下面这样编写函数:
\begin{pyc}
def multiplier(factor):
    def multiplierByFactor(number):
        return number * factor
    return multiplierByFactor
\end{pyc}
在这里,一个函数位于另一个函数中,且外面的函数返回里面的函数。也就是返回一个函数,而不是调用它。重要的是,返回的函数能够访问其定义所在的作用域。换而言之,它携带着自己所在的环境(和相关的局部变量)!

每当外部函数被调用时,都将重新定义内部的函数,而变量factor的值也可能不同。由于Python的嵌套作用域,可在内部函数中访问这个来自外部局部作用域(multiplier)的变
量,如下所示:
\begin{pyc}
double = multiplier(2)
double(5)  # 10
triple = multiplier(3)
triple(3)  # 9
multiplier(5)(4)  # 20
\end{pyc}

像multiplyByFactor这样存储其所在作用域的函数称为闭包。

通常,不能给外部作用域内的变量赋值,但如果一定要这样做,可使用关键字nonlocal。这个关键字的用法与global很像,让你能够给外部作用域(非全局作用域)内的变量赋值。
\end{tcolorbox}
\section{递归}
函数可调用其他函数,但可能让你感到惊讶的是,函数还可调用自己。简单地说,递归意味着引用(这里是调用)自身。下面是一个常见的递归定义:

递归[名词]:参见``递归'';

下面是一个递归式函数定义:
\begin{pyc}
def recursion():
    return recursion()
\end{pyc}
这个定义显然什么都没有做。如果你运行它,结果将如何呢?你将发现运行一段时间后,这个程序崩溃了(引发异常)。从理论上说,这个程序将不断运行下去,但每次调用函数时,都将消耗一些内存。因此函数调用次数达到一定的程度(且之前的函数调用未返回)后,将耗尽所有的内存空间,导致程序终止并显示错误消息“超过最大递归深度”。

这个函数中的递归称为无穷递归(就像以while True打头且不包含break和return语句的循环被称为无限循环一样),因为它从理论上说永远不会结束。你想要的是能对你有所帮助的递归函数,这样的递归函数通常包含下面两部分。
\begin{dinglist}{43}
    \item \textbf{基线条件}(针对最小的问题):满足这种条件时函数将直接返回一个值。
    \item \textbf{递归条件}:包含一个或多个调用,这些调用旨在解决问题的一部分。
\end{dinglist}

这里的关键是,通过将问题分解为较小的部分,可避免递归没完没了,因为问题终将被分解成基线条件可以解决的最小问题。

前面说过,每次调用函数时,都将为此创建一个新的命名空间。这意味着函数调用自身时,是两个不同的函数[更准确地说,是不同版本(即命名空间不同)的同一个函数]在交流。你可将此视为两个属于相同物种的动物在彼此交流。

\subsection{两个经典案例:阶乘和幂}
本节探讨两个经典的递归函数。假设你要计算数字n的阶乘,关键在于阶
乘的数学定义,可表述如下。
\begin{dinglist}{43}
    \item 1的阶乘为1。
    \item 对于大于1的数字n,其阶乘为n  1的阶乘再乘以n
\end{dinglist}
\begin{pyc}
def factorial(n):
    if n == 1:
        return 1
    else:
        return n * factorial(n - 1)
\end{pyc}

再来看一个示例。假设你要计算幂,就像内置函数pow和运算符**所做的那样。这是一个非常简单的小型函数,但也可将定义修改成递归式的。
\begin{dinglist}{43}
    \item 对于任何数字x,$power(x, 0)$都为1。
    \item $n>0$时,$power(x, n)$为$power(x, n-1)$与$x$的乘积。
\end{dinglist}
\begin{pyc}
def power(x, n):
    if n == 0:
        return 1
    else:
        return x * power(x, n - 1)
power(2, 10)  # 1024
\end{pyc}

\begin{tcolorbox}
    如果函数或算法复杂难懂,在实现前用自己的话进行明确的定义将大有裨益。以这种``准编程语言”编写的程序通常称为\textbf{伪代码}。
\end{tcolorbox}

\subsection{另一个经典案例:二分查找}
一个常见的问题是:指定的数字是否包含在已排序的序列中?如果包含,在什么位置?

先来回顾一下定义,确保你知道该如何做。
\begin{dinglist}{43}
    \item 如果上限和下限相同,就说明它们都指向数字所在的位置,因此将这个数字返回。
    \item 否则,找出区间的中间位置(上限和下限的平均值),再确定数字在左半部分还是右半部分。然后在继续在数字所在的那部分中查找。
\end{dinglist}

在这个递归案例中,关键在于元素是经过排序的。(请注意,这种查找算法返回数字应该在的位置。如果这个数字不在序列中,那么这个位置上的自然是另一个数字。)

\begin{pyc}
def search(sequence, number, lower=0, upper=None):
    if upper is None:
        upper = len(sequence) - 1
    if lower == upper:
        assert number == sequence[upper]
        return upper
    else:
        middle = (upper + lower) // 2
        if number > sequence[middle]:
            return search(sequence, number, middle + 1, upper)
        else:
            return search(sequence, number, lower, middle)
\end{pyc}

下面来看看这是否可行:
\begin{pyc}
seq = [34, 67, 8, 123, 4, 100, 95]
seq.sort()
search(seq, 34)  # 2
search(seq, 100)  # 5
\end{pyc}

然而,为何要如此麻烦呢?首先,你可使用列表方法index来查找。其次,即便你要自己实现这种功能,也可创建一个循环,让它从序列开头开始迭代,直至找到指定的数字。

确实,使用index挺好,但使用简单循环可能效率低下。前面说过,要在100个数字中找到指定的数字,只需问7次;但使用循环时,在最糟的情况下需要问100次。你可能觉得``没什么大不了的”。但如果列表包含100 000 000 000 000 000 000 000 000 000 000 000个元素(对Python列表来说,这样的长度可能不现实),使用循环也将需要问这么多次,情况开始变得“很大”了。然而,如果使用二分查找,只需问117次。
\begin{tcolorbox}[breakable]
    实际上,模块bisect提供了标准的二分查找实现。
\end{tcolorbox}

效率非常高吧\footnote{事实上,在可观察到的宇宙中,包含的粒子数大约为1087个。要找出其中的一个粒子,只需问大约290次!}

\begin{tcolorbox}[title=函数式编程, breakable]
    至此,你可能习惯了像使用其他对象(字符串、数、序列等)一样使用函数:将其赋给变量,将其作为参数进行传递,以及从函数返回它们。在有些语言(如 scheme 和 Lisp)中,几乎所有的任务都是以这种方式使用函数来完成的。在 Python 中,通常不会如此倚重函数(而是创建自定义对象),但完全可以这样做。

    Python提供了一些有助于进行这种函数式编程的函数:map、filter和reduce。在较新的Python版本中,函数map和filter的用途并不大,应该使用列表推导来替代它们。你可使用map将序列的所有元素传递给函数。
    \begin{pyc}
list(map(str, range(10)))
# ['0', '1', '2', '3', '4', '5', '6', '7', '8', '9']
[str(i) for i in range(10)]
    \end{pyc}

    你可使用filter根据布尔函数的返回值来对元素进行过滤。
    \begin{pyc}
def func(x):
    return x.isalnum()
seq = ['foo', 'bar', 'x41', '***', '?!']
list(filter(func, seq))  # ['foo', 'bar', 'x41']
    \end{pyc}
    
    就这个示例而言,如果转而使用列表推导,就无需创建前述自定义函数。

    \verb|[x for x in seq if x.isalnum()]|

    实际上,Python提供了一种名为lambda表达式的功能\footnote{lambda来源于希腊字母,在数学中用于表示匿名函数。},让你能够创建内嵌的简单函数(主要供map、filter和reduce使用)。

    \verb|list(filter(lambda x: x.isalnum(), seq))|

    要使用列表推导来替换函数reduce不那么容易,而这个函数提供的功能即便能用到,也用得不多。\notes{它使用指定的函数将序列的前两个元素合二为一,再将结果与第3个元素合二为一,依此类推,直到处理完整个序列并得到一个结果}。例如,如果你要将序列中的所有数相加,可结合使用reduce和\verb|lambda x, y: x+y|\footnote{实际上,可不使用这个lambda函数,而是导入模块operator中的函数add(这个模块包含对应于每个内置运算符的函数)。与使用自定义函数相比,使用模块operator中的函数总是效率更高。}。

    \begin{pyc}
numbers = [72, 101, 108, 108, 111, 44, 32, 119, 111, 114, 108, 100, 33]
from functools import reduce
reduce(lambda x, y: x + y, numbers)  # 1161
    \end{pyc}
\end{tcolorbox}
\section{小结}
本章介绍了抽象的基本知识以及函数。
\begin{dinglist}{43}
    \item 抽象:抽象是隐藏不必要细节的艺术。
    \item 函数定义:函数是使用def语句定义的。函数由语句块组成,它们从外部接受值(参数),并可能返回一个或多个值(计算结果)。
    \item 参数:函数通过参数(调用函数时被设置的变量)接收所需的信息。在Python中,参数有两类:位置参数和关键字参数。通过给参数指定默认值,可使其变成可选的。
    \item 作用域:变量存储在作用域(也叫命名空间)中。在Python中,作用域分两大类:全局作用域(globals)和局部作用域(locals)。作用域可以嵌套(nonlocal)。
    \item 递归:函数可调用自身,这称为递归。可使用递归完成的任何任务都可使用循环来完成,但有时使用递归函数的可读性更高。
    \item 函数式编程:Python提供了一些函数式编程工具,其中包括lambda表达式以及函数map、filter和reduce。
\end{dinglist}
