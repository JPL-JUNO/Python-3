\chapter{开箱即用\label{chapter10}}
\subsection{包}
为组织模块,可将其编组为包(package)。包其实就是另一种模块,但有趣的是它们可包含其他模块。模块存储在扩展名为.py的文件中,而包则是一个目录。要被Python视为包,目录必须包含文件\verb|__init__.py|。如果像普通模块一样导入包,文件\verb|__init__.py|的内容就将是包的内容。

要将模块加入包中,只需将模块文件放在包目录中即可。你还可以在包中嵌套其他包。例如,要创建一个名为drawing的包,其中包含模块shapes和colors,需要创建如\autoref{packageLayout}所示的文件和目录(UNIX路径名)。

\begin{table}
    \centering
    \caption{一种简单的包布局}
    \label{packageLayout}
    \begin{tabular}{ll}
        \hline
        文件/目录                               & 描 述            \\
         \hline
        \verb|~/python/|                    & PYTHONPATH中的目录 \\
        \verb|~/python/drawing/|            & 包目录(包drawing)  \\
        \verb|~/python/drawing/__init__.py| & 包代码(模块drawing) \\
        \verb|~/python/drawing/colors.py|   & 模块colors       \\
        \verb|~/python/drawing/shapes.py|   & 模块shapes       \\
        \hline
    \end{tabular}
\end{table}

完成这些准备工作后,下面的语句都是合法的:
\begin{pyc}
import drawing
import drawing.colors
from drawing import shapes
\end{pyc}
执行第1条语句后,便可使用目录drawing中文件\verb|__init__.py|的内容,但不能使用模块shapes和colors的内容。执行第2条语句后,便可使用模块colors,但只能通过全限定名drawing.colors来使用。执行第3条语句后,便可使用简化名(即shapes)来使用模块shapes。请注意,这些语句只是示例,并不用像这里做的那样,先导入包再导入其中的模块。换而言之,完全可以只使用第2条语句,第3条语句亦如此。
\section{探索模块}
这是一种很有用的技能,因为你编写程序可能会遇到很多很有用的模块。
\subsection{模块包含什么}
要探索模块,最直接的方式是使用Python解释器进行研究。假设导入copy标准模块\verb|import copy|。
\subsubsection{使用dir}
要查明模块包含哪些东西,可使用函数dir,它列出对象的所有属性(对于模块,它列出所有的函数、类、变量等)。如果将dir(copy)的结果打印出来,将是一个很长的名称列表(请试试看)。在这些名称中,有几个以下划线打头。根据约定,这意味着它们并非供外部使用。有鉴于此,使用一个简单的列表推导将这些名称过滤掉。
\begin{pyc}
[n for n in dir(copy) if not n.startswith('_')]
# ['Error', 'copy', 'deepcopy', 'dispatch_table', 'error']
\end{pyc}

\subsection{变量\_\_all\_\_}
可以使用列表推导来猜测可在模块copy中看到哪些内容,然而可直接咨询这个模块来获得正确的答案。你可能注意到了,在dir(copy)返回的完整清单中,包含名称\verb|__all__|。这个变量包含一个列表,它与前面使用列表推导创建的列表类似,但是在模块内部设置的。下面来看看这个列表包含的内容:

\verb|copy.__all__  # ['Error', 'copy', 'deepcopy']|

这个\verb|__all__|列表是怎么来的呢?为何要提供它?第一个问题很容易回答:它是在模块copy中像下面这样设置的(这些代码是直接从copy.py复制而来的):

\begin{pyc}
__all__ = ['Error', 'copy', 'deepcopy']
\end{pyc}

为何要提供它呢?旨在定义模块的公有接口。具体地说,它告诉解释器从这个模块导入所有的名称意味着什么。因此,如果你使用\verb|from copy import *|。将只能得到变量\verb|__all__|中列出的3个函数。要导入\verb|dispatch_table|,必须显式地:导入copy并使用\verb|copy.dispatch_table|;或者使用

\verb|from copy import dispatch_table|。

编写模块时,像这样设置\verb|__all__|也很有用。因为模块可能包含大量其他程序不需要的变量、函数和类,比较周全的做法是将它们过滤掉。如果不设置\verb|__all__|,则会在以import *方式导入时,导入所有不以下划线打头的全局名称。

\subsection{使用 help 获取帮助}
前面一直在巧妙地利用你熟悉的各种Python函数和特殊属性来探索模块copy。对这种探索来说,交互式解释器是一个强大的工具,因为使用它来探测模块时,探测的深度仅受限于你对Python语言的掌握程度。然而,有一个标准函数可提供你通常需要的所有信息,它就是help。

\begin{pyc}
help(copy.copy)
# Help on function copy in module copy:

# copy(x)
#     Shallow copy operation on arbitrary Python objects.
    
#     See the module's __doc__ string for more info.
\end{pyc}

上述帮助信息指出,函数copy只接受一个参数x,且执行的是浅复制。文档字符串就是在函数开头编写的字符串,用于对函数进行说明,而函数的属性\verb|__doc__|可能包含这个字符串。从前面的帮助信息可知,模块也可能有文档字符串(它们位于模块的开头),而类也可能如此(位于类的开头)。

实际上,前面的帮助信息是从函数copy的文档字符串中提取的:
\begin{pyc}
print(copy.copy.__doc__)
# Shallow copy operation on arbitrary Python objects.

#     See the module's __doc__ string for more info.
\end{pyc}
相比于直接查看文档字符串,使用help的优点是可获取更多的信息,如函数的特征标(即它接受的参数)。请尝试对模块copy本身调用help,看看将显示哪些信息。这将打印大量的信息,包括对copy和deepcopy之间差别的详细讨论(大致而言,deepcopy(x)创建x的属性的副本并依此类推;而copy(x)只复制x,并将副本的属性关联到x的属性值)。
\subsection{文档}
显然,文档是有关模块信息的自然来源。
\begin{pyc}
print(range.__doc__)
# range(stop) -> range object
# range(start, stop[, step]) -> range object

# Return an object that produces a sequence of integers from start (inclusive)
# to stop (exclusive) by step.  range(i, j) produces i, i+1, i+2, ..., j-1.
# start defaults to 0, and stop is omitted!  range(4) produces 0, 1, 2, 3.
# These are exactly the valid indices for a list of 4 elements.
# When step is given, it specifies the increment (or decrement).
\end{pyc}
这样就获得了函数range的准确描述。

\subsection{使用源代码}
在大多数情况下,前面讨论的探索技巧都够用了。但要真正理解Python语言,可能需要了解一些不阅读源代码就无法了解的事情。事实上,要学习Python,阅读源代码是除动手编写代码外的最佳方式。

实际阅读源代码应该不成问题,但源代码在哪里呢?假设你要阅读标准模块copy的代码,可以在什么地方找到呢?一种办法是像解释器那样通过sys.path来查找,但更快捷的方式是查看模块的特性\verb|__file__|。

\begin{pyc}
print(copy.__file__)
# d:\Python3.11.1\Lib\copy.py
\end{pyc}
如果列出的文件名以.pyc结尾,可打开以.py结尾的相应文件。

\begin{tcolorbox}
在文本编辑器中打开标准库文件时,存在不小心修改它的风险。这可能会破坏文件。因此关闭文件时,千万不要保存你可能对其所做的修改。
\end{tcolorbox}

\section{标准库:一些深受欢迎的模块}
在Python中,短语“开箱即用”(batteries included)最初是由Frank Stajano提出的,指的是Python丰富的标准库。
\subsection{sys}
模块sys让你能够访问与Python解释器紧密相关的变量和函数,\autoref{sys}列出了其中的一些。
\begin{table}
    \centering
    \caption{模块sys中一些重要的函数和变量}
    \label{sys}
    \begin{tabular}{ll}
        \hline
        函数/变量       & 描 述                      \\
        \hline
        argv        & 命令行参数,包括脚本名              \\
        exit([arg]) & 退出当前程序,可通过可选参数指定返回值或错误消息 \\
        modules     & 一个字典,将模块名映射到加载的模块        \\
        path        & 一个列表,包含要在其中查找模块的目录的名称    \\
        platform    & 一个平台标识符,如sunos5或win32    \\
        stdin       & 标准输入流——一个类似于文件的对象        \\
        stdout      & 标准输出流——一个类似于文件的对象        \\
        stderr      & 标准错误流——一个类似于文件的对象        \\
        \hline
    \end{tabular}
\end{table}

变量sys.argv包含传递给Python解释器的参数,其中包括脚本名。

函数sys.exit退出当前程序。(在第8章讨论的try/finally块中调用它时,finally子句依然会执行。)你可向它提供一个整数,指出程序是否成功,这是一种UNIX约定。在大多数情况下,使用该参数的默认值(0,表示成功)即可。也可向它提供一个字符串,这个字符串将成为错误消息,对用户找出程序终止的原因很有帮助。在这种情况下,程序退出时将显示指定的错误消息以及一个表示失败的编码。

映射sys.modules将模块名映射到模块(仅限于当前已导入的模块)。

变量sys.path在本章前面讨论过,它是一个字符串列表,其中的每个字符串都是一个目录名,执行import语句时将在这些目录中查找模块。

变量sys.platform(一个字符串)是运行解释器的“平台”名称。这可能是表示操作系统的名称(如sunos5或win32),也可能是表示其他平台类型(如Java虚拟机)的名称(如java1.4.0)——如果你运行的是Jython。

变量sys.stdin、sys.stdout和sys.stderr是类似于文件的流对象,表示标准的UNIX概念:标准输入、标准输出和标准错误。简单地说,Python从sys.stdin获取输入(例如,用于input中),并将输出打印到sys.stdout。

举个例子,来看看按相反顺序打印参数的问题。从命令行调用Python脚本时,你可能指定一些参数,也就是所谓的命令行参数。这些参数将放在列表sys.argv中,其中sys.argv[0]为Python脚本名。
\begin{pyc}
# reverseargs.py
import sys
args = sys.argv[1:]
args.reverse()
print(' '.join(args))
\end{pyc}
创建了一个sys.argv的副本。也可修改sys.argv,但一般而言,不这样做更安全,因为程序的其他部分可能依赖于包含原始参数的sys.argv。另外,注意到我跳过了sys.argv的第一个元素,即脚本的名称。

在命令行中输入\verb|python reverseargs.py this is a test|,就能得到\verb|test a is this|。

\section{os}
模块os让你能够访问多个操作系统服务。它包含的内容很多,表10-3只描述了其中几个最有用的函数和变量。除此之外,os及其子模块os.path还包含多个查看、创建和删除目录及文件的函数,以及一些操作路径的函数。

\begin{table}
    \centering
    \caption{模块os中一些重要的函数和变量}
    \label{os}
    \begin{tabular}{ll}
        \hline
        函数/变量           & 描 述                                         \\
        \hline
        environ         & 包含环境变量的映射                                   \\
        system(command) & 在子shell中执行操作系统命令                            \\
        sep             & 路径中使用的分隔符                                   \\
        pathsep         & 分隔不同路径的分隔符                                  \\
        linesep         & 行分隔符(\verb|'\n'|、\verb|'\r'|或\verb|'\r\n'|) \\
        urandom(n)      & 返回n个字节的强加密随机数据                              \\
        \hline
    \end{tabular}
\end{table}

映射os.environ包含本章前面介绍的环境变量。例如,要访问环境变量PYTHONPATH,可使用表达式os.environ['PYTHONPATH']。这个映射也可用于修改环境变量,但并非所有的平台都支持这样做。

函数os.system用于运行外部程序。还有其他用于执行外部程序的函数,如execv和popen。前者退出Python解释器,并将控制权交给被执行的程序,而后者创建一个到程序的连接(这个连接类似于文件)。

变量os.sep是用于路径名中的分隔符。在UNIX(以及macOS的命令行Python版本)中,标准分隔符为/。在Windows中,标准分隔符为\verb|\\|(这种Python语法表示单个反斜杠)。

可使用os.pathsep来组合多条路径,就像PYTHONPATH中那样。pathsep用于分隔不同的路径名:在UNIX/macOS中为:,而在Windows中为;。

变量os.linesep是用于文本文件中的行分隔符:在UNIX/OS X中为单个换行符(\verb|\n|),在Windows中为回车和换行符(\verb|\r\n|)。

函数urandom使用随系统而异的“真正”(至少是强加密)随机源。如果平台没有提供这样的随机源,将引发NotImplementedError异常。

\begin{tcolorbox}[title=webbrowser]
函数os.system可用于完成很多任务,但就启动Web浏览器这项任务而言,有一种更佳的解决方案:使用模块webbrowser。这个模块包含一个名为open的函数,让你能够启动启动Web浏览器并打开指定的URL。例如,要让程序在Web浏览器中打开Python网站(启动浏览器或使用正在运行的浏览器,只需像下面这样做:
\begin{pyc}
import webbrowser
webbrowser.open('http://www.python.org')
\end{pyc}
这将弹出指定的网页。
\end{tcolorbox}
\subsection{fileinput}
模块fileinput让你能够轻松地迭代一系列文本文件中的所有行。如果你这样调用脚本(假设是在UNIX命令行中):

\verb|python some_script.py file1.txt file2.txt file3.txt|

就能够依次迭代文件file1.txt到file3.txt中的行。

\autoref{fileinput}描述了模块fileinput中最重要的函数。

\begin{table}
    \centering
    \caption{模块fileinput中一些重要的函数和变量}
    \label{fileinput}
    \begin{tabular}{ll}
        \hline
        函数                                  & 描 述                 \\
        \hline
        input([files[, inplace[, backup]]]) & 帮助迭代多个输入流中的行        \\
        filename()                          & 返回当前文件的名称           \\
        lineno()                            & 返回(累计的)当前行号         \\
        filelineno()                        & 返回在当前文件中的行号         \\
        isfirstline()                       & 检查当前行是否是文件中的第一行     \\
        isstdin()                           & 检查最后一行是否来自sys.stdin \\
        nextfile()                          & 关闭当前文件并移到下一个文件      \\
        close()&关闭序列                                               \\
        \hline
    \end{tabular}
\end{table}

fileinput.input是其中最重要的函数,它返回一个可在for循环中进行迭代的对象。如果要覆盖默认行为(确定要迭代哪些文件),可以序列的方式向这个函数提供一个或多个文件名。还可将参数inplace设置为True(inplace=True),这样将就地进行处理。对于你访问的每一行,都需打印出替代内容,这些内容将被写回到当前输入文件中。就地进行处理时,可选参数backup用于给从原始文件创建的备份文件指定扩展名。

函数fileinput.filename返回当前文件(即当前处理的行所属文件)的文件名。

函数fileinput.lineno返回当前行的编号。这个值是累计的,因此处理完一个文件并接着处理下一个文件时,不会重置行号,而是从前一个文件最后一行的行号加1开始。

函数fileinput.filelineno返回当前行在当前文件中的行号。每次处理完一个文件并接着处理下一个文件时,将重置这个行号并从1重新开始。

函数fileinput.isfirstline在当前行为当前文件中的第一行时返回True,否则返回False。

函数fileinput.isstdin在当前文件为sys.stdin时返回True,否则返回False。

函数fileinput.nextfile关闭当前文件并跳到下一个文件,且计数时忽略跳过的行。这在你知道无需继续处理当前文件时很有用。例如,如果每个文件包含的单词都是按顺序排列的,而你要查找特定的单词,则过了这个单词所在的位置后,就可放心地跳到下一个文件。

函数fileinput.close关闭整个文件链并结束迭代。

\begin{tcolorbox}[title=fileinput小案例]
假设你编写了一个Python脚本,并想给其中的代码行加上行号。鉴于你希望这样处理后程序依然能够正常运行,因此必须在每行末尾以注释的方式添加行号。为让这些行号对齐,可使用字符串格式设置功能。假设只允许每行代码最多包含50个字符,并在第51个字符处开始添加注释。

\begin{pyc}
# numberlines.py
import fileinput

for line in fileinput.input(inplace=True):
    line = line.rstrip()
    num = fileinput.lineno()
    print('{:<50} # {:2d}'.format(line, num))
\end{pyc}
如果像下面这样运行这个程序,并将其作为参数传入:

\verb|$ python numberlines.py numberlines.py|

注意到,程序本身也被修改了,如果像上面这样运行它多次,每行都将包含多个行号。
\begin{pyc}
# numberlines.py                                   #  1
import fileinput                                   #  2
                                                   #  3
for line in fileinput.input(inplace=True):         #  4
    line = line.rstrip()                           #  5
    num = fileinput.lineno()                       #  6
    print('{:<50} # {:2d}'.format(line, num))      #  7
\end{pyc}
\begin{tcolorbox}[title=警告,colback=red!10, colframe=red!50!black]
务必慎用参数inplace,因为这很容易破坏文件。你应在不设置inplace的情况下仔细测试程序(这样将只打印结果),确保程序能够正确运行后再让它修改文件。
\end{tcolorbox}
\end{tcolorbox}
\subsection{集合、堆和双端队列}
\subsubsection{集合}
很久以前,集合是由模块sets中的Set类实现的。在较新的版本中,集合是由内置类set实现的,这意味着你可直接创建集合,而无需导入模块sets。

可使用序列(或其他可迭代对象)来创建集合,也可使用花括号显式地指定。\important{请注意,不能仅使用花括号来创建空集合,因为这将创建一个空字典。}

必须在不提供任何参数的情况下调用set。集合主要用于成员资格检查,因此将忽略重复的元素。与字典一样,集合中元素的排列顺序是不确定的,因此不能依赖于这一点。

\begin{pyc}
set(range(10))  # {0, 1, 2, 3, 4, 5, 6, 7, 8, 9}

type({})  # dict

{0, 1, 2, 3, 0, 1, 2, 3, 4, 5}
# {0, 1, 2, 3, 4, 5}
\end{pyc}
除成员资格检查外,还可执行各种标准集合操作,如并集和交集,为此可使用对整数执行按位操作的运算符。

\begin{pyc}
a = {1, 2, 3}
b = {2, 3, 4}
a.union(b)  # {1, 2, 3, 4}
a | b  # {1, 2, 3, 4}

c = a & b
c.issubset(a)  # True
c <= a  # True

c.issuperset(a)  # False
c >= a  # False

a.intersection(b)  # {2, 3}
a & b  # {2, 3}
a.difference(b)  # {1}
a - b  # {1}

# A | B - A & B
a.symmetric_difference(b)  # {1, 4}
a ^ b  # {1, 4}
a.copy()  # {1, 2, 3}
a.copy() is a  # False
\end{pyc}

另外,还有对应于各种就地操作的方法以及基本方法add和remove。

集合是可变的,因此不能用作字典中的键。另一个问题是,集合只能包含不可变(可散列)的值,因此不能包含其他集合。由于在现实世界中经常会遇到集合的集合,因此这可能是个问题。所幸还有frozenset类型,它表示不可变(可散列)的集合。
\begin{pyc}
a = set()
b = set()
a.add(b)  # TypeError: unhashable type: 'set'
a.add(frozenset(b))
a  # {frozenset()}
\end{pyc}
构造函数frozenset创建给定集合的副本。在需要将集合作为另一个集合的成员或字典中的键时,frozenset很有用。

\subsubsection{堆}

另一种著名的数据结构是堆(heap),它是一种优先队列。优先队列让你能够以任意顺序添加对象,并随时(可能是在两次添加对象之间)找出(并删除)最小的元素。相比于列表方法min,这样做的效率要高得多。

Python没有独立的堆类型,而只有一个包含一些堆操作函数的模块。这个模块名为heapq(其中的q表示队列),它包含6个函数(如\autoref{heapq}所示),其中前4个与堆操作直接相关。必须使用列表来表示堆对象本身。

函数heappush用于在堆中添加一个元素。\important{请注意,不能将它用于普通列表,而只能用于使用各种堆函数创建的列表}。原因是元素的顺序很重要(虽然元素的排列顺序看起来有点随意,并没有严格地排序)。

\begin{table}
    \centering
    \caption{模块heapq中一些重要的函数和变量}
    \label{heapq}
    \begin{tabular}{ll}
        \hline
        函数                   & 描 述             \\
        \hline
        heappush(heap, x)    & 将x压入堆中          \\
        heappop(heap)        & 从堆中弹出最小的元素      \\
        heapify(heap)        & 让列表具备堆特征        \\
        heapreplace(heap, x) & 弹出最小的元素,并将x压入堆中 \\
        nlargest(n, iter)    & 返回iter中n个最大的元素  \\
        nsmallest(n, iter)   & 返回iter中n个最小的元素  \\
        \hline
    \end{tabular}
\end{table}

\begin{pyc}
from random import shuffle
from heapq import *

data = list(range(10))
shuffle(data)
data
heap = []
for n in data:
    heappush(heap, n)

heappush(heap, .5)
heap  # [0, 0.5, 1, 5, 3, 2, 6, 8, 7, 9, 4]
\end{pyc}

元素的排列顺序并不像看起来那么随意。它们虽然不是严格排序的,但必须保证一点:位置i处的元素$x_i$总是大于位置i // 2处的元素$x_{[i // 2]}$(反过来说就是小于$x_{2i}$和$x_{2i+1}$处的元素)。这是底层堆算法的基础,称为堆特征(\textbf{heap property})。

函数heappop弹出最小的元素(总是位于索引0处),并确保剩余元素中最小的那个位于索引0处(保持堆特征)。虽然弹出列表中第一个元素的效率通常不是很高,但这不是问题,因为heappop会在幕后做些巧妙的移位操作。

\begin{pyc}
heappop(heap)  # 0
heappop(heap)  # 0.5
heappop(heap)  # 1
heap  # [2, 4, 3, 5, 6, 7, 8, 9]
\end{pyc}

函数heapify通过执行尽可能少的移位操作将列表变成合法的堆(即具备堆特征)。如果你的堆并不是使用heappush创建的,应在使用heappush和heappop之前使用这个函数。
\begin{pyc}
heap = [5, 8, 0, 3, 6, 7, 9, 1, 4, 2]
heapify(heap)
heap  # [0, 1, 5, 3, 2, 7, 9, 8, 4, 6]
\end{pyc}

函数heapreplace用得没有其他函数那么多。它从堆中弹出最小的元素,再压入一个新元素。相比于依次执行函数heappop和heappush,这个函数的效率更高。

\begin{pyc}
heapreplace(heap, .5)
heap  # [0.5, 1, 5, 3, 2, 7, 9, 8, 4, 6]
heapreplace(heap, 10)
heap  # [1, 2, 5, 3, 6, 7, 9, 8, 4, 10]
\end{pyc}

模块heapq中还有两个函数没有介绍:nlargest(n, iter)和nsmallest(n, iter),:分别用于找出可迭代对象iter中最大和最小的n个元素。这种任务也可通过先排序(如使用函数sorted)再切片来完成,但堆算法的速度更快,使用的内存更少(而且使用起来也更容易)。

\begin{pyc}
data = list(range(1000))
shuffle(data)
nlargest(5, data)  # [999, 998, 997, 996, 995]
nsmallest(5, data)  # [0, 1, 2, 3, 4]
\end{pyc}
\subsubsection{双端队列(及其他集合)}
在需要按添加元素的顺序进行删除时,双端队列很有用。在模块collections中,包含类型deque以及其他几个集合(collection)类型。

与集合(set)一样,双端队列也是从可迭代对象创建的:

\begin{pyc}
from collections import deque
q = deque(range(5))
q.append(5)
q.appendleft(6)
q  # deque([6, 0, 1, 2, 3, 4, 5])
q.pop()  # 5
q.popleft()  # 6
q.rotate(3)
q  # deque([2, 3, 4, 0, 1])

q.rotate(-1)
q  # deque([3, 4, 0, 1, 2])
\end{pyc}

双端队列很有用,因为它支持在队首(左端)高效地附加和弹出元素,而使用列表无法这样做。另外,还可高效地旋转元素(将元素向右或向左移,并在到达一端时环绕到另一端)。双端队列对象还包含方法extend和extendleft,其中extend类似于相应的列表方法,而extendleft类似于appendleft。请注意,用于extendleft的可迭代对象中的元素将按相反的顺序出现在双端队列中。
\begin{pyc}
q.extend([1, 2])
# deque([3, 4, 0, 1, 2, 1, 2])
q.extendleft([1, 2])
q  # deque([2, 1, 3, 4, 0, 1, 2, 1, 2])
\end{pyc}
\subsection{time}
模块time包含用于获取当前时间、操作时间和日期、从字符串中读取日期、将日期格式化为字符串的函数。日期可表示为实数(从“新纪元”1月1日0时起过去的秒数。“新纪元”是一个随平台而异的年份,在UNIX中为1970年),也可表示为包含9个整数的元组。\autoref{timePython}解释了这些整数。例如,元组(2008, 1, 21, 12, 2, 56, 0, 21, 0)表示2008年1月21日12时2分56秒。这一天是星期一,2008年的第21天(不考虑夏令时)。

\begin{table}[H]
    \centering
    \caption{Python日期元组中的字段}
    \label{timePython}
    \begin{tabular}{lll}
        \hline
        索 引 & 字 段 & 值              \\
        \hline
        0   & 年   & 如2000、2001等    \\
        1   & 月   & 范围1~12         \\
        2   & 日   & 范围1~31         \\
        3   & 时   & 范围0~23         \\
        4   & 分   & 范围0~59         \\
        5   & 秒   & 范围0~61         \\
        6   & 星期  & 范围0~6,其中0表示星期一 \\
        7   & 儒略日 & 范围1~366        \\
        8   & 夏令时 & 0、1或-1         \\
        \hline
    \end{tabular}
\end{table}

秒的取值范围为0~61,这考虑到了闰一秒和闰两秒的情况。夏令时数字是一个布尔值(True或False),但如果你使用-1,那么mktime[将时间元组转换为时间戳(从新纪元开始后的秒数)的函数]可能得到正确的值。\autoref{time}描述了模块time中一些最重要的函数。

\begin{table}[H]
    \centering
    \caption{模块time中一些重要的函数和变量}
    \label{time}
    \begin{tabular}{lll}
        \hline
        函 数                        & 描 述                     \\
        \hline
        asctime([tuple])           & 将时间元组转换为字符串             \\
        localtime([secs])          & 将秒数转换为表示当地时间的日期元组       \\
        mktime(tuple)              & 将时间元组转换为当地时间            \\
        sleep(secs)                & 休眠(什么都不做)secs秒          \\
        strptime(string[, format]) & 将字符串转换为时间元组             \\
        time()                     & 当前时间(从新纪元开始后的秒数,以UTC为准) \\
        \hline
    \end{tabular}
\end{table}
函数time.asctime将当前时间转换为字符串,如下所示:
\begin{pyc}
import time
time.asctime()  # 'Sat Feb 18 17:32:01 2023'
\end{pyc}

如果不想使用当前时间,也可向它提供一个日期元组(如localtime创建的日期元组)。要设置更复杂的格式,可使用函数strftime,标准文档对此做了介绍。

函数time.localtime将一个实数(从新纪元开始后的秒数)转换为日期元组(本地时间)。如果要转换为国际标准时间,应使用gmtime。

函数time.mktime将日期元组转换为从新纪元后的秒数,这与localtime的功能相反。

函数time.sleep让解释器等待指定的秒数。

函数time.strptime将一个字符串(其格式与asctime所返回字符串的格式相同)转换为日期元组。(可选参数format遵循的规则与strftime相同,详情请参阅标准文档。)

函数time.time返回当前的国际标准时间,以从新纪元开始的秒数表示。虽然新纪元随平台而异,但可这样进行可靠的计时:存储事件(如函数调用)发生前后time的结果,再计算它们的差。

还有两个较新的与时间相关的模块:datetime和timeit。前者提供了日期和时间算术支持,而后者可帮助你计算代码段的执行时间。“Python库参考手册”提供了有关这两个模块的详细信息。
\subsection{random}
模块random包含生成伪随机数的函数,有助于编写模拟程序或生成随机输出的程序。请注意,虽然这些函数生成的数字好像是完全随机的,但它们背后的系统是可预测的。如果你要求真正的随机(如用于加密或实现与安全相关的功能),应考虑使用模块os中的函数urandom。模块random中的SystemRandom类基于的功能与urandom类似,可提供接近于真正随机的数据。

\autoref{random}列出了这个模块中一些重要的函数。

\begin{table}
    \centering
    \caption{模块random中一些重要的函数和变量}
    \label{random}
    \begin{tabular}{ll}
        \hline
        函 数                              & 描 述                                \\
        \hline
        random()                         & 返回一个$0\sim 1$(含)的随机实数                    \\
        getrandbits(n)                   & 以长整数方式返回n个随机的二进制位                  \\
        uniform(a, b)                    & 返回一个$a\sim b$(含)的随机实数                    \\
        randrange([start], stop, [step]) & 从range(start, stop, step)中随机地选择一个数 \\
        choice(seq)                      & 从序列seq中随机地选择一个元素                   \\
        shuffle(seq[, random])           & 就地打乱序列seq                          \\
        sample(seq, n)                   & 从序列seq中随机地选择n个值不同的元素               \\
        \hline
    \end{tabular}
\end{table}
函数random.random是最基本的随机函数之一,它返回一个0~1(含)的伪随机数。除非这正是你需要的,否则可能应使用其他提供了额外功能的函数。

函数random.getrandbits以一个整数的方式返回指定数量的二进制位。

向函数random.uniform提供了两个数字参数a和b时,它返回一个a~b(含)的随机(均匀分布的)实数。

函数random.randrange是生成随机整数的标准函数。为指定这个随机整数所在的范围,你可像调用range那样给这个函数提供参数。例如,生成奇数或者偶数等等。

函数random.choice从给定序列中随机(均匀)地选择一个元素。

函数random.shuffle随机地打乱一个可变序列中的元素,并确保每种可能的排列顺序出现的概率相同。

函数random.sample从给定序列中随机(均匀)地选择指定数量的元素,并确保所选择元素的值各不相同。

来看几个使用模块random的示例。
\begin{tcolorbox}[title=随机选择一个时间]
\begin{pyc}
from time import *
from random import *
date1 = (2023, 1, 1, 0, 0, 0, -1, -1, -1)
date2 = (2023, 12, 31, 23, 59, 59, -1, -1, -1)
time1 = mktime(date1)
time2 = mktime(date2)
random_time = uniform(time1, time2)
print(asctime(localtime(random_time)))
# Tue Oct 17 09:08:16 2023
\end{pyc}
\end{tcolorbox}

\begin{tcolorbox}[title=询问用户要掷多少个骰子、每个骰子有多少面]
\begin{pyc}
from random import randrange
num = int(input("How many dice?"))
sides = int(input("How many sides per die?"))
sum = 0
for i in range(num):
    sum += randrange(sides) + 1
print('The result is', sum)
# How many dice? 3
# How many sides per die? 6
# The result is 13
\end{pyc}
\end{tcolorbox}

\begin{tcolorbox}[title=用户每次按回车键时都给他发一张牌]
\begin{pyc}
from random import shuffle
values = list(range(1, 11)) + 'Jack Queen King'.split()
suits = 'diamonds clubs hearts spades'.split()
deck = ['{} of {}'.format(v, s) for v in values for s in suits]
shuffle(deck)
while deck:
    input(deck.pop())
    # ignore = input(deck.pop())
\end{pyc}
请注意,如果在交互式解释器中尝试运行这个while循环,那么每当你按回车键时都将打印一个空字符串。这是因为input返回你输入的内容(什么都没有),然后这些内容将被打印出来。在普通程序中,将忽略input返回的值。要在交互式解释器中也忽略input返回的值,只需将其赋给一个你不会再理会的变量,并将这个变量命名为ignore。
\end{tcolorbox}
\subsection{shelve 和 json}
如果需要的是简单的存储方案,模块shelve可替你完成大部分工作——你只需提供一个文件名即可。对于模块shelve,你唯一感兴趣的是函数open。这个函数将一个文件名作为参数,并返回一个Shelf对象,供你用来存储数据。你可像操作普通字典那样操作它(只是键必须为字符串),操作完毕(并将所做的修改存盘)时,可调用其方法close。

\paragraph{一个潜在的陷阱} 至关重要的一点是认识到shelve.open返回的对象并非普通映射。
\begin{pyc}
s = shelve.open('test')
s['x'] = ['a', 'b', 'c']
s['x'].append('d')
s['x']
\end{pyc}
当你查看shelf对象中的元素时,将使用存储版重建该对象,而当你将一个元素赋给键时,该元素将被存储。

要正确地修改使用模块shelve存储的对象,必须将获取的副本赋给一个临时变量,并在修改这个副本后再次存储:
\begin{pyc}
s = shelve.open('test')
s['x'] = ['a', 'b', 'c']
temp = s['x']
temp.append('d')
s['x'] = temp
s['x']
\end{pyc}
还有另一种避免这个问题的办法:将函数open的参数writeback设置为True。这样,从shelf对象读取或赋给它的所有数据结构都将保存到内存(缓存)中,并等到你关闭shelf对象时才将它们写入磁盘中。如果你处理的数据不多,且不想操心这些问题,将参数writeback设置为True可能是个不错的主意。在这种情况下,你必须确保在处理完毕后将shelf对象关闭。为此,一种办法是像处理打开的文件那样,将shelf对象用作上下文管理器。

\begin{pyc}
import sys
import shelve
def store_person(db):
    """
    Query user  for data and store it in the shelf object
    """
    pid = input('Enter unique ID number:')
    person = {}
    person['name'] = input('Enter name:')
    person['age'] = input('Enter age:')
    person['phone'] = input('Enter phone:')
    db[pid] = person
def lookup_person(db):
    """
    Query user for ID and desired field, and fetch the corresponding data from the shelf object
    """
    pid = input('Enter ID number:')
    field = input('What would you like to know? (name, age, phone)')
    field = field.strip().lower()

    print(field.capitalize() + ':', db[pid][field])
def print_help():
    print('The available commands are:')
    print('store : Stores information about a person')
    print('lookup : Looks up a person from ID number')
    print('quit : Save changes and exit')
    print('? : Prints this message')
def enter_command():
    cmd = input('Enter command (? for help): ')
    cmd = cmd.strip().lower()
    return cmd
def main():
    database = shelve.open('database')
    try:
        while True:
            cmd = enter_command()
            if cmd == 'store':
                store_person(database)
            elif cmd == 'lookup':
                lookup_person(database)
            elif cmd == '?':
                print_help()
            elif cmd == 'quit':
                return
    finally:
        database.close()

if __name__ == '__main__':
    main()
\end{pyc}

为确保数据库得以妥善的关闭,使用了try和finally。不知道什么时候就会出现问题,进而引发异常。如果程序终止时未妥善地关闭数据库,数据库文件可能受损,变得毫无用处。通过使用try和finally,可避免这样的情况发生。

\begin{tcolorbox}[breakable]
如果要以这样的格式保存数据,也就是让使用其他语言编写的程序能够轻松地读取它们,可考虑使用JSON格式。Python标准库提供了用于处理JSON字符串(在这种字符串和Python值之间进行转换)的模块json。
\end{tcolorbox}
\subsection{re}
\begin{quotation}
有些人面临问题时会想:“我知道,我将使用正则表达式来解决这个问题。”这让他们面临的问题变成了两个。
\begin{flushright}
---Jamie Zawinski
\end{flushright}
\end{quotation}

