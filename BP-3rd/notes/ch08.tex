\chapter{异常\label{chapter08}}
编写计算机程序时,通常能够区分正常和异常(不正常)情况。异常事件可能是错误(如试图除以零),也可能是通常不会发生的事情。为处理这些异常事件,可在每个可能发生这些事件的地方都使用条件语句。然而,这样做不仅效率低下、缺乏灵活性,还可能导致程序难以卒读。你可能很想忽略这些异常事件,希望它们不会发生,但Python提供功能强大的替代解决方案——\textbf{异常处理机制}。
\section{异常是什么}
Python使用异常对象来表示异常状态,并在遇到错误时引发异常。异常对象未被处理(或捕获)时,程序将终止并显示一条错误消息(traceback)。

\verb|1 / 0  # ZeroDivisionError: division by zero|

如果异常只能用来显示错误消息,就没多大意思了。但事实上,每个异常都是某个类(这里是ZeroDivisionError)的实例。你能以各种方式引发和捕获这些实例,从而逮住错误并采取措施,而不是放任整个程序失败。
\section{让事情能够沿着你指定的轨道出错}
出现问题时,将自动引发异常。先来看看如何自主地引发异常,还有如何创建异常,然后再学习如何处理这些异常。
\subsection{raise语句}
要引发异常,可使用raise语句,并将一个类(必须是Exception的子类)或实例作为参数。将类作为参数时,将自动创建一个实例。

\begin{pyc}
raise Exception  # Exception:
raise Exception("hyperdrive overload")  # Exception: hyperdrive overload
\end{pyc}
在第一个示例(raise Exception)中,引发的是通用异常,没有指出出现了什么错误。在第二个示例中,添加了错误消息hyperdrive overload。

有很多内置的异常类,\autoref{builtinexceptionclasses}描述了最重要的几个,这些异常类都可用于raise语句中。

\begin{table}
    \centering
    \caption{一些内置的异常类}
    \label{builtinexceptionclasses}
    \begin{tabular}{ll}
        \hline
        类 名               & 描 述                              \\
        \hline
        Exception         & 几乎所有的异常类都是从它派生而来的                \\
        AttributeError    & 引用属性或给它赋值失败时引发                   \\
        OSError           & 操作系统不能执行指定的任务(如打开文件)时引发,有多个子类    \\
        IndexError        & 使用序列中不存在的索引时引发,为LookupError的子类   \\
        KeyError          & 使用映射中不存在的键时引发,为LookupError的子类    \\
        NameError         & 找不到名称(变量)时引发                     \\
        SyntaxError       & 代码不正确时引发                         \\
        TypeError         & 将内置操作或函数用于类型不正确的对象时引发            \\
        ValueError        & 将内置操作或函数用于这样的对象时引发:其类型正确但包含的值不合适 \\
        ZeroDivisionError & 在除法或求模运算的第二个参数为零时引发              \\
        \hline
    \end{tabular}
\end{table}
\subsection{自定义的异常类}
虽然内置异常涉及的范围很广,能够满足很多需求,但有时你可能想自己创建异常类。可基于异常所属的类选择性地处理异常。因此,就必须有一个专门用于表示这些异常的类。

那么如何创建异常类呢?就像创建其他类一样,但务必直接或间接地继承Exception(这意味着从任何内置异常类派生都可以)。因此,自定义异常类的代码类似于下面这样:
\begin{pyc}
class SomeCustomException(Exception):
    pass
\end{pyc}
当然,如果你愿意,也可在自定义异常类中添加方法。
\section{捕获异常}
异常比较有趣的地方是可对其进行处理,通常称之为捕获异常。为此,可使用try/except语句。

\begin{pyc}
x = int(input('Enter the first number: '))
y = int(input('Enter the second number: '))
print(x / y)
\end{pyc}
这个程序运行正常,直到用户输入的第二个数为零。为捕获这种异常并对错误进行处理,可像下面这样重写这个程序:

\begin{pyc}
try:
    x = int(input('Enter the first number: '))
    y = int(input('Enter the second number: '))
    print(x / y)
except ZeroDivisionError:
    print("The second number can't be zero.")
\end{pyc}
使用一条if语句来检查y的值好像简单些,就本例而言,这可能也是更佳的解决方案。然而,如果这个程序执行的除法运算更多,则每个除法运算都需要一条if语句,而使用try/except的话只需要一个错误处理程序。

\begin{tcolorbox}
异常从函数向外传播到调用函数的地方。如果在这里也没有被捕获,异常将向程序的最顶层传播。这意味着你可使用try/except来捕获他人所编写函数引发的异常。
\end{tcolorbox}
\subsection{不用提供参数}
捕获异常后,如果要重新引发它(即继续向上传播),可调用raise且不提供任何参数(也可显式地提供捕获到的异常)。

为说明这很有用,来看一个能够“抑制”异常ZeroDivisionError的计算器类。如果启用了这种功能,计算器将打印一条错误消息,而不让异常继续传播。在与用户交互的会话中使用这个计算器时,抑制异常很有用;但在程序内部使用时,引发异常是更佳的选择(此时应关闭“抑制”功能)。下面是这样一个类的代码:

\begin{pyc}
class MuffledCalculator:
    muffled = False

    def calc(self, expr):
        try:
            return eval(expr)
        except ZeroDivisionError:
            if self.muffled:
                print('Division by zero is illegal')
            else:
                raise
\end{pyc}

\begin{tcolorbox}
发生除零行为时,如果启用了“抑制”功能,方法calc将(隐式地)返回None。换而言之,如果启用了“抑制”功能,就不应依赖返回值。
\end{tcolorbox}

下面的示例演示了这个类的用法(包括启用和关闭了抑制功能的情形):

\begin{pyc}
calculator = MuffledCalculator()
calculator.calc('10 / 2')  # 5.0
# calculator.calc('10 / 0')  # ZeroDivisionError: division by zero
calculator.muffled = True
calculator.calc('10 / 0')  # Division by zero is illegal
ret = calculator.calc('10 / 0')
print(ret)  # None
\end{pyc}

如果无法处理异常,在except子句中使用不带参数的raise通常是不错的选择,但有时你可能想引发别的异常。在这种情况下,导致进入except子句的异常将被作为异常上下文存储起来,并出现在最终的错误消息中,如下所示:
\begin{pyc}
try:
    1/0
except ZeroDivisionError:
    raise ValueError
\end{pyc}

你可使用raise ... from ...语句来提供自己的异常上下文,也可使用None来禁用上下文。
\begin{pyc}
try:
    1/0
except ZeroDivisionError:
    raise ValueError from None
\end{pyc}

\subsection{多个except子句}
在前面的输入两个数做除法的案例中,如果用户输入的是字符串,那么这样的异常回导致程序运行失败。该程序中的except子句只捕获ZeroDivisionError异常。为同时捕获这种异常,可在try/except语句中再添加一个except子句。
\begin{pyc}
try:
    x = int(input('Enter the first number: '))
    y = int(input('Enter the second number: '))
except ZeroDivisionError:
    print("The second number can't be zero!")
except ValueError:
    print("That wasn't a number, was it?")
\end{pyc}
\subsection{一箭双雕}
如果要使用一个except子句捕获多种异常,可在一个元组中指定这些异常:
\begin{pyc}
try:
    x = int(input('Enter the first number: '))
    y = int(input('Enter the second number: '))
except (ZeroDivisionError, ValueError, TypeError):
    print('Your numbers were bogus...')
\end{pyc}
当然,仅仅打印错误消息帮助不大。另一种解决方案是不断地要求用户输入数字,直到能够执行除法运算为止。

在except子句中,异常两边的圆括号很重要。一种常见的错误是省略这些括号,这可能导致你不想要的结果。
\subsection{捕获对象}
要在except子句中访问异常对象本身,可使用两个而不是一个参数。(请注意,即便是在你捕获多个异常时,也只向except提供了一个参数——一个元组。)

下面的示例程序打印发生的异常并继续运行:
\begin{pyc}
try:
    x = int(input('Enter the first number: '))
    y = int(input('Enter the second number: '))
except (ZeroDivisionError, ValueError, TypeError) as e:
    print(e)
\end{pyc}
在这个程序中,except子句也捕获两种异常,但由于你同时显式地捕获了对象本身,因此可将其打印出来,让用户知道发生了什么情况。
\subsection{一网打尽}
即使程序处理了好几种异常,还是可能有一些漏网之鱼。然而,如果你就是要使用一段代码捕获所有的异常,只需在except语句中不指定任何异常类即可。

\begin{pyc}
try:
    x = int(input('Enter the first number: '))
    y = int(input('Enter the second number: '))
except:
    print('Something wrong happened...')
\end{pyc}

像这样捕获所有的异常很危险,因为这不仅会隐藏你有心理准备的错误,还会隐藏你没有考虑过的错误。这还将捕获用户使用Ctrl + C终止执行的企图、调用函数sys.exit来终止执行的企图等。在大多数情况下,更好的选择是使用except Exception as e并对异常对象进行检查。这样做将让不是从Exception派生而来的为数不多的异常成为漏网之鱼,其中包括SystemExit和KeyboardInterrupt,因为它们是从BaseException(Exception的超类)派生而来的。

\subsection{万事大吉时}
在有些情况下,在没有出现异常时执行一个代码块很有用。为此,可像条件语句和循环一样,给try/except语句添加一个else子句。
\begin{pyc}
try:
    print('A simple example')
except:
    print('What? Something went wrong ... ')
else:
    print('Ah ... It went as planned.')
# A simple example
# Ah ... It went as planned.
\end{pyc}
通过使用else子句,可以实现在用户输入错误参数时不断地再次输入。
\begin{pyc}
while True:
    try:
        x = int(input('Enter the first number: '))
        y = int(input('Enter the second number: '))
        value = x / y
        print('x / y is', value)
    except:
        print('Invalid input. Please try again.')
    else:
        break
\end{pyc}
在这里,仅当没有引发异常时,才会跳出循环。换而言之,只要出现错误,程序就会要求用户提供新的输入。

一种更佳的替代方案是使用空的except子句来捕获所有属于类Exception(或其子类)的异常。你不能完全确定这将捕获所有的异常,因为try/except语句中的代码可能使用旧式的字符串异常或引发并非从Exception派生而来的异常。然而,如果使用\verb|except Exception as e|,就可以在这个小型除法程序中打印更有用的错误消息。
\begin{pyc}
while True:
    try:
        x = int(input('Enter the first number: '))
        y = int(input('Enter the second number: '))
        value = x / y
        print('x / y is', value)
    except Exception as e:
        print('Invalid input.', e)
        print('Please try again.')
    else:
        break
# Invalid input. division by zero
# Please try again.
# Invalid input. invalid literal for int() with base 10: 'a'
# Please try again.
# x / y is 0.5
\end{pyc}
\subsection{finally}
最后,还有finally子句,可用于在发生异常时执行清理工作。这个子句是与try子句配套的。
\begin{pyc}
x = None
try:
    x = 1 / 0 
finally:
    print('Cleaning up...')
    del x
\end{pyc}
在上述示例中,不管try子句中发生什么异常,都将执行finally子句。为何在try子句之前初始化x呢?因为如果不这样做,ZeroDivisionError将导致根本没有机会给它赋值,进而导致在finally子句中对其执行del时引发未捕获的异常。

如果运行这个程序,它将在执行清理工作后崩溃。

虽然使用del来删除变量是相当愚蠢的清理措施,但finally子句非常适合用于确保文件或网络套接字等得以关闭。

也可在一条语句中同时包含try、except、finally和else(或其中的3个)。
\begin{pyc}
try:
    1 / 0
except ZeroDivisionError:
    print('Unknown variable')
else:
    print('That went well!')
finally:
    print('Cleaning up...')
\end{pyc}

\section{异常和函数}
异常和函数有着天然的联系。如果不处理函数中引发的异常,它将向上传播到调用函数的地方。如果在那里也未得到处理,异常将继续传播,直至到达主程序(全局作用域)。如果主程序中也没有异常处理程序,程序将终止并显示栈跟踪消息。来看一个示例:
\begin{pyc}
def faulty():
    raise Exception('Something went wrong!')
def ignore_exception():
    faulty()
def handle_exception():
    try:
        faulty()
    except:
        print('Exception handled.')
faulty()  # Exception: Something went wrong!
handle_exception()  # Exception handled.
\end{pyc}
faulty中引发的异常依次从faulty和\verb|ignore_exception|向外传播,最终导致显示一条栈跟踪消息。调用\verb|handle_exception|时,异常最终传播到\verb|handle_exception|,并被这里的try/except语句处理。
\section{异常之禅}
如果你知道代码可能引发某种异常,且不希望出现这种异常时程序终止并显示栈跟踪消息,可添加必要的try/except或try/finally语句(或结合使用)来处理它。

有时候,可使用条件语句来达成异常处理实现的目标,但这样编写出来的代码可能不那么自然,可读性也没那么高。另一方面,有些任务使用if/else完成时看似很自然,但实际上使用try/except来完成要好得多。下面来看两个示例。

\begin{tcolorbox}[title=假设有一个字典,你要在指定的键存在时打印与之相关联的值,否则什么都不做。]
\begin{pyc}
def describe_person(person):
    print('Description of', person['name'])
    print('Age: ', person['age'])
    if 'occupation' in person:
        print('Occupation: ', person['occupation'])
person = {'name': 'John', 'age': 42}
describe_person(person)
# Description of John
# Age:  42
\end{pyc}

这段代码很直观,但效率不高(虽然这里的重点是代码简洁),因为它必须两次查找'occupation'键:一次检查这个键是否存在(在条件中),另一次获取这个键关联的值,以便将其打印出来。下面是另一种解决方案:

\begin{pyc}
def describe_person(person):
    print('Description of', person['name'])
    print('Age: ', person['age'])
    try:
        print('Occupation: ', person['occupation'])
    except:
        pass
describe_person(person)
\end{pyc}
\end{tcolorbox}

你可能发现,检查对象是否包含特定的属性时,try/except也很有用。例如,假设你要检查一个对象是否包含属性write,可使用类似于下面的代码:
\begin{pyc}
try:
    obj.write
except AttributeError:
    print('The object is not writeable')
else:
    print('The object is writeable')
\end{pyc}

请注意,这里在效率方面的提高并不大(实际上是微乎其微)。一般而言,除非程序存在性能方面的问题,否则不应过多考虑这样的优化。关键是在很多情况下,相比于使用if/else,使用try/except语句更自然,也更符合Python的风格。因此你应养成尽可能使用try/except语句的习惯\footnote{海军少将Grace Hopper有句至理名言:请求宽恕比获得允许更容易。这解释了Python偏向于使用try/except的原因。这种策略可总结为习语“闭眼就跳”——直接去做,有问题再处理,而不是预先做大量的检查。}。
\section{不那么异常的情况}
如果你只想发出警告,指出情况偏离了正轨,可使用模块warnings中的函数warn。

\begin{pyc}
from warnings import warn
warn("I've got a bad feeling about this.")
# UserWarning: I've got a bad feeling about this.
\end{pyc}

如果其他代码在使用你的模块,可使用模块warnings中的函数filterwarnings来抑制你发出的警告(或特定类型的警告),并指定要采取的措施,如"error"或"ignore"。
\begin{pyc}
from warnings import filterwarnings
filterwarnings('ignore')
warn('Anyone out there?')

filterwarnings('error')
warn('Something is very wrong!')
# UserWarning: Something is very wrong!
\end{pyc}
引发的异常为UserWarning。发出警告时,可指定将引发的异常(即警告类别),但必须是Warning的子类。如果将警告转换为错误,将使用你指定的异常。

另外,还可根据异常来过滤掉特定类型的警告。
\begin{pyc}
from warnings import warn
from warnings import filterwarnings
filterwarnings('error')
warn('This function is really old...', DeprecationWarning)
# DeprecationWarning: This function is really old...

filterwarnings('ignore', category=DeprecationWarning)
warn('Another deprecation warning', DeprecationWarning)
warn('Something else')  # UserWarning: Something else
\end{pyc}

除上述基本用途外,模块warnings还提供了一些高级功能。如果你对此感兴趣,请参阅库参考手册。
\section{小结}

本章介绍了如下重要主题。
\begin{dinglist}{43}
\item 异常对象:异常情况(如发生错误)是用异常对象表示的。对于异常情况,有多种处理方式;如果忽略,将导致程序终止。
\item 引发异常:可使用raise语句来引发异常。它将一个异常类或异常实例作为参数,但你也可提供两个参数(异常和错误消息)。如果在except子句中调用raise时没有提供任何参数,它将重新引发该子句捕获的异常。
\item 自定义的异常类:你可通过从Exception派生来创建自定义的异常。
\item 捕获异常:要捕获异常,可在try语句中使用except子句。在except子句中,如果没有指定异常类,将捕获所有的异常。你可指定多个异常类,方法是将它们放在元组中。如果向except提供两个参数,第二个参数将关联到异常对象。在同一条try/except语句中,可包含多个except子句,以便对不同的异常采取不同的措施。
\item else子句:除except子句外,你还可使用else子句,它在主try块没有引发异常时执行。
\item finally:要确保代码块(如清理代码)无论是否引发异常都将执行,可使用try/finally,并将代码块放在finally子句中。
\item 异常和函数:在函数中引发异常时,异常将传播到调用函数的地方(对方法来说亦如此)。
\item 警告:警告类似于异常,但(通常)只打印一条错误消息。你可指定警告类别,它们是Warning的子类。
\end{dinglist}

\subsection{一些新函数}
\begin{table}[H]
    \centering
    \begin{tabularx}{\textwidth}{lX}
        \hline
        函 数                                                          & 描 述    \\
        \hline
        \verb|warnings.filterwarnings(action,category=Warning, ...)| & 用于过滤警告 \\
        \verb|warnings.warn(message, category=None)|                 & 用于发出警告 \\
        \hline
    \end{tabularx}
\end{table}