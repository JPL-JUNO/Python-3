\chapter{再谈抽象\label{chapter07}}
创建自定义对象(尤其是对象类型或类)是一个Python核心概念。事实上,这个概念非常重要,以至于Python与Smalltalk、C++、Java等众多语言一样,被视为一种面向对象的语言。

\section{对象魔法}
在面向对象编程中,术语\important{对象大致意味着一系列数据(属性)以及一套访问和操作这些数据的方法}。

下面列出了使用对象的最重要的好处。
\begin{dinglist}{43}
    \item 多态:可对不同类型的对象执行相同的操作,而这些操作就像“被施了魔法”一样能够正常运行。
    \item 封装:对外部隐藏有关对象工作原理的细节。
    \item 继承:可基于通用类创建出专用类。
\end{dinglist}

\subsection{多态}
术语多态(polymorphism)源自希腊语,意思是``有多种形态”。\tips{这大致意味着即便你不知道变量指向的是哪种对象,也能够对其执行操作,且操作的行为将随对象所属的类型(类)而异}。
\subsection{多态和方法}
你收到一个对象,却根本不知道它是如何实现的——它可能是众多``形态”中的任何一种。与对象属性相关联的函数称为方法。

\subsubsection{多态形式多样}
每当无需知道对象是什么样的就能对其执行操作时,都是多态在起作用。这不仅仅适用于方法,我们还通过内置运算符和函数大量使用了多态。
\begin{pyc}
1 + 2  # 3
'Fish ' + 'license'  # 'Fish license'
\end{pyc}

如果要编写一个函数,通过打印一条消息来指出对象的长度,可以像下面这样做(它对参数的唯一要求是有长度,可对其执行函数len)。

\begin{pyc}
def length_message(x):
    print('The length of', repr(x), 'is', len(x))
length_message('Hello, world!')  # The length of 'Hello, world!' is 13
length_message([1, 2, 3])  # The length of [1, 2, 3] is 3
\end{pyc}

repr是多态的集大成者之一,可用于任何对象。

要破坏多态,唯一的办法是使用诸如type、issubclass等函数显式地执行类型检查,但你应尽可能避免以这种方式破坏多态。

\begin{tcolorbox}
    这里讨论的多态形式是Python编程方式的核心,有时称为鸭子类型。鸭子类型(英语:duck typing)在程序设计中是动态类型的一种风格。在这种风格中,一个对象有效的语义,不是由继承自特定的类或实现特定的接口,而是由“当前方法和属性的集合”决定。在鸭子类型中,关注点在于对象的行为,能做什么;而不是关注对象所属的类型。例如,在不使用鸭子类型的语言中,我们可以编写一个函数,它接受一个类型为“鸭子”的对象,并调用它的“走”和“叫”方法。在使用鸭子类型的语言中,这样的一个函数可以接受一个任意类型的对象,并调用它的“走”和“叫”方法。如果这些需要被调用的方法不存在,那么将引发一个运行时错误。任何拥有这样的正确的“走”和“叫”方法的对象都可被函数接受的这种行为引出了以上表述,这种决定类型的方式因此得名。有关鸭子类型的详细信息,请参阅\href{http://en.wikipedia.org/wiki/Duck_typing}{鸭子类型}。
\end{tcolorbox}

\subsection{封装}
封装(encapsulation)指的是向外部隐藏不必要的细节。这听起来有点像多态(无需知道对象的内部细节就可使用它)。这两个概念很像,因为它们都是抽象的原则。它们都像函数一样,可帮助你处理程序的组成部分,让你无需关心不必要的细节。

\important{
    但封装不同于多态。多态让你无需知道对象所属的类(对象的类型)就能调用其方法,而封装让你无需知道对象的构造就能使用它。
}

属性是归属于对象的变量,就像方法一样。实际上,方法差不多就是与函数相关联的属性。

对象的状态由其属性(如名称)描述。对象的方法可能修改这些属性,因此对象将一系列函数(方法)组合起来,并赋予它们访问一些变量(属性)的权限,而属性可用于在两次函数调用之间存储值。

\subsection{继承}
继承是另一种偷懒的方式(这里是褒义)。如果你已经有了一个类,并要创建一个与之很像的类(可能只是新增了几个方法),该如何办呢?创建这个新类时,你不想复制旧类的代码,将其粘贴到新类中。

\section{类}
\subsection{类到底是什么}
前面反复提到了类,并将其用作类型的同义词。从很多方面来说,这正是类的定义------一种对象。每个对象都属于特定的类,并被称为该类的实例。一个类的对象为另一个类的对象的子集时,前者就是后者的子类,后者是前者的超类。

\begin{tcolorbox}
    在英语日常交谈中,使用复数来表示类,如birds(鸟类)和larks(云雀)。在Python中,约定使用单数并将首字母大写,如Bird和Lark。
\end{tcolorbox}

通过这样的陈述,子类和超类就很容易理解。但在面向对象编程中,子类关系意味深长,因为类是由其支持的方法定义的。类的所有实例都有该类的所有方法,因此子类的所有实例都有超类的所有方法。有鉴于此,要定义子类,只需定义多出来的方法(还可能重写一些既有的方法)。
\subsection{创建自定义类}
\begin{pyc}
class Person():
    def set_name(self, name):
        self.name = name

    def get_name(self):
        return self.name

    def greet(self):
        print("Hello, world! I'm {}".format(self.name))
\end{pyc}
这个示例包含三个方法定义,它们类似于函数定义,但位于class语句内。Person当然是类的名称。class语句创建独立的命名空间,用于在其中定义函数(参见7.2.5节)。一切看起来都挺好,但你可能想知道参数self是什么。它指向对象本身。那么是哪个对象呢?下面通过创建两个实例来说明这一点。
\begin{pyc}
foo = Person()
bar = Person()
foo.set_name('Luke Skywalker')
bar.set_name('Anakin Skywalker')
foo.greet()  # Hello, world! I'm Luke Skywalker.
bar.greet()  # Hello, world! I'm Anakin Skywalker.
\end{pyc}
个示例可能有点简单,但澄清了self是什么。对foo调用\verb|set_name|和greet时,foo都会作为第一个参数自动传递给它们。将这个参数命名为self,这非常贴切。实际上,可以随便给这个参数命名,但鉴于它总是指向对象本身,因此习惯上将其命名为self。

显然,self很有用,甚至必不可少。如果没有它,所有的方法都无法访问对象本身------要操作的属性所属的对象。

\begin{tcolorbox}
    如果foo是一个Person实例,可将foo.greet()视为Person.greet(foo)的简写,但后者的多态性更低。
\end{tcolorbox}

\subsection{属性、函数和方法}
方法(更准确地说是关联的方法)将其第一个参数关联到它所属的实例,因此无需提供这个参数。无疑可以将属性关联到一个普通函数,但这样就没有特殊的self参数了。

\begin{pyc}
class Class:
    def method(self):
        print("I have a self!")
def function():
    print("I don't...")
instance = Class()
instance.method()  # I have a self!
instance.method = function
instance.method()  # I don't...
\end{pyc}
\important{请注意,有没有参数self并不取决于是否以刚才使用的方式(如instance.method)调用方法}。实际上,完全可以让另一个变量指向同一个方法。
\begin{pyc}
    class Bird():
    song = 'Squaawk!'
    def sing(self):
        print(self.song)
bird = Bird()
bird.sing()  # Squaawk!
birdsong = bird.sing
birdsong()  # Squaawk!
\end{pyc}
虽然最后一个方法调用看起来很像函数调用,但变量birdsong指向的是关联的方法bird.sing,这意味着它也能够访问参数self(即它也被关联到类的实例)。

\subsection{再谈隐藏}
默认情况下,可从外部访问对象的属性。有些程序员认为这没问题,但有些程序员(如Smalltalk之父)认为这违反了封装原则。他们认为应该对外部完全隐藏对象的状态(即不能从外部访问它们)。

关键是其他程序员可能不知道(也不应知道)对象内部发生的情况。为避免这类问题,可将属性定义为私有。私有属性不能从对象外部访问,而只能通过存取器方法(如\verb|get_name|和\verb|set_name|)来访问。

Python没有为私有属性提供直接的支持,而是要求程序员知道在什么情况下从外部修改属性是安全的。毕竟,你必须在知道如何使用对象之后才能使用它。

要让方法或属性成为私有的(不能从外部访问),只需让其名称以两个下划线打头即可。
\begin{pyc}
class Secretive():
    def __inaccessible(self):
        print("Bet you can't see me...")

    def accessible(self):
        self.__inaccessible()
s = Secretive()
s.__inaccessible  # 'Secretive' object has no attribute '__inaccessible'
s.accessible()  # Bet you can't see me...
\end{pyc}
现在从外部不能访问\verb|__inaccessible|,但在类中(如accessible中)依然可以使用它。

\important{虽然以两个下划线打头有点怪异,但这样的方法类似于其他语言中的标准私有方法。然而,幕后的处理手法并不标准:在类定义中,对所有以两个下划线打头的名称都进行转换,即在开头加上一个下划线和类名}。

\verb|s._Secretive__inaccessible()  # Bet you can't see me...|

总之,你无法禁止别人访问对象的私有方法和属性,但这种名称修改方式发出了强烈的信号,让他们不要这样做。

如果你不希望名称被修改,又想发出不要从外部修改属性或方法的信号,可用一个下划线打头。这虽然只是一种约定,但也有些作用。例如,\verb|from module import *|不会导入以一个下划线打头的名称\footnote{对于成员变量(属性),有些语言支持多种私有程度。例如,Java支持4种不同的私有程度。Python没有提供这样的支持,不过从某种程度上说,以一个和两个下划线打头相当于两种不同的私有程度。}。

\subsection{类的命名空间}
在class语句中定义的代码都是在一个特殊的命名空间(类的命名空间)内执行的,而类的所有成员都可访问这个命名空间。类定义其实就是要执行的代码段,并非所有的Python程序员都知道这一点,但知道这一点很有帮助。

\begin{pyc}
class MemberCounter:
    members = 0
    def init(self):
        MemberCounter.members += 1
m1 = MemberCounter()
m1.init()
MemberCounter.members  # 1
m2 = MemberCounter()
m2.init()
MemberCounter.members  # 2
\end{pyc}
上述代码在类作用域内定义了一个变量,所有的成员(实例)都可访问它,这里使用它来计算类实例的数量。

每个实例都可访问这个类作用域内的变量,就像方法一样。如果你在一个实例中给属性members赋值,结果将如何呢?
\begin{pyc}
m1.members = "Two"
m1.members  # 'Two'
m2.members  # 2
\end{pyc}
新值被写入m1的一个属性中,这个属性遮住了类级变量。
\subsection{指定超类}
要指定超类,可在class语句中的类名后加上超类名,并将其用圆括号括起。
\begin{pyc}
class Filter():
    def init(self):
        self.blocked = []
    def filter(self, sequence):
        return [x for x in sequence if x not in self.blocked]
class SPAMFilter(Filter):  # SPAMFilter is a subclass of Filter
    def init(self):  # Overrides init method from Filter superclass
        self.blocked = ['SPAM']

f = Filter()
f.init()
f.filter([1, 2, 3])  # [1, 2, 3]

s = SPAMFilter()
s.init()
s.filter(["SPAM", "SPAM", "bacon", "SPAM", "SPAM", "Eggs"])
# ['bacon', 'Eggs']
\end{pyc}
Filter是一个过滤序列的通用类。实际上,它不会过滤掉任何东西。Filter类的用途在于可用作其他类(如将'SPAM'从序列中过滤掉的SPAMFilter类)的基类(超类)。

请注意SPAMFilter类的定义中有两个要点。
\begin{dinglist}{43}
    \item 以提供新定义的方式重写了Filter类中方法init的定义。
    \item 直接从Filter类继承了方法filter的定义,因此无需重新编写其定义。
\end{dinglist}
第二点说明了继承很有用的原因:可以创建大量不同的过滤器类,它们都从Filter类派生而来,并且都使用已编写好的方法filter。

\subsection{深入探讨继承}
\begin{itemize}
    \item 要确定一个类是否是另一个类的子类,可使用内置方法\verb|issubclass|。
    \item 如果你有一个类,并想知道它的基类,可访问其特殊属性\verb|__bases__|。
    \item 要确定对象是否是特定类的实例,可使用\verb|isinstance|。
    \item 如果你要获悉对象属于哪个类,可使用属性\verb|__class__|。
\end{itemize}

\begin{pyc}
issubclass(SPAMFilter, Filter)  # True
issubclass(Filter, SPAMFilter)  # False

SPAMFilter.__bases__  # (__main__.Filter,)
Filter.__bases__  # (object,)

s = SPAMFilter()
isinstance(s, SPAMFilter)  # True
isinstance(s, Filter)  # True
isinstance(s, str)  # False
\end{pyc}
\begin{tcolorbox}
    使用isinstance通常不是良好的做法,依赖多态在任何情况下都是更好的选择。一个重要的例外情况是使用抽象基类和模块abc时。
\end{tcolorbox}

\subsection{多个超类}
你肯定注意到了一个有点奇怪的细节:复数形式的\verb|__bases__|。前面说过,你可使用它来获悉类的基类,而基类可能有多个。
\begin{pyc}
class Calculator:
    def calculate(self, expression):
        self.value = eval(expression)
class Talker:
    def talk(self):
        print('Hi, my value is', self.value)
class TalkingCalculator(Calculator, Talker):
    pass
\end{pyc}
子类TalkingCalculator本身无所作为,其所有的行为都是从超类那里继承的。关键是通过从Calculator那里继承calculate,并从Talker那里继承talk,它成了会说话的计算器。
\begin{pyc}
tc = TalkingCalculator()
tc.calculate('1 + 2 * 3')
tc.talk()  # Hi, my value is 7
\end{pyc}
这被称为多重继承,是一个功能强大的工具。然而,除非万不得已,否则应避免使用多重继承,因为在有些情况下,它可能带来意外的``并发症”。

使用多重继承时,有一点务必注意:如果多个超类以不同的方式实现了同一个方法(即有多个同名方法),必须在class语句中小心排列这些超类,因为位于前面的类的方法将覆盖位于后面的类的方法。因此,在前面的示例中,如果Calculator类包含方法talk,那么这个方法将覆盖Talker类的方法talk(导致它不可访问)。

\subsection{接口和内省}
接口这一概念与多态相关。处理多态对象时,你只关心其接口(协议)——\important{\textbf{对外}暴露的方法和属性}。

通常,你要求对象遵循特定的接口(即实现特定的方法),但如果需要,也可非常灵活地提出要求:不是直接调用方法并期待一切顺利,而是检查所需的方法是否存在;如果不存在,就改弦易辙。
\begin{pyc}
hasattr(tc, 'talk')  # True
hasattr(tc, 'fnord')  # False
\end{pyc}

\begin{pyc}
callable(getattr(tc, 'talk', None))  # True
callable(getattr(tc, 'fnord', None))  # False
\end{pyc}
请注意,这里没有在if语句中使用hasattr并直接访问属性,而是使用了getattr(它让我能够指定属性不存在时使用的默认值,这里为None),然后对返回的对象调用callable。
\begin{tcolorbox}
    setattr与getattr功能相反,可用于设置对象的属性:
    \begin{pyc}
setattr(tc, 'name', 'Mr. Gumby')
tc.name  # 'Mr. Gumby'
    \end{pyc}
\end{tcolorbox}

要查看对象中存储的所有值,可检查其\verb|__dict__|属性。如果要确定对象是由什么组成的,应研究模块inspect。这个模块主要供高级用户创建对象浏览器(让用户能够以图形方式浏览Python对象的程序)以及其他需要这种功能的类似程序。

\subsection{抽象基类}
然而,有比手工检查各个方法更好的选择。在历史上的大部分时间内,Python几乎都只依赖于鸭子类型,即假设所有对象都能完成其工作,同时偶尔使用hasattr来检查所需的方法是否存在。很多其他语言(如Java和Go)都采用显式指定接口的理念,而有些第三方模块提供了这种理念的各种实现。最终,Python通过引入模块abc提供了官方解决方案。这个模块为所谓的抽象基类提供了支持。一般而言,\notes{抽象类是不能(至少是不应该)实例化的类,其职责是定义子类应实现的一组抽象方法}。
\begin{pyc}
from abc import ABC, abstractmethod
class Talker(ABC):
    @abstractmethod
    def talk(self):
        pass
\end{pyc}
形如@this的东西被称为装饰器,这里的要点是你使用@abstractmethod来将方法标记为抽象的——\important{在子类中必须实现的方法}。抽象类(即包含抽象方法的类)最重要的特征是不能实例化。\\
\verb|Talker()  # TypeError: Can't instantiate abstract class Talker with abstract method talk|

假设像下面这样从它派生出一个子类:
\begin{pyc}
class Knigget(Talker):
    pass
\end{pyc}
由于没有重写方法talk,因此这个类也是抽象的,不能实例化。然而,你可重新编写这个类,使其实现要求的方法。
\begin{pyc}
class Knigget(Talker):
    def talk(self):
        print("Ni!")
\end{pyc}

现在实例化它没有任何问题。\tips{这是抽象基类的主要用途,而且只有在这种情形下使用isinstance才是妥当的:如果先检查给定的实例确实是Talker对象,就能相信这个实例在需要的情况下有方法talk}。

\begin{pyc}
k = Knigget()
isinstance(k, Talker)  # True
k.talk()  # Ni!
\end{pyc}

然而,还缺少一个重要的部分——让isinstance的多态程度更高的部分。正如你看到的,\notes{抽象基类让我们能够本着鸭子类型的精神使用这种实例检查!我们不关心对象是什么,只关心对象能做什么(它实现了哪些方法)}。因此,只要实现了方法talk,即便不是Talker的子类,依然能够通过类型检查。

下面来创建另一个类:
\begin{pyc}
class Herring:
    def talk(self):
        print('Blub.') 
\end{pyc}

这个类的实例能够通过是否为Talker对象的检查,但是检查的结果:它并不是Talker对象。
\begin{pyc}
h = Herring()
isinstance(h, Talker)  # False
\end{pyc}

诚然,你可从Talker派生出Herring,这样就万事大吉了,但Herring可能是从他人的模块中导入的。在这种情况下,就无法采取这样的做法。为解决这个问题,你可将Herring注册为Talker(而不从Herring和Talker派生出子类),这样所有的Herring对象都将被视为Talker对象。
\begin{pyc}
Talker.register(Herring)  # __main__.Herring
isinstance(h, Talker)  # True
issubclass(Herring, Talker)  # True
\end{pyc}

然而,这种做法存在一个缺点,就是直接从抽象类派生提供的保障没有了。
\begin{pyc}
class Clam:
    pass
Talker.register(Clam)  # __main__.Clam
issubclass(Clam, Talker)  # True
c = Clam()
isinstance(c, Talker)  # True
c.talk()  # AttributeError: 'Clam' object has no attribute 'talk'
\end{pyc}
换而言之,\important{应将isinstance返回True视为一种意图表达}。在这里,Clam有成为Talker的意图。本着鸭子类型的精神,我们相信它能承担Talker的职责,但可悲的是它失败了(\notes{这里是Clam没有重写方法,使用register跳过了重写的要求,否则根本不能实例化})。

标准库(如模块collections.abc)提供了多个很有用的抽象类,有关模块abc的详细信息,请参阅标准库参考手册。

\section{关于面向对象设计的一些思考}
专门探讨面向对象程序设计的图书很多,虽然这并非本书的重点,但还是要提供一些指南。
\begin{dinglist}{43}
    \item 将相关的东西放在一起。如果一个函数操作一个全局变量,最好将它们作为一个类的属性和方法。
    \item 不要让对象之间过于亲密。方法应只关心其所属实例的属性,对于其他实例的状态,让它们自己去管理就好了。
    \item 慎用继承,尤其是多重继承。继承有时很有用,但在有些情况下可能带来不必要的复杂性。要正确地使用多重继承很难,要排除其中的bug更难。
    \item 保持简单。让方法短小紧凑。一般而言,应确保大多数方法都能在30秒内读完并理解。对于其余的方法,尽可能将其篇幅控制在一页或一屏内。
\end{dinglist}

确定需要哪些类以及这些类应包含哪些方法时,尝试像下面这样做。
\begin{enumerate}
    \item 将有关问题的描述(程序需要做什么)记录下来,并给所有的名词、动词和形容词加上标记。
    \item 在名词中找出可能的类。
    \item 在动词中找出可能的方法。
    \item 在形容词中找出可能的属性。
    \item 将找出的方法和属性分配给各个类。
\end{enumerate}

有了面向对象模型的草图后,还需考虑类和对象之间的关系(如继承或协作)以及它们的职责。为进一步改进模型,可像下面这样做。
\begin{enumerate}
    \item 记录(或设想)一系列用例,即使用程序的场景,并尽力确保这些用例涵盖了所有的功能。
    \item 透彻而仔细地考虑每个场景,确保模型包含了所需的一切。如果有遗漏,就加上;如果有不太对的地方,就修改。不断地重复这个过程,直到对模型满意为止。
\end{enumerate}

\section{小结}
本章不仅介绍了有关Python语言的知识,还介绍了多个你可能一点都不熟悉的概念。下面来总结一下。
\begin{dinglist}{43}
    \item 对象:对象由属性和方法组成。属性不过是属于对象的变量,而方法是存储在属性中的函数。相比于其他函数,(关联的)方法有一个不同之处,那就是它总是将其所属的对象作为第一个参数,而这个参数通常被命名为self。
    \item 类:类表示一组(或一类)对象,而每个对象都属于特定的类。类的主要任务是定义其实例将包含的方法。
    \item 多态:多态指的是能够同样地对待不同类型和类的对象,即无需知道对象属于哪个类就可调用其方法。
    \item  封装:对象可能隐藏(封装)其内部状态。在有些语言中,这意味着对象的状态(属性)只能通过其方法来访问。在Python中,所有的属性都是公有的,但直接访问对象的状态时程序员应谨慎行事,因为这可能在不经意间导致状态不一致。
    \item  继承:一个类可以是一个或多个类的子类,在这种情况下,子类将继承超类的所有方法。你可指定多个超类,通过这样做可组合正交(独立且不相关)的功能。为此,一种常见的做法是使用一个核心超类以及一个或多个混合超类。
    \item  接口和内省:一般而言,你无需过于深入地研究对象,而只依赖于多态来调用所需的方法。然而,如果要确定对象包含哪些方法或属性,有一些函数可供你用来完成这种工作。
    \item  抽象基类:使用模块abc可创建抽象基类。抽象基类用于指定子类必须提供哪些功能,却不实现这些功能。
    \item  面向对象设计:关于该如何进行面向对象设计以及是否该采用面向对象设计,有很多不同的观点。无论你持什么样的观点,都必须深入理解问题,进而创建出易于理解的设计。
\end{dinglist}