\chapter{快速上手:基础知识}
\href{www.python.org}{根据系统下载适配的Python版本}
\section{交互式解释器}
如果你熟悉其他计算机语言,可能习惯了在每行末尾都加上分号。在Python中无需这样做,
因为在Python中,一行就是一行。如果你愿意,也可加上分号,但不会有任何影响(除非后面还
有其他代码),况且大家通常都不这样做。
\section{算法是什么}
算法只不过是流程或菜谱的时髦说法,详尽地描述了如何完成某项任务。
\section{数和表达式}
所有常见算术运算符的工作原理都与你预期的一致。除法
运算的结果为小数,即浮点数(float或floating-point number)。如果你想丢弃小数部分,即执行整除运算,可使用双斜杠。

$x~\%~ y$的结果为$x$除以$y$的余数。换而言之,结果为执行整除时余下的部分,即$x~\%~ y$等价于$x~-~ ((x ~//~ y)~ *~ y)$。

求余运算符也可用于浮点数,甚至可用于负数,但可能不那么好理解。

对于整除运算,需要明白的一个重点是它向下圆整结果。因此在结果为负数的情况下,圆整后将离0更远。

乘方运算符的优先级比求负(单目减)高,因此\verb|-3 ** 2|等价于\verb|-(3 ** 2)|。如果你要计算的是\verb|(-3) ** 2|,必须明确指出。

\subsection{十六进制、八进制和二进制}
十六进制数、八进制数和二进制数分别以下面的方式表示, 这些表示法都以0打头。:
\begin{pyc}
0xAf # 175
0o1000 # 512
0b1010100111 # 679
\end{pyc}
\section{变量}
变量是表示(或指向)特定值的名称。例如,你可能想使用名称x来表示3,\verb|x = 3|,这称为赋值(assignment),我们将值3赋给了变量x。换而言之,就是将变量x与值(或对象)
3关联起来。

不同于其他一些语言,使用Python变量前必须给它赋值,因为Python变量没有默认值。
\warning{
    在Python中,名称(标识符)只能由字母、数字和下划线构成,且不能以数字打头。
}
\section{语句}
表达式是一些东西,而语句做一些事情。涉及赋值时,语句和表达式的差别更明显:鉴于赋值语句不是表达式,它们没有可供交互式
解释器打印的值。

执行赋值语句后,交互式解释器只是再次显示提示符,但发生了一些变化:有一个名为x的新变量,与值3相关联。可以说,这是所有语句的一个根本特征:执行修改操作。例如,赋值语
句改变变量,而print语句改变屏幕的外观。
\section{获取用户输入}
你编写的程序可能供他人使用,无法预测用户会向程序提供什么样的值。我们来看看很有用的函数\verb|input|:
\begin{pyc}
input("The meaning of life: ")
x = input("x: ")
y = input("y: ")
print(int(x) * int(y))
\end{pyc}
\section{函数}
使用乘方运算符(**)来执行幂运算。实际上,可不使用这个运算符,而使用函数pow。我们通常将pow等标
准函数称为内置函数。
\begin{pyc}
>>> 2 ** 3
8
>>> pow(2, 3)
8
\end{pyc}
\section{模块}
\section{保存并执行程序}
\section{字符串}
\section{小结}